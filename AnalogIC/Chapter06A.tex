\section{频率响应的基本概念}

\subsection{频率响应与零点和极点}
我们可能会这样抱怨:MOS管已经足够复杂了!再考虑那些到处变来变去的电容可真是没完没了!但事实是,我们只需要了解最基本的RC电路的频率响应就够了。\xref{fig:低通滤波}和\xref{fig:高通滤波}是我们在电路基础中都非常熟悉的低通滤波和高通滤波电路,两者的系统函数很容易求出来。

\begin{Figure}[基本RC电路]
    \begin{FigureSub}[低通滤波电路]
        \includegraphics[scale=0.8]{build/Chapter06A_03.fig.pdf}
    \end{FigureSub}
    \hspace{1cm}
    \begin{FigureSub}[低通滤波电路]
        \includegraphics[scale=0.8]{build/Chapter06A_04.fig.pdf}
    \end{FigureSub}
\end{Figure}

对于低通滤波,输出是电容$C$上的分压
\begin{Equation}[低通滤波]
    A=\frac{V_{out}}{V_{in}}=\frac{(sC)^{-1}}{R+(sC)^{-1}}=\frac{1}{1+sRC}
\end{Equation}
对于高通滤波,输出是电阻$R$上的分压
\begin{Equation}[高通滤波]
    A=\frac{V_{out}}{V_{in}}=\frac{R}{R+(sC)^{-1}}=\frac{sRC}{1+sRC}
\end{Equation}
上述计算我们很熟悉,但这一次,我们将换一个角度来考虑问题。我们知道,拉普拉斯域上的函数$H(s)$是一个复变函数,在数学物理方法中,曾提及过复变函数的极点和零点的定义。

\begin{itemize}
    \item 复变函数$H(s)$的极点,是指那些令该复变函数$H(s)$取值为无穷大的点$s=s_p$
    \item 复变函数$H(s)$的零点,是指那些令该复变函数$H(s)$取值为零的点$s=s_z$
\end{itemize}

低通滤波电路具有一个极点$\omega_p$
\begin{Equation}[低通滤波零极点]
    \omega_p=-1/RC
\end{Equation}
高通滤波电路具有一个极点$\omega_p$和一个位于原点处的零点$\omega_z$
\begin{Equation}[高通滤波零极点]
    \omega_p=-1/RC\qquad \omega_z=0
\end{Equation}

\begin{Figure}[基本RC电路的系统函数可视化]
    \begin{FigureSub}[低通滤波的函数图]
        \includegraphics[scale=0.8]{build/Chapter06A_02c.fig.pdf}
    \end{FigureSub}
    \hspace{1cm}
    \begin{FigureSub}[高通滤波的函数图]
        \includegraphics[scale=0.8]{build/Chapter06A_02d.fig.pdf}
    \end{FigureSub}
\end{Figure}


那么,为何我们如此关心系统函数的极点和零点呢?\xref{fig:低通滤波的幅频响应}和\xref{fig:高通滤波的幅频响应}是低通滤波和高通滤波的幅频响应曲线。而事实上,极点和零点直接决定了幅频响应曲线的特性。有这样的断言
\begin{itemize}
    \item 每遇到一个极点,幅频曲线将以$\SI{20}{dB.dec^{-1}}$下降(变化率减小$\SI{20}{dB.dec^{-1}}$)。
    \item 每遇到一个零点,幅频曲线将以$\SI{20}{dB.dec^{-1}}$上升(变化率增大$\SI{20}{dB.dec^{-1}}$)。
\end{itemize}

\begin{Figure}[基本RC电路的幅频响应]
    \begin{FigureSub}[低通滤波的幅频响应]
        \includegraphics[scale=0.8]{build/Chapter06A_02a.fig.pdf}
    \end{FigureSub}
    \begin{FigureSub}[高通滤波的幅频响应]
        \includegraphics[scale=0.8]{build/Chapter06A_02b.fig.pdf}
    \end{FigureSub}
\end{Figure}

幅频响应常用$\si{dB}$作为增益$A$的单位,这是一种对数单位,定义为$20\log A$,换言之,一个数量级就对应了$20\si{dB}$。例如,\xref{fig:基本RC电路的幅频响应}的纵坐标没有采用$\si{dB}$,而是普通的对数坐标,但很容易将两者对应起来:$10^0$即$\SI{0}{dB}$,$10^{-1}$即$\SI{-20}{dB}$,$10^{-2}$即$\SI{-40}{dB}$,依此类推。让我们回到上面的断言,这样的提法其实很让人困惑,首先为什么是$\SI{20}{dB.dec^{-1}}$这样很突兀的数字,其次如\xref{fig:基本RC电路的零极点图}的零极点图(这类图中,极点标识为叉,零点标识为圈)所示,极点$\omega_p$和零点$\omega_z$都分布在实轴$\sigma$上(尽管用$\omega$标识),而探讨频率响应时,频率$\omega$的变化却反映为虚轴上的移动,那么,频率沿着虚轴增加又何谈“遇到”实轴上的零点和极点呢?这是需要我们解决的。

\begin{Figure}[基本RC电路的零极点图]
    \begin{FigureSub}[低通滤波的零极点图]
        \includegraphics[scale=1]{build/Chapter06A_05.fig.pdf}
    \end{FigureSub}
    \hspace{1cm}
    \begin{FigureSub}[高通滤波的零极点图]
        \includegraphics[scale=1]{build/Chapter06A_06.fig.pdf}
    \end{FigureSub}
\end{Figure}

以低通滤波为例,在\xrefeq{低通滤波}中依据\xrefeq{低通滤波零极点}的$\omega_p=-1/RC$代换\setpeq{20dB解释}
\begin{Equation}&[1]
    A=\frac{1}{1-s/\omega_p}
\end{Equation}
讨论频率响应时,令$s=\j\omega$
\begin{Equation}&[2]
    A=\frac{1}{1-\j\omega/\omega_p}
\end{Equation}
求幅值
\begin{Equation}&[3]
    |A|=\frac{1}{\sqrt{1+(\omega/\omega_p)^2}}
\end{Equation}
我们注意到,\xrefpeq{3}的意义是,幅值是频率点$\j\omega$至极点$\omega_p$的距离与$\omega_p$的比值的倒数。
\begin{itemize}
    \item 当$\omega\ll|\omega_p|$时,此时$|A|=1$,因为两者的距离近似等于$\omega_p$。
    \item 当$\omega\hspace{0.027cm}=\hspace{0.027cm}|\omega_p|$时,此时$|A|=\sqrt{2}/2$,因为两者与原点构成了等腰直角三角形,斜边是直角边的$\sqrt{2}$倍,注意到$20 \log\sqrt{2}/2=\SI{-3}{dB}$,故频率达到极点频率时,出现$\SI{-3}{dB}$点。
    \item 当$\omega\gg |\omega_p|$时,由于$(\omega/\omega_p)^2\gg 1$,此时$|A|\propto\omega^{-1}$,换言之,频率增加一个数量级,幅值减小一个数量级,幅值习惯上以$\si{dB}$为单位,而一个数量级对应$\SI{20}{dB}$,故有频率遇到极点后幅频曲线将以$\SI{20}{dB.dec^{-1}}$的速率下降的提法。通过上述讨论我们可以看出,极点$\omega_p$位于实轴,频率点$\j\omega$位于虚轴,因此,这里所谓的“遇到”并不是频率点触及了极点,而是指两者到原点距离的相同,即$\omega=|\omega_p|$。另外,遇到极点后以$\SI{20}{dB.dec^{-1}}$的速率下降本身也是近似的提法,从\xref{fig:低通滤波的幅频响应}或上一段讨论中都可以看出,遇到极点时幅度已经减小至原先的$\sqrt{2}/2$倍或$\SI{-3}{dB}$处了,但是,$\SI{20}{dB.dec^{-1}}$的线性近似在趋势上仍然是相当准确的,只需知道零点和极点的位置,就可以基于此大致的画出幅频曲线。
\end{itemize}
高通滤波和低通滤波很相似,但区别在于,高通滤波多了一个零点$\omega_z$,且该零点还恰好位于原点$\omega_z=0$处。这意味着两点:当$\omega=0$时幅度为零(感觉有些奇怪?请不要忘记$\omega$同样是以对数坐标画出的,最左端不是零),当$\omega>0$时,零点代表的$+\SI{20}{dB.dec^{-1}}$的增长立即生效,直至$\omega>\omega_p$时,极点生效,极点$-\SI{20}{dB.dec^{-1}}$的下降与零点$+\SI{20}{dB.dec^{-1}}$的上升相互抵消,幅频曲线因此保持恒定。至此,我们将零点和极点与幅频曲线的变化关联起来了。

我们常考虑四类情况:左极点(LHP),右极点(RHP),左零点(LHZ),右零点(RHZ)。这里的LH和RH代表的是左半复平面Left Half和右半复平面Right Half,而P和Z则代表极点(Poles)和零点(Zeros)。区分左右的原因是相频响应。首先,无论左、右、零点、极点
\begin{itemize}
    \item 当$\omega\hspace{0.027cm}=\hspace{0.027cm}|\omega_p|,|\omega_z|$时,将产生$\pm \pi/4$的相移。
    \item 当$\omega\gg|\omega_p|,|\omega_z|$时,将产生$\pm \pi/2$的相移。
\end{itemize}
我们也可以直接从\xref{fig:低通滤波的零极点图}这样的零极点图中通过“频率点--零极点--原点”的夹角直观的看出相移的绝对值,但问题在于,相移到底是正还是负,这就与左和右有关
\begin{itemize}
    \item 极点导致$-\SI{20}{dB.dec^{-1}}$的幅频响应,左极点导致负相移,右极点导致正相移。
    \item 零点导致$-\SI{20}{dB.dec^{-1}}$的幅频响应,左零点导致正相移,右零点导致负相移。
\end{itemize}

\xref{tab:四种情况的幅频响应和相频响应}系统总结了上述特性
\begin{Tablex}[四种情况的幅频响应和相频响应]{|Y|Y|Y|Y|}
<左极点LHP&右极点RHP&左零点LHZ&右零点RHZ\\>
\xcell<Y>[2ex][-3ex]{\includegraphics[scale=1.2]{build/Chapter06A_07.fig.pdf}}&
\xcell<Y>[2ex][-3ex]{\includegraphics[scale=1.2]{build/Chapter06A_08.fig.pdf}}&
\xcell<Y>[2ex][-3ex]{\includegraphics[scale=1.2]{build/Chapter06A_09.fig.pdf}}&
\xcell<Y>[2ex][-3ex]{\includegraphics[scale=1.2]{build/Chapter06A_10.fig.pdf}}\\
\xcell<Y>[2ex][-3ex]{\includegraphics[scale=0.8]{build/Chapter06A_02e.fig.pdf}}&
\xcell<Y>[2ex][-3ex]{\includegraphics[scale=0.8]{build/Chapter06A_02f.fig.pdf}}&
\xcell<Y>[2ex][-3ex]{\includegraphics[scale=0.8]{build/Chapter06A_02g.fig.pdf}}&
\xcell<Y>[2ex][-3ex]{\includegraphics[scale=0.8]{build/Chapter06A_02h.fig.pdf}}\\
\xcell<Y>[2ex][-3ex]{\includegraphics[scale=0.8]{build/Chapter06A_02i.fig.pdf}}&
\xcell<Y>[2ex][-3ex]{\includegraphics[scale=0.8]{build/Chapter06A_02j.fig.pdf}}&
\xcell<Y>[2ex][-3ex]{\includegraphics[scale=0.8]{build/Chapter06A_02k.fig.pdf}}&
\xcell<Y>[2ex][-3ex]{\includegraphics[scale=0.8]{build/Chapter06A_02l.fig.pdf}}\\
\end{Tablex}

\subsection{频率响应与放大器}
现在让我们回到放大器,\xref{fig:放大器的频率响应模型}是一个考虑放大器的频率响应的模型,其中$A$是一个理想的全通放大器,其前的$R_1,C_1$和其后的$R_2,C_2$分别构成两个独立的低通滤波电路,这些电阻和电容通常不是故意添加的,可能来自,放大器本身的输入输出电阻、放大器中MOS管的寄生电容、输入内阻、输出负载电阻等。根据之前的知识,我们可以很容易的写出系统函数为
\begin{Equation}
    A(s)=A_0\cdot\frac{1}{1-s/\omega_1}\frac{1}{1-s/\omega_2}
\end{Equation}
其中$\omega_1$和$\omega_2$分别是
\begin{Equation}
    \omega_1=-1/R_1C_1\qquad \omega_2=-1/R_2C_2
\end{Equation}
这是一个很重要的想法:放大器中的一个节点代表了一个极点的存在。

\begin{Figure}[放大器的频率响应模型]
    \includegraphics[scale=0.8]{build/Chapter06A_11.fig.pdf}
\end{Figure}

\subsection{频率响应与米勒定理}
当然,并非所有放大器都恰好是\xref{fig:放大器的频率响应模型}这种最简单的形式,很多时候,完整的电路分析仍然是必要的。\xref{fig:放大器节点间的跨间电容}就是一个不符合\xref{fig:放大器的频率响应模型}的例子,两个节点间的跨接电容$C_3$使我们无法处理。

\begin{Figure}[放大器节点间的跨间电容]
    \includegraphics[scale=0.8]{build/Chapter06A_12.fig.pdf}
\end{Figure}

米勒定理提供了一种手段,使我们可以将节点间的跨接阻抗转换为对地阻抗,如\xref{fig:米勒定理}所示。

\begin{Figure}[米勒定理]
    \begin{FigureSub}[等效前]
        \includegraphics[scale=0.8]{build/Chapter06A_13.fig.pdf}
    \end{FigureSub}
    \hspace{1cm}
    \begin{FigureSub}[等效后]
        \includegraphics[scale=0.8]{build/Chapter06A_14.fig.pdf}
    \end{FigureSub}
\end{Figure}

如\xref{fig:等效前}所示,理想放大器$A$上不通电流,但$V_1,V_2$节点间又有电势差,因此必然有电流$I$流过$Z$,对于$V_1$节点有电流$I$流出,对于$V_2$节点有电流$I$流入。而阻抗等效的关键就在于,令等效的接地阻抗$Z_1,Z_2$能模仿出上述“$V_1$有电流$I$流出,$V_2$有电流$I$流入”。\setpeq{米勒等效}

首先,电流$I$依据等效前的电路应当表示为
\begin{Equation}&[1]
    I=\frac{V_1-V_2}{Z}
\end{Equation}
等效后,阻抗$Z_1$上流过的电流$I$为(由$V_1$至地)
\begin{Equation}&[2]
    I=+\frac{V_1}{Z_1}
\end{Equation}
等效后,阻抗$Z_2$上流过的电流$I$为(由地至$V_2$)
\begin{Equation}&[3]
    I=-\frac{V_2}{Z_2}
\end{Equation}
联立\xrefpeq{1}和\xrefpeq{2},可以解出$Z_1$
\begin{Equation}
    \frac{V_1-V_2}{Z}=
    \frac{V_1}{Z_1}\qquad
    Z_1=\frac{ZV_1}{V_1-V_2}=\frac{Z}{1-V_2/V_1}=\frac{Z}{1-A}
\end{Equation}
联立\xrefpeq{1}和\xrefpeq{3},可以解出$Z_2$
\begin{Equation}
    \frac{V_2-V_1}{Z}=
    \frac{V_2}{Z_2}\qquad
    Z_2=\frac{ZV_2}{V_2-V_1}=\frac{Z}{1-V_1/V_2}=\frac{Z}{1-A^{-1}}
\end{Equation}
在多数情况下,由于$A\gg 1$,输出等效阻抗$Z_2$往往被进一步近似为$Z$。
\begin{BoxTheorem}[米勒定理]
    \uwave{米勒定理}(Miller Theorem)是指,跨接阻抗$Z$可以转换为对地阻抗$Z_1,Z_2$
    \begin{Equation}
        Z_1=\frac{1}{1-A}\qquad Z_2=\frac{Z}{1-A^{-1}}\approx Z
    \end{Equation}
\end{BoxTheorem}

应当指出的是,尽管上述分析相当严密,但实践中,米勒定理实质上具有近似属性。这是因为我们假定了$V_{out}/V_{in}$恒等于$A$,然而,跨接阻抗$Z$势必对$V_{out}/V_{in}$存在影响,米勒定理的等效则使这种影响被近似忽略,从而造成不准确的结果。米勒定理能否适用要取决于实际情况。\goodbreak

在实践中,米勒定理面对的浮动阻抗,往往是纯电阻或纯电容(尤其是纯电容),因此,直观的感受增益的变化对于等效结果的影响是有必要的,如\xref{fig:米勒定理的等效结果}所示,当增益的绝对值$|A|$较大时
\begin{itemize}
    \item 对于跨接电阻$R$,输入$R_1$趋近于$0$,输出$R_2$趋近于$R$。
    \item 对于跨接电容$C$,输入$C_1$等于$C(1-A)$,输出$C_2$趋近于$C$。
\end{itemize}
\begin{Figure}[米勒定理的等效结果]
    \begin{FigureSub}[电阻的米勒等效]
        \includegraphics[scale=0.8]{build/Chapter06A_01a.fig.pdf}
    \end{FigureSub}
    \begin{FigureSub}[电容的米勒等效]
        \includegraphics[scale=0.8]{build/Chapter06A_01b.fig.pdf}
    \end{FigureSub}
\end{Figure}