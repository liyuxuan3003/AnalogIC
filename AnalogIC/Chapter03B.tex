\section{共漏级放大器}
作为一个放大器,我们希望其具有以下三点性质
\begin{itemize}
    \item 增益$A_V$尽可能大。
    \item 输出电阻相较负载电阻要尽可能小,即$R_{out}$应较小。
    \item 输入电阻相较信号内阻要尽可能大,即$R_{in}$应较大。
\end{itemize}
共源级放大器满足了高增益和高输入电阻的要求,然而,共源级放大器输出电阻较大,当其连接低阻负载时,将会损失增益。共漏级放大器可以很好的弥补这种缺点,它并不能提供很高增益,事实上其增益通常是略小于$1$的,忠实的在漏端输出源端输入的电压,因此,共漏放大器也被称为\uwave{源跟随器}。但是,共漏放大器的输出电阻很小,故可以在输出端作为缓冲级使用。

\subsection{采用电阻负载的共漏级}
共漏级的电路如\xref{fig:采用电阻负载的共漏级电路}所示,尽管共漏级和共源级性质差异很大,但两者的结构很相似
\begin{itemize}
    \item 共源级放大器,栅极输入,源极输出,负载$R_D$在漏极,如\xref{fig:使用电阻}所示。
    \item 共漏极放大器,栅极输入,漏极输出,负载$R_S$在源极,如\xref{fig:采用电阻负载的共漏级电路}所示。
\end{itemize}
虽然$R_S$也出现在了共源级中,但是,$R_S$在共源级中只是负反馈,$R_S$在共漏极中却是负载。\goodbreak

现在我们来分析一下,为什么\xref{fig:采用电阻负载的共漏级电路}的电路可以实现源跟随的功能?\nopagebreak

\begin{Figure}[采用电阻负载的共漏级电路]
    \includegraphics[scale=0.8]{build/Chapter03B_03.fig.pdf}
\end{Figure}
这种组态下,代换关系是
\begin{Equation}
    V_{GS}=V_{in}-V_{out}\qquad V_{DS}=V_{DD}-V_{out}\qquad V_{BS}=-V_{out}
\end{Equation}

建立方程,假定$M_1$处于饱和区,通过电流建立方程
\begin{Equation}
    \mu_nC_{ox}(W/L)[(V_{in}-V_{out}-V_{TH})^2/2]=V_{out}/R_S
\end{Equation}
两边同乘$R_S$,并对$V_{in}$求导得到,注意$R_S$的存在使体效应必须被考虑
\begin{Equation}
    \mu_nC_{ox}(W/L)(V_{in}-V_{out}-V_{TH})\qty(1-\pdv{V_{out}}{V_{in}}-\pdv{V_{TH}}{V_{in}})R_S=\pdv{V_{out}}{V_{in}}
\end{Equation}
应用链式法则,第一项给出$-\eta$,第二项中$V_{BS}=-V_{out}$
\begin{Equation}
    \pdv{V_{TH}}{V_{in}}=\pdv{V_{TH}}{V_{BS}}\cdot\pdv{V_{BS}}{V_{in}}=\eta\pdv{V_{out}}{V_{in}}
\end{Equation}
基于此,我们就得到
\begin{Equation}
    A_V=\pdv{V_{out}}{V_{in}}=\frac{\mu_nC_{ox}(W/L)(V_{in}-V_{out}-V_{TH})R_S}{1+\mu_nC_{ox}(W/L)(V_{in}-V_{out}-V_{TH})R_S(1+\eta)}
\end{Equation}
我们知道
\begin{Equation}
    g_{m}=\mu_nC_{ox}(W/L)(V_{in}-V_{out}-V_{TH})\qquad g_{mb}=\eta g_m
\end{Equation}
因此有
\begin{Equation}
    A_V=\frac{g_mR_s}{1+(g_m+g_{mb})R_S}
\end{Equation}
很明显,这是略一个小于$1$的增益,这也就是共漏放大器作为源跟随器的根源。\goodbreak

值得注意的是,即便$R_S\to\infty$,增益也不会达到$1$,这是体效应的结果
\begin{Equation}
    A_V=\frac{g_m}{g_m+g_{mb}}=\frac{1}{1+\eta}
\end{Equation}
整理如下
\begin{BoxFormula}[采用电阻负载的共漏极增益]
    共漏极放大,采用电阻负载,增益为
    \begin{Equation}
        A_V=\frac{g_mR_S}{1+(g_m+g_{mb})R_S}
    \end{Equation}
    若令$R_S\to\infty$,简化为
    \begin{Equation}
        A_V=\frac{g_m}{g_m+g_{mb}}=\frac{1}{1+\eta}
    \end{Equation}
\end{BoxFormula}


在\xref{fig:采用电阻负载的共漏级增益}中,我们绘制了共漏放大器的电压特性和增益特性。容易看出,较大的漏极负载可以使增益更快也更加接近$1$。值得注意的是,共漏放大器只会工作于饱和区,不存在线性区。
\begin{Figure}[采用电阻负载的共漏级增益]
    \begin{FigureSub}[电压图;电压图采用电阻负载的共漏级增益]
        \includegraphics[scale=0.82]{build/Chapter03B_01a.fig.pdf}
    \end{FigureSub}
    \begin{FigureSub}[增益图;增益图采用电阻负载的共漏级增益]
        \includegraphics[scale=0.82]{build/Chapter03B_01b.fig.pdf}
    \end{FigureSub}
\end{Figure}

接下来,我们分析一下小信号模型,完整起见,考虑沟道调制,如\xref{fig:采用电阻负载的共漏级小信号电路}所示,请注意,现在输出$V_{out}$位于源极处,并且由于漏极直接连接了电源,在小信号中等效于D和B短接了。
\begin{Figure}[采用电阻负载的共漏级小信号电路]
    \includegraphics[scale=0.8]{build/Chapter03B_04.fig.pdf}
\end{Figure}

直接计算增益是可能的,但是我们还是想沿袭\xref{subsec:带源极负反馈的共源放大}中的分析方法。

\subsubsection{计算共漏放大等效转移电导}
计算等效转移电导,对于\xref{fig:采用电阻负载的共漏级小信号电路},将输出端口短接,如\xref{fig:计算共漏放大等效转移电导}所示
\begin{Figure}[计算共漏放大等效转移电导]
    \includegraphics[scale=0.8]{build/Chapter03B_05.fig.pdf}
\end{Figure}
列写代换关系
\begin{Equation}
    V_{GS}=V_{in}\qquad
    V_{BS}=0\qquad
    V_{DS}=0
\end{Equation}
有趣的是,由于S,D,B间均被短接了,有$V_{DS}=V_{BS}=0$,因此
\begin{Equation}
    I_{out}+g_mV_{GS}=0
\end{Equation}
即
\begin{Equation}
    I_{out}+g_mV_{in}=0
\end{Equation}
这就得到
\begin{Equation}
    G_m=\frac{I_{out}}{V_{in}}=-g_m
\end{Equation}
这里不必为负值结果感到惊讶,由于$A_V=-G_m(R_{O}\parallel R_S)$,负的等效跨导意味着正增益。

\begin{BoxFormula}[共漏放大器的等效转移电导]
    共漏极放大,等效转移电导为
    \begin{Equation}
        G_m=-g_m
    \end{Equation}
\end{BoxFormula}

\subsubsection{计算共漏放大等效输出电导}
计算等效输出电导,对于\xref{fig:采用电阻负载的共漏级小信号电路},将输入端口短接,如\xref{fig:计算共漏放大等效输出电导}所示
\begin{Figure}[计算共漏放大等效输出电导]
    \includegraphics[scale=0.8]{build/Chapter03B_06.fig.pdf}
\end{Figure}
列写代换关系
\begin{Equation}
    V_{GS}=-V_{out}\qquad V_{BS}=-V_{out}\qquad V_{DS}=-V_{out}
\end{Equation}
有趣的是,各端口的电压都等于$-V_{out}$,这意味着共漏下$g_{m},g_{mb},r_{O}$都可等效为电阻
\begin{Equation}
    I_{out}+g_mV_{GS}+g_{m}V_{BS}+r_{O}^{-1}V_{DS}
\end{Equation}
即
\begin{Equation}
    I_{out}-g_mV_{out}-g_{mb}V_{out}-r_{O}^{-1}V_{out}=0
\end{Equation}
这就得到
\begin{Equation}
    G_O=\frac{I_{out}}{V_{out}}=g_m+g_{mb}+r_{O}^{-1}
\end{Equation}
这就是说,输出电导是$g_m,g_{mb},r_{O}^{-1}$的和,就好比是它们是三个并联电阻一般。

\begin{BoxFormula}[共漏放大器的等效输出电导]
    共漏级放大,等效输出电导为
    \begin{Equation}
        G_O=g_m+g_{mb}+r_{O}^{-1}
    \end{Equation}
\end{BoxFormula}


\subsubsection{计算共漏放大的增益和输入输出电阻}
再一次,回到增益的计算上
\begin{Equation}
    A_V=-G_m(R_O\parallel R_S)
\end{Equation}
我们同样记
\begin{Equation}
    A_V=-G_mR_{out}
\end{Equation}
这里先来计算$R_{out}=R_O\parallel R_S$
\begin{Equation}
    R_{out}=\frac{1}{g_m+g_{mb}+r_{O}^{-1}+R_S^{-1}}
\end{Equation}
由于$r_{O}^{-1}+R_{S}^{-1}=(r_{O}+R_{S})/(r_{O}R_S)$,不妨上下同乘$r_{O}R_S$
\begin{Equation}
    R_{out}=\frac{r_OR_S}{R_S+r_O\qty[1+(g_m+g_{mb})R_S]}
\end{Equation}

由于$G_m=-g_m$,因此$A_V$只需要在$R_{out}$的基础上乘以$g_m$就可以了,总结如下。

\begin{BoxFormula}[共漏级放大器综述]
    共漏级放大器的增益为
    \begin{Equation}
        A_V=\frac{g_mr_OR_S}{R_S+r_O\qty[1+(g_m+g_{mb})R_S]}
    \end{Equation}
    共漏级放大器的输出电阻为
    \begin{Equation}
        R_{out}=\frac{r_OR_S}{R_S+r_O\qty[1+(g_m+g_{mb})R_S]}
    \end{Equation}
    共漏级放大器的输入电阻为
    \begin{Equation}
        R_{in}=\infty
    \end{Equation}
\end{BoxFormula}
