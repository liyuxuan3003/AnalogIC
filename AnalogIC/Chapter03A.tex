\section{共源级放大器}

共源级放大,就是指,输入信号接至栅极,输出信号接至漏极,而源极作为公共端接地,漏极同时需要接电源$V_{DD}$,然而,漏极和$V_{DD}$间还有一个负载,负载的类型是本节的中心脉络
\begin{itemize}
    \item 采用电阻负载的共源级:负载也可以使用偏置于深线性区的PMOS管。
    \item 采用电流源负载的共源级:负载也可以使用偏置于饱和区的PMOS管。
    \item 采用二极管负载的共源级:这里的二极管指“栅漏短接”的MOS管。
    \item 采用反相器构型的共源级:也称为有源负载,这是因为负载管同时参与了放大过程。
\end{itemize}

\subsection{采用电阻负载的共源级}
共源级放大最常见的负载形式,就是电阻,毕竟电阻看起来最像是“负载”。

\begin{Figure}[采用电阻负载的共源级电路]
    \begin{FigureSub}[使用电阻]
        \includegraphics[scale=0.8]{build/Chapter03A_07.fig.pdf}
    \end{FigureSub}\hspace{1cm}
    \begin{FigureSub}[使用深线性区PMOS]
        \includegraphics[scale=0.8]{build/Chapter03A_12.fig.pdf}
    \end{FigureSub}
\end{Figure}

电阻可以是一个真正的电阻,如\xref{fig:使用电阻}所示,电阻也可以是一个被$V_b$恰当的偏置于深线性区的PMOS管,如\xref{fig:使用深线性区PMOS}所示,按照\fancyref{fml:MOS的输出电导}的结论,其等效为
\begin{Equation}
    R_{D}=-\frac{1}{\mu_pC_{ox}(W/L)_2(V_{b}-V_{DD}-V_{THP})}
\end{Equation}
因此,\xref{fig:使用深线性区PMOS}就可以化归为\xref{fig:使用电阻},故我们之后总是讨论\xref{fig:使用电阻}。

在继续分析之前,我们先需要弄明白我们到底想要分析什么。既然是放大电路,那我们最关心的必然是增益$A_V$。增益听上去是输出和输入的比$A_V=V_{out}/V_{in}$,但不要忘记,我们放大的是小信号,因此,增益正确的定义是输出对输入的导数,即$A_V=\pdv*{V_{out}}{V_{in}}$,记录如下。
\begin{BoxDefinition}[增益]
    增益是输出对输入的导数
    \begin{Equation}
        A_V=\pdv{V_{out}}{V_{in}}
    \end{Equation}
\end{BoxDefinition}

那么,如何计算增益?最朴素(也是最暴力)的思想是,既然增益$A_V=\pdv*{V_{out}}{V_{in}}$,那么只要我们能求出$V_{out}=V_{out}(V_{in})$的函数关系,完成求导计算,增益$A_V$也就被求出来了!\setpeq{电阻负载共源}

\xref{fig:使用电阻}中,$V_{in},V_{out}$和我们熟悉的那些MOS电压的关系是
\begin{Equation}&[1]
    V_{GS}=V_{in}\qquad V_{DS}=V_{out}
\end{Equation}

\xref{fig:使用电阻}的约束是,流经$R_D$的电流和流经$M_1$的电流应当是相等的,当然,首先我们处理一个特殊情况,若$V_{in}<V_{TH}$时,即NMOS截止,此时$I_D=0$电流为零,则$V_{out}=V_{DD}$。

当NMOS处于饱和区时,即$V_{in}<V_{out}+V_{TH}$,根据\fancyref{fml:MOS的理想特性}
\begin{Equation}&[2]
    \mu_nC_{ox}(W/L)[(V_{in}-V_{TH})^2/2]=(V_{DD}-V_{out})/R_D
\end{Equation}

当NMOS处于线性区时,即$V_{in}>V_{out}+V_{TH}$,根据\fancyref{fml:MOS的理想特性}
\begin{Equation}&[3]
    \mu_nC_{ox}(W/L)[(V_{in}-V_{TH})V_{out}-V_{out}^2/2]=(V_{DD}-V_{out})/R_D
\end{Equation}
这两者的分界$V_{in1}$可以由\xrefpeq{2}中,代入$V_{in}=V_{in1}$和$V_{out}=V_{in1}-V_{TH}$求出。

对于\xrefpeq{2},容易解出
\begin{Equation}&[4]
    V_{out}=V_{DD}-\mu_nC_{ox}(W/L)\qty[(V_{in}-V_{TH})^2/2]R_D
\end{Equation}
对于\xrefpeq{3},或许我们会相似的写出
\begin{Equation}&[5]
    V_{out}=V_{DD}-\mu_nC_{ox}(W/L)\qty[(V_{in}-V_{TH})V_{out}-V_{out}^2/2]R_D
\end{Equation}
然而,应当注意到$V_{out}$的表达式中仍然出现了$V_{out}$,这实质是一个方程,不过,介于它是二次方程,求解是可能的,但是其结论相当冗长,其表达式无法带给我们任何知识。不过,计算机是可以完成这种求解的。至此,$V_{out}=V_{out}(Vin)$在截止区、饱和区、线性区的表达式,以及$V_{in}$对于这三个区域的两个分界线$V_{TH},V_{in1}$从理论上都是可求的了,那么增益$A_V$理论上也是可求的了。\xref{fig:采用电阻负载的共源级增益}给出了计算机基于这种想法的求解结果,我们可以得出一些有价值的结论
\begin{itemize}
    \item 随着$R_{D}$的增加,$V_{in1}$在减小,$V_{in}$会更早的进入线性区。
    \item 在电压特性\xref{fig:电压图采用电阻负载的共源级}中,我们看到,$V_{out}$最初为$V_{DD}$,当NMOS导通后,$V_{out}$随$V_{in}$的增加而减小,当$V_{in}$越过$V_{in1}$使NMOS由饱和区进入线性区后,这种减小放缓了。
    \item 在增益特性\xref{fig:增益图采用电阻负载的共源级}中,我们首先注意到,增益是负的,这种反相特性是共源放大的一个基本特征。增益在饱和区随着$V_{in}$线性增大,进入线性区后则迅速回落至零,这也是为何当用于放大时,MOS总是应处于饱和区。同时,也注意到,漏极电阻$R_D$越大,增益越大,增益随$V_{in}$变化越大,增益处于饱和区的$V_{in}$范围越小。这三者分别代表了三个性质:增益、线性、摆幅。而漏极电阻$R_{D}$的取值将在这三个性质间形成Trade Off。
    \item 漏极电阻$R_D$取值较大时,增益较高,线性差,摆幅较小(牺牲线性和摆幅,换取增益)。
    \item 漏极电阻$R_D$取值较小时,增益较小,线性好,摆幅较大(牺牲增益,换取线性和摆幅)。
\end{itemize}

\begin{Figure}[采用电阻负载的共源级增益]
    \begin{FigureSub}[区间图;区间图采用电阻负载的共源级]
        \includegraphics[scale=0.82]{build/Chapter03A_01a.fig.pdf}
    \end{FigureSub}\\ \vspace{0.5cm}
    \begin{FigureSub}[电压图;电压图采用电阻负载的共源级]
        \includegraphics[scale=0.82]{build/Chapter03A_01b.fig.pdf}
    \end{FigureSub}
    \begin{FigureSub}[增益图;增益图采用电阻负载的共源级]
        \includegraphics[scale=0.82]{build/Chapter03A_01c.fig.pdf}
    \end{FigureSub}
\end{Figure}

现在让我们从这些绚丽的图像回到$A_V$的解析式的求解上来。糟糕的是,我们目前只有饱和区的$V_{out}=V_{out}(V_{in})$,不过介于用于放大的MOS管总是处于饱和区,我们实际上也只关心饱和区的增益,线性区的情况我们并不在意,通过\xref{fig:采用电阻负载的共源级增益}这样的图像定性了解就已经足够了。

对于\xrefpeq{4}求导得到
\begin{Equation}&[6]
    A_V=\pdv{V_{out}}{V_{in}}=-\mu_nC_{ox}(W/L)(V_{in}-V_{TH})R_{D}
\end{Equation}
若引用\fancyref{fml:MOS的转移电导}的结论
\begin{Equation}&[7]
    A_V=-g_mR_D
\end{Equation}
这是一个非常有价值的结论,从中我们也可以再次看到增益$A_V$随$R_D$增大的特点。

以上是基于$V_{out}=V_{out}(V_{in})$直接求导得到$A_V$的大信号分析方法,但请不要忘记,我们真正关心的只是增益$A_V$,因此计算$V_{out}=V_{out}(V_{in})$并不是必须的。事实是,我们可以通过小信号电路直接计算得到$A_V$。小信号电路的分析非常简单,且可以帮助我们更好的考虑二阶效应。\goodbreak

\xref{fig:采用电阻负载的共源级小信号电路}给出了\xref{fig:使用电阻}的小信号电路图,比较令人奇怪的是,为什么$R_D$被绘制在了D和B之间,在\xref{fig:使用电阻}中$R_D$很明显在$D$和$V_{DD}$间!这背后是小信号电路绘制的几个重要想法
\begin{itemize}
    \item 电源和地,在小信号电路中都视为接地。
    \item 电压源在小信号电路中短路。
    \item 电流源在小信号电路中开路。
\end{itemize}
\begin{Figure}[采用电阻负载的共源级小信号电路]
    \includegraphics[scale=0.8]{build/Chapter03A_14.fig.pdf}    
\end{Figure}
这里$R_D$连接了D和$V_{DD}$,在小信号上即连接了D和GND,而衬底B就是GND。

在\xref{fig:采用电阻负载的共源级小信号电路}中,由于S和B被短接了,电流$g_mV_{GS}$实质上流经了$r_O$和$R_{D}$的并联电阻,而输出电压$V_{out}$是这电压降的负值,同时,我们也注意到$V_{BS}=0$使得$V_{GS}=V_{in}$,故有
\begin{Equation}&[8]
    V_{out}=-g_mV_{in}\qty(r_{O}\parallel R_D)
\end{Equation}
在小信号下,增益就是简单的输出和输入的比了
\begin{Equation}&[9]
    A_V=\frac{V_{out}}{V_{in}}
\end{Equation}
因此有
\begin{Equation}&[10]
    A_V=-g_m(r_{O}\parallel R_D)
\end{Equation}
若不考虑沟道调制,令$r_O\to\infty$,则有
\begin{Equation}&[11]
    A_V=-g_mR_D
\end{Equation}
这就回到了原先大信号的结论\xrefpeq{7}。总结一下这里得到的结论。

\begin{BoxFormula}[采用电阻负载的共源级]
    共源级放大,采用电阻负载,增益为
    \begin{Equation}
        A_V=-g_m\frac{r_OR_D}{r_O+R_D}
    \end{Equation}
    实质上是(此处的近似考虑到通常$r_O\gg R_D$)
    \begin{Equation}
        A_V=-g_m(r_O\parallel R_D)\approx -g_mR_D
    \end{Equation}
\end{BoxFormula}

至此,让我们回顾一下上述过程中的经验
\begin{enumerate}
    \item 试图手工写出所有工作区的电压特性和增益特性的数学表达式是不切实际的努力,但是这种最直接的思路对于计算机而言是可能的,这可以帮助我们绘制出相当全面的图像。
    \item 大信号的手工分析是可能的,但牢记我们关心的只是饱和区的增益!
    \item 小信号分析是相当简便的,尤其是当需要考虑沟道调制或衬偏调制这类二阶效应时。
    \item 小信号分析的本质是:\empx{用微分近似替代差分}。
\end{enumerate}

\subsection{采用电流源负载的共源级}
实践中,有时我们会需要很高的增益,而\xref{subsec:采用电阻负载的共源级}指出增益$A_V=-g_m(r_O\parallel R_D)$,那么理论上只要令$R_D\to\infty$就可以获得最大增益$A_V=-g_mr_O$,也称为\uwave{本征增益}。但这显然是不可行的,如果负载开路,就没有任何电流了(增大负载电阻会消耗直流压降)。所以,一个更实际的思路是,采用一些不符合欧姆特性的器件,例如用电流源作为负载,如\xref{fig:使用电流源}所示,电流源在小信号中相当于一个无穷大的电阻,但是,电流源在大信号下又能提供用于偏置的直流电流,维持放大器的运作。当然,实际的电流源是通过一个偏置在饱和区的PMOS构成的。

\begin{Figure}[采用电流源负载的共源级电路]
    \begin{FigureSub}[使用电流源]
        \includegraphics[scale=0.8]{build/Chapter03A_08.fig.pdf}
    \end{FigureSub}\hspace{1cm}
    \begin{FigureSub}[使用饱和区PMOS]
        \includegraphics[scale=0.8]{build/Chapter03A_13.fig.pdf}
    \end{FigureSub}
\end{Figure}

既然电流源在小信号下相当于开路,那么增益即为本征增益
\begin{Equation}
    A_V=-g_mr_{O}
\end{Equation}
然而,若考虑到电流源实际由饱和区PMOS构成,其输出电阻也需要被考虑
\begin{Equation}
    A_V=-g_m(r_{O1}\parallel r_{O2})
\end{Equation}
\begin{BoxFormula}[采用电流源负载的共源级]
    共源极放大,采用电流源负载,增益为
    \begin{Equation}
        A_V=-g_m\frac{r_{O1}r_{O2}}{r_{O1}+r_{O2}}
    \end{Equation}
    实质上是
    \begin{Equation}
        A_V=-g_m(r_{O1}\parallel r_{O2})
    \end{Equation}
\end{BoxFormula}
这里\xref{fig:采用电流源负载的共源级小信号电路}绘制了对应的小信号电路,在一定意义上,这就是用$r_{O2}$替换了$R_D$。
\begin{Figure}[采用电流源负载的共源级小信号电路]
    \includegraphics[scale=0.8]{build/Chapter03A_15.fig.pdf}    
\end{Figure}

这里我们再来讨论一下理想电流源的情况,增益为
\begin{Equation}
    A_V=-g_m r_O
\end{Equation}
依照\fancyref{fml:MOS的转移电导}
\begin{Equation}
    g_m=\sqrt{2\mu_nC_{ox}(W/L)I_D}\qquad g_m\propto I_D^{1/2}
\end{Equation}
依照\fancyref{fml:MOS的输出电导}
\begin{Equation}
    g_O=\lambda I_D\qquad r_{O}\propto I_D^{-1}
\end{Equation}
因此,我们可以推断出$A_V\propto I_D^{-1/2}$,这就表明,电流源负载的共源级放大器,电流越大,增益的幅值越小。\xref{fig:采用电流源负载的共源级增益}形象展示了这种趋势,其给出了理想电流源负载下共源放大的电压特性和增益特性。值得注意的是,由于采用了理想电流源,在\xref{fig:电压图采用电流源负载的共源级}中输出会迅速超过$V_{DD}$,这种情况对于通过PMOS饱和区偏置的电流源是不会发生的。故\xref{fig:电压图采用电流源负载的共源级}中观察$A_V\propto I_D^{-1/2}$即增益$A_V$随$I_D$增大而减小的关系时,应当比较各自$V_{in1}$处的$A_V$而非同一$V_{in}$处的$A_V$。\footnote{问题的根源并不是理想电流源,而是在于使用理想电流源,$V_{in}$在小于$V_{in1}$后$V_{out}$会迅速增大(超过$V_{DD}$与否并不重要),这就意味着$V_{DS}$是非常大的,而$g_{O}=\lambda I_D$的关系是通过$g_{O}=\lambda I_D/(1+\lambda V_{DS})$取$\lambda V_{DS}$很小的近似得到的,$V_{DS}$很大时这种关系将被破坏,从而无法再得到$A_V\propto I_D^{-1/2}$的关系。故考察$A_V\propto I_D^{-1/2}$只能在$V_{in1}$附近使$V_{out}=V_{DS}$不太大处进行。}
\begin{Figure}[采用电流源负载的共源级增益]
    \begin{FigureSub}[区间图;区间图采用电流源负载的共源级]
        \includegraphics[scale=0.82]{build/Chapter03A_03a.fig.pdf}
    \end{FigureSub}\\ \vspace{0.5cm}
    \begin{FigureSub}[电压图;电压图采用电流源负载的共源级]
        \includegraphics[scale=0.82]{build/Chapter03A_03b.fig.pdf}
    \end{FigureSub}
    \begin{FigureSub}[增益图;增益图采用电流源负载的共源级]
        \includegraphics[scale=0.82]{build/Chapter03A_03c.fig.pdf}
    \end{FigureSub}
\end{Figure}

\subsection{采用二极管负载的共源级}\setpeq{采用二极管负载的共源级}
这里所说的二极管,并非通常意义上的那种PN结二极管,而是指二极管接法的MOS管,具体而言,\empx{就是将MOS管的栅和漏短接},如\xref{fig:MOS管的二极管接法}所示。这种情况下有$V_{DS}=V_{GS}$成立,而饱和区条件是$V_{DS}>V_{GS}-V_{TH}$,换言之,二极管接法的MOS管将始终在饱和区工作。

\begin{Figure}[MOS管的二极管接法]
    \begin{FigureSub}[NMOS二极管接法]
        \includegraphics{build/Chapter03A_16.fig.pdf}
    \end{FigureSub}
    \hspace{1cm}
    \begin{FigureSub}[PMOS二极管接法]
        \includegraphics{build/Chapter03A_17.fig.pdf}
    \end{FigureSub}
\end{Figure}

\xref{fig:采用二极管负载的共源级电路}给出了二极管负载的共源级电路,负载二极管分别由NMOS和PMOS构成,我们先以NMOS为例来讨论。然而,两个MOS管看起来就很复杂!直接分析是困难的。因此,我们接下来先单独讨论MOS作为二极管负载时的性质,再来考虑其加入共源电路后的影响。

\begin{Figure}[采用二极管负载的共源级电路]
    \begin{FigureSub}[采用NMOS二极管]
        \includegraphics[scale=0.8]{build/Chapter03A_09.fig.pdf}
    \end{FigureSub}\hspace{1cm}
    \begin{FigureSub}[采用PMOS二极管]
        \includegraphics[scale=0.8]{build/Chapter03A_10.fig.pdf}
    \end{FigureSub}
\end{Figure}

\xref{fig:未采用二极管接法}是通常的NMOS对应的小信号电路,而二极管接法的本质就是将MOS管的栅极和漏极短接,\xref{fig:已采用二极管接法}展现了NMOS二极管接法后的小信号电路,它对应了\xref{fig:NMOS二极管接法}的电路。
\begin{Figure}[二极管接法的小信号电路]
    \begin{FigureSub}[未采用二极管接法]
        \includegraphics[scale=0.8]{build/Chapter03A_18.fig.pdf}
    \end{FigureSub}\\ \vspace{0.5cm}
    \begin{FigureSub}[已采用二极管接法]
        \includegraphics[scale=0.8]{build/Chapter03A_19.fig.pdf}
    \end{FigureSub}
\end{Figure}

采用二极管接法后,在小信号下,将产生一个有趣的关系
\begin{Equation}&[1]
    V_{GS}=V_{DS}
\end{Equation}

现在我们考虑在\xref{fig:采用NMOS二极管}中的作为二极管的NMOS管$M_2$的小信号电路,如\xref{fig:二极管接法在共源放大器中的小信号电路}所示,我们注意到,\xref{fig:二极管接法在共源放大器中的小信号电路}相对\xref{fig:已采用二极管接法}的主要区别是,漏端D接地了,这是因为二极管作为共源级放大器的负载时,漏端D连接了$V_{DD}$,在小信号中相当于接地了。\xref{fig:二极管接法在共源放大器中的小信号电路}中的$V_{in}$和$I_{in}$是对于二极管接法的NMOS而言的,$V_{in}$代表其源端的电压,$I_{in}$代表源端至漏端的电流,此处的$V_{in}$与\xref{fig:采用NMOS二极管}中的$V_{in}$无关。我们现在要计算的就是该二极管的等效电阻$V_{in}/I_{in}$的值。

\begin{Figure}[二极管接法在共源放大器中的小信号电路]
    \includegraphics[scale=0.8]{build/Chapter03A_20.fig.pdf}
\end{Figure}

首先,由于D也接地了,\xrefpeq{1}的关系将扩展为
\begin{Equation}
    V_{GS}=V_{DS}=V_{BS}
\end{Equation}
这里$V_{in}$是
\begin{Equation}
    V_{in}=-V_{GS}=-V_{DS}=-V_{BS}
\end{Equation}
这里$I_{in}$则是
\begin{Equation}
    I_{in}=-g_mV_{GS}-g_{mb}V_{BS}-V_{DS}/r_O
\end{Equation}
将电压用$V_{in}$代换
\begin{Equation}
    I_{in}=g_mV_{in}+g_{mb}V_{in}+V_{in}/r_O=(g_m+g_{mb}+r_O^{-1})V_{in}
\end{Equation}
因此有
\begin{BoxFormula}[二极管负载的等效电阻]
    当作为共源极负载时,二极管接法的NMOS管的等效电阻是(可忽略$r_{O}^{-1}$)
    \begin{Equation}
        R_D=\frac{1}{g_m+g_{mb}+r_O^{-1}}\approx\frac{1}{g_m+g_{mb}}
    \end{Equation}
\end{BoxFormula}

这里需要考虑体效应的原因是,\xref{fig:采用NMOS二极管}中$M_2$的源并不是地电位,源和体未短接。

\begin{Figure}[采用二极管负载的共源级小信号电路]
    \includegraphics[scale=0.8]{build/Chapter03A_21.fig.pdf}
\end{Figure}

这样一来,事情就简单很多了。二极管负载,不过就是阻值不太好计算的电阻罢了。引用\fancyref{fml:二极管负载的等效电阻}和\fancyref{fml:采用电阻负载的共源级},忽略沟道调制\setpeq{采用二极管负载的共源级}
\begin{Equation}&[2]
    A_V=-g_{m1}R_D=-g_{m1}\frac{1}{g_{m2}+g_{mb2}}
\end{Equation}
稍作变换
\begin{Equation}&[3]
    A_V=-\frac{g_{m1}}{g_{m2}(1+\eta)}
\end{Equation}
即
\begin{Equation}&[4]
    A_V=-\frac{g_{m1}}{g_{m2}}\frac{1}{1+\eta}
\end{Equation}
这里我们可以再做一些工作,就$g_{m1},g_{m2}$援引\fancyref{fml:MOS的转移电导}
\begin{Equation}&[5]
    A_V=-\frac{\sqrt{2\mu_nC_{ox}(W/L)_1I_D}}{\sqrt{2\mu_n C_{ox}(W/L)_2I_D}}\frac{1}{1+\eta}=-\sqrt{\frac{(W/L)_1}{(W/L)_2}}\frac{1}{1+\eta}
\end{Equation}
这表明,采用二极管负载时,\empx{增益正比于放大管和负载管的尺寸比的平方根}。
\begin{BoxFormula}[采用NMOS二极管负载的共源极增益]
    共源极放大,采用NMOS二极管负载,增益为
    \begin{Equation}
        A_V=-\frac{g_{m1}}{g_{m2}}\frac{1}{1+\eta}
    \end{Equation}
    或进一步表示为
    \begin{Equation}
        A_V=-\sqrt{\frac{(W/L)_1}{(W/L)_2}}\frac{1}{1+\eta}
    \end{Equation}
\end{BoxFormula}

\xref{fig:采用二极管负载的共源级增益}给出了二极管负载的共源放大器的大信号特性,从中我们可以洞见一些小信号分析下无法展现的特征。如\xref{fig:增益图采用二极管负载的共源级}所示,二极管负载下增益在饱和区是常数,这意味着绝对的线性!相比之下,电阻负载下增益在饱和区是线性变化的。由此可见,先前我们说,二极管负载相当于一个特定阻值的电阻负载只对于小信号分析下增益$A_V$的表达式是正确的,两者的增益$A_V$随$V_{in}$是截然不同的。这是值得警惕的,完全脱离大信号特性的小信号分析是危险的!\goodbreak

二极管负载虽然有极好的线性放大特性,但也有其问题
\begin{itemize}
    \item 二极管负载下,输出电压最高只能达到$V_{DD}-V_{TH}$,这是因为NMOS传输强$0$和弱$1$的特性所致,这里负载NMOS管恰好传输了弱$1$,不过这并不太重要,首先,模拟电路不关心电压特性,其次,只要将负载管从NMOS变成PMOS就可以妥当解决该问题。
    \item 二极管负载的主要问题是,其难以实现高增益。增益正比于放大管和负载管的尺寸之比的平方根,这就意味着,提高增益需要以平方速率提高放大管的尺寸,这将导致面积的消耗和大电容的引入。从某种意义上说,以二极管代替电阻,用增益Trade Off了线性。
\end{itemize}
二极管负载下的增益随尺寸比的平方根变化的关系,可以很清楚的在\xref{fig:增益图采用二极管负载的共源级}中观察到。

\begin{Figure}[采用二极管负载的共源级增益]
    \begin{FigureSub}[区间图;区间图采用二极管负载的共源级]
        \includegraphics[scale=0.82]{build/Chapter03A_02a.fig.pdf}
    \end{FigureSub}\\ \vspace{0.5cm}
    \begin{FigureSub}[电压图;电压图采用二极管负载的共源级]
        \includegraphics[scale=0.82]{build/Chapter03A_02b.fig.pdf}
    \end{FigureSub}
    \begin{FigureSub}[增益图;增益图采用二极管负载的共源级]
        \includegraphics[scale=0.82]{build/Chapter03A_02c.fig.pdf}
    \end{FigureSub}
\end{Figure}

若采用PMOS二极管代替NMOS二极管,将发生一些变化
\begin{itemize}
    \item PMOS传输强$1$和弱$0$,故输出电压最高可以达到$V_{DD}$,而不是$V_{DD}-V_{TH}$。
    \item PMOS的源在这种组态下接$V_{DD}$,故不存在体效应。
    \item PMOS的迁移率和NMOS的迁移率不同。
\end{itemize}\setpeq{采用二极管负载的共源级}

在\xrefpeq{4}中,我们应当移除有关体效应的项
\begin{Equation}
    A_V=-\frac{g_{m1}}{g_{m2}}
\end{Equation}
在\xrefpeq{5}中,代入电流后$\mu_n,\mu_p$将分别保留
\begin{Equation}
    A_V=-\frac{\sqrt{2\mu_nC_{ox}(W/L)_1I_D}}{\sqrt{2\mu_pC_{ox}(W/L)_2I_D}}=-\sqrt{\frac{\mu_n(W/L)_1}{\mu_p(W/L)_2}}
\end{Equation}
由于$\mu_n>\mu_p$且不再有体效应,采用PMOS二极管相较NMOS二极管在增益上是有优势的。
\begin{BoxFormula}[采用PMOS二极管负载的共源极增益]
    共源极放大,采用PMOS二极管负载,增益为
    \begin{Equation}
        A_V=-\frac{g_{m1}}{g_{m2}}
    \end{Equation}
    或进一步表示为
    \begin{Equation}
        A_V=-\sqrt{\frac{\mu_n(W/L)_1}{\mu_p(W/L)_2}}
    \end{Equation}
\end{BoxFormula}

\subsection{采用反相器构型的共源极}
采用PMOS器件作为负载引申出这样一种思路,我们能否让PMOS也同步参与放大?答案是肯定的,这种设计被称为有源负载的共源极,如\xref{fig:采用反相器构型的共源级电路}所示,其实质就是数字电路的反相器。
\begin{Figure}[采用反相器构型的共源级电路]
    \includegraphics[scale=0.8]{build/Chapter03A_11.fig.pdf}
\end{Figure}

\xref{fig:采用反相器构型的共源级小信号电路}是\xref{fig:采用反相器构型的共源级电路}对应的小信号电路,这里有个问题,此处G,S,D,B到底是指那一MOS管的电极?事实是,在小信号下,$M_1$和$M_2$管实际共用了所有电极!栅极G和漏极D连接了相同的$V_{in}$和$V_{out}$,源极S和体电极B均连接了电源,因此,结果就是,$M_1$和$M_2$的电流源$g_{m1}V_{GS}$和$g_{m2}V_{GS}$发生并联变为$(g_{m1}+g_{m2})V_{GS}$,协同放大。两者的输出电阻也并联。
\begin{Figure}[采用反相器构型的共源级小信号电路]
    \includegraphics[scale=0.8]{build/Chapter03A_22.fig.pdf}
\end{Figure}

这样一来,反相器构型下的增益就很容易计算了。
\begin{BoxFormula}[采用反相器构型的共源极增益]
    共源极放大,采用反相器构型,增益为
    \begin{Equation}
        A_V=-(g_{m1}+g_{m2})\frac{r_{O1}r_{O2}}{r_{O1}+r_{O2}}
    \end{Equation}
    实质上是
    \begin{Equation}
        A_V=-(g_{m1}+g_{m2})(r_{O1}\parallel r_{O2})
    \end{Equation}
\end{BoxFormula}

\xref{fig:采用反相器构型的共源级增益}给出了反相器的大信号特性
\begin{itemize}
    \item 由于反相器构型中两个MOS管均参与放大,因此工作区的划分会更复杂,随着$V_{in}$的增大,$M_1/M_2$经历:截止/线性、饱和/线性、饱和/饱和、线性/饱和、线性/截止。而在这些阶段中,只有当两个MOS管同时工作在饱和区时,该结构才能被用于放大。
    \item 反相器构型的优点在于,两个MOS管同时参与放大,有利于实现高增益。
    \item 反相器构型的缺点在于,两个MOS管同时处于放大区的范围非常窄,另外,模集电的设计讲究PVT,PVT即Process, Voltage, Temperature,换言之PVT代表了电路对工艺偏差和环境的敏感性。反相器构型对这种偏差很敏感(PVT的强函数),这很不好。
    \item 较大的NMOS尺寸会使曲线向左移动。
    \item 较大的PMOS尺寸会使曲线向右移动。
    \item 当两者尺寸一致时,曲线也会偏左,这是因为两者的载流子迁移率不同。
\end{itemize}

总而言之,尽管反相器对于数字电路非常重要,但在模拟电路中并没有那么适合放大。

\begin{Figure}[采用反相器构型的共源级增益]
    \begin{FigureSub}[区间图;区间图采用反相器构型的共源级增益]
        \includegraphics[scale=0.82]{build/Chapter03A_04a.fig.pdf}
    \end{FigureSub}\\ \vspace{0.5cm}
    \begin{FigureSub}[电压图;电压图采用反相器构型的共源级增益]
        \includegraphics[scale=0.82]{build/Chapter03A_04b.fig.pdf}
    \end{FigureSub}
    \begin{FigureSub}[增益图;增益图采用反相器构型的共源级增益]
        \includegraphics[scale=0.82]{build/Chapter03A_04c.fig.pdf}
    \end{FigureSub}
\end{Figure}

\subsection{带源极负反馈的共源放大}
反馈总是会让一切计算变得复杂!但同时,反馈也是控制的基础!在放大器中,反馈常常以所谓“源极负反馈”的形式存在,具体而言,源极原先都是直接接地的,源极负反馈就是指在源端串联一个负反馈电阻。源极负反馈可以搭配任何负载,但其最常用于电阻负载的共源放大。

\begin{Figure}[带源极负反馈的共源级电路]
    \includegraphics[scale=0.8]{build/Chapter03A_23.fig.pdf}
\end{Figure}

反馈在这里的引入的目的是什么?答案是:源极负反馈电阻可以软化非线性关系。在\xref{fig:采用电阻负载的共源级增益}中我们就注意到,电阻负载会带来非线性,引入源极负反馈就可以在一定程度上减小这种非线性,当然,这种非线性的减小将以增益的损失为代价,这一点在\xref{fig:带源极负反馈的共源极增益}中可以被清晰看到。
\begin{Figure}[带源极负反馈的共源极增益]
    \begin{FigureSub}[区间图;区间图带源极负反馈的的共源级增益]
        \includegraphics[scale=0.82]{build/Chapter03A_05a.fig.pdf}
    \end{FigureSub}\\ \vspace{0.5cm}
    \begin{FigureSub}[电压图;电压图带源极负反馈的的共源级增益]
        \includegraphics[scale=0.82]{build/Chapter03A_05b.fig.pdf}
    \end{FigureSub}
    \begin{FigureSub}[增益图;增益图带源极负反馈的的共源级增益]
        \includegraphics[scale=0.82]{build/Chapter03A_05c.fig.pdf}
    \end{FigureSub}
\end{Figure}
% 当然,采用二极管负载事实上可以更好的解决非线性的问题,但源极负反馈也是一种手段。

源极负反馈下,增益应该如何计算呢?如\xref{fig:带源极负反馈的共源级小信号电路},源极电阻导致了许多麻烦
\begin{itemize}
    \item 源极电阻使源和地不在短接,这使我们不得不考虑体效应。
    \item 源极电阻使电路网络更加复杂了,现在$g_{m}V_{GS}$不再直接位于输出端口了。
\end{itemize}
\begin{Figure}[带源极负反馈的共源级小信号电路]
    \includegraphics[scale=0.8]{build/Chapter03A_24.fig.pdf}
\end{Figure}

这些麻烦都对增益的计算带来了挑战。不过,对策就存在于电路基础中——诺顿定理!

其实,稍稍回顾一下我们就会发现,在本小节之前的所有放大电路,都是\xref{fig:共源放大的诺顿等效电路}的结构。
\begin{Figure}[共源放大的诺顿等效电路]
    \includegraphics[scale=0.8]{build/Chapter03A_27.fig.pdf}
\end{Figure}
很明显,这里有
\begin{Equation}
    V_{out}=-I_{out}R_{out}=-G_mV_{in}(R_O\parallel R_D)
\end{Equation}
即
\begin{Equation}
    A_V=-G_m(R_O\parallel R_D)
\end{Equation}
在\xref{fig:采用电阻负载的共源级小信号电路}中,对应到\xref{fig:共源放大的诺顿等效电路},各个参量就分别是
\begin{Equation}
    G_{m}=g_m\qquad R_{O}=r_{O}\qquad V_{in}=V_{GS}
\end{Equation}
这就会回到我们相当熟悉的$A_V=-g_m(r_{O}\parallel R_D)$的形似了。现在我们的想法是,将\xref{fig:带源极负反馈的共源级小信号电路}也转化为\xref{fig:共源放大的诺顿等效电路}的形式,当然$G_m, R_O$等参量需要被重新计算。这听上去可能很困难,但幸运的是,诺顿定理告诉我们,任何电路网络都可以等效为电流源和一个电阻的并联,具体而言
\begin{itemize}
    \item 计算电流,即$G_m$时,输出端口短接$V_{out}=0$,而$G_m$由$G_m=I_{out}/V_{in}$给出。
    \item 计算电阻,即$G_O=R_O^{-1}$时,输入端口短接$V_{in}=0$,而$G_O$由$G_O=I_{in}/V_{out}$给出。
\end{itemize}
请注意,上述两个步骤中负载$R_D$均不需要被考虑,它不是电路网络的一部分。

% 现在,我们来实践这种想法。

\subsubsection{计算等效转移电导}\setpeq{计算等效转移电导共源}
计算等效转移电导,对于\xref{fig:共源放大的诺顿等效电路},将输出端口短接,如\xref{fig:计算共源放大等效转移电导}所示
\begin{Figure}[计算共源放大等效转移电导]
    \includegraphics[scale=0.8]{build/Chapter03A_25.fig.pdf}
\end{Figure}
列写代换关系
\begin{Equation}&[1]
    V_{GS}=V_{in}-I_{out}R_S\qquad
    V_{BS}=-I_{out}R_S\qquad
    V_{DS}=-I_{out}R_S
\end{Equation}
基于$I_{out}$建立方程
\begin{Equation}&[2]
    I_{out}=g_mV_{GS}+g_{mb}V_{BS}+r_{O}^{-1}V_{DS}
\end{Equation}
在\xrefpeq{2}中代入\xrefpeq{1}
\begin{Equation}&[3]
    I_{out}=g_m(V_{in}-I_{out}R_S)+g_{mb}(-I_{out}R_S)+r_{O}^{-1}(-I_{out}R_S)
\end{Equation}
稍作整理
\begin{Equation}
    I_{out}=g_mV_{in}-(g_m+g_{mb}+r_{O}^{-1})I_{out}R_S
\end{Equation}
将$I_{out}$均移至左侧
\begin{Equation}
    I_{out}\qty[1+(g_m+g_{mb}+r_{O}^{-1})R_S]=g_mV_{in}
\end{Equation}
这就得到
\begin{Equation}
    G_m=I_{out}/V_{in}=\frac{g_m}{1+(g_m+g_{mb}+r_{O}^{-1})R_S}
\end{Equation}
这是第一种形式,上下同乘$r_O$可以得到第二种形式
\begin{Equation}
    G_m=\frac{g_mr_{O}}{R_S+r_O\qty[1+(g_m+g_{mb})R_S]}
\end{Equation}
% 将结论整理如下
\begin{BoxFormula}[共源放大器的等效转移电导]
    共源极放大,当存在源极反馈时,等效转移电导为
    \begin{Equation}
        G_m=\frac{g_m}{1+(g_m+g_{mb}+r_{O}^{-1})R_S}
    \end{Equation}
    也可以表示为
    \begin{Equation}
        G_m=\frac{g_mr_{O}}{R_S+r_O\qty[1+(g_m+g_{mb})R_S]}
    \end{Equation}
\end{BoxFormula}
\subsubsection{计算等效输出电导}\setpeq{计算等效输出电导共源}
计算等效输出电导,对于\xref{fig:共源放大的诺顿等效电路},将输出端口短接,如\xref{fig:计算共源放大等效转移电导}所示
\begin{Figure}[计算共源放大等效输出电导]
    \includegraphics[scale=0.8]{build/Chapter03A_26.fig.pdf}
\end{Figure}
列写代换关系
\begin{Equation}&[1]
    V_{GS}=-I_{out}R_S\qquad
    V_{BS}=-I_{out}R_S\qquad
    V_{DS}=V_{out}-I_{out}R_S
\end{Equation}
我们可以关注一下\xrefpeq{1}和\xrefpeq[计算等效转移电导共源]{1}的联系和差异,它们对应了\xref{fig:计算共源放大等效输出电导}和\xref{fig:计算共源放大等效转移电导}。

基于$I_{out}$建立方程,和之前一样
\begin{Equation}&[2]
    I_{out}=g_mV_{GS}+g_{mb}V_{BS}+r_{O}^{-1}V_{BS}
\end{Equation}
在\xrefpeq{2}中代入\xrefpeq{1},得到
\begin{Equation}
    I_{out}=g_m(-I_{out}R_S)+g_{mb}(-I_{out}R_S)+r_{O}^{-1}(V_{out}-I_{out}R_S)
\end{Equation}
稍作整理
\begin{Equation}
    I_{out}=r_{O}^{-1}V_{out}-(g_m+g_{mb}+r_{O}^{-1})I_{out}R_S
\end{Equation}
将$I_{out}$均移至左侧
\begin{Equation}
    I_{out}\qty[1+(g_m+g_{mb}+r_{O}^{-1})R_S]=r_{O}^{-1}V_{out}
\end{Equation}
这就得到
\begin{Equation}
    G_{O}=I_{out}/V_{out}=\frac{r_O^{-1}}{1+(g_m+g_{mb}+r_{O}^{-1})R_S}
\end{Equation}
这是第一种形式,上下同乘$r_O$可以得到第二种形式
\begin{Equation}
    G_{O}=\frac{1}{R_S+r_O\qty[1+(g_m+g_{mb})R_S]}
\end{Equation}
对于输出电导,这第二种形式是很有益的,因为其倒数给出的输出电阻的形式尤为简单。
\begin{BoxFormula}[共源放大器的等效输出电导]
    共源极放大,当存在源极反馈时,等效输出电导为
    \begin{Equation}
        G_O=\frac{r_O^{-1}}{1+(g_m+g_{mb}+r_{O}^{-1})R_S}
    \end{Equation}
    也可以表示为
    \begin{Equation}
        G_O=\frac{1}{R_S+r_O\qty[1+(g_m+g_{mb})R_S]}
    \end{Equation}
    输出电阻即
    \begin{Equation}
        R_O=R_S+r_O\qty[1+(g_m+g_{mb})R_S]
    \end{Equation}
\end{BoxFormula}

对比\xref{fml:共源放大器的等效转移电导}和\xref{fml:共源放大器的等效输出电导},若改记$r_{O}^{-1}=g_O$
\begin{Equation}
    G_m=\frac{g_m}{1+(g_m+g_{mb}+r_{O}^{-1})R_S}\qquad
    G_O=\frac{g_O}{1+(g_m+g_{mb}+r_{O}^{-1})R_S}
\end{Equation}
我们注意到:\empx{源极负反馈的效果是,使$g_m,g_O$分别减小了$1+(g_m+g_{mb}+r_{O}^{-1})R_S$倍}。

\subsubsection{计算增益}\setpeq{计算增益共源极}
现在,让我们回到增益的计算上
\begin{Equation}&[1]
    A_V=-G_m(R_O\parallel R_D)
\end{Equation}
我们不妨记
\begin{Equation}&[2]
    A_V=-G_mR_{out}
\end{Equation}
这里$R_O$代表MOS管的输出电阻,而$R_{out}=R_O\parallel R_{D}$则是整个放大器的输出电阻。

这里先来计算$R_{out}$,应用\fancyref{fml:共源放大器的等效输出电导}
\begin{Equation}&[3]
    R_{out}=R_{O}\parallel R_D=\qty\big(R_S+r_O\qty[1+(g_m+g_{mb})R_S])\parallel R_D
\end{Equation}
得到
\begin{Equation}&[4]
    R_{out}=\frac{R_S+r_O\qty[1+(g_m+g_{mb})R_S]}{R_D+R_S+r_O\qty[1+(g_m+g_{mb})R_S]}R_D
\end{Equation}
而\fancyref{fml:共源放大器的等效转移电导}又告诉我们
\begin{Equation}&[5]
    G_m=\frac{g_mr_O}{R_S+r_O\qty[1+(g_m+g_{mb})R_S]}
\end{Equation}
当我们应用\xrefpeq{2}计算增益时,\xrefpeq{4}的分子和\xrefpeq{5}的分母可以约掉
\begin{Equation}
    A_V=\frac{-g_mr_OR_D}{R_D+R_S+r_O\qty[1+(g_m+g_{mb})R_S]}
\end{Equation}
这里得到的$A_V, R_{out}$代表了我们对共源放大电路最深刻的定量理解,整理如下。
\begin{BoxFormula}[共源级放大器综述]
    共源级放大器的增益为
    \begin{Equation}
        A_V=\frac{-g_mr_OR_D}{R_D+R_S+r_O\qty[1+(g_m+g_{mb})R_S]}
    \end{Equation}
    共源级放大器的输出电阻为
    \begin{Equation}
        R_{out}=\frac{R_S+r_O\qty[1+(g_m+g_{mb})R_S]}{R_D+R_S+r_O\qty[1+(g_m+g_{mb})R_S]}R_D
    \end{Equation}
    共源级放大器的输入电阻为
    \begin{Equation}
        R_{in}=\infty
    \end{Equation}
\end{BoxFormula}

至此,我们也最终引入了描述一个放大器特性的三个要素:增益、输出电阻、输入电阻。不同组态的放大器在这三个要素上的表现各有优劣,暂且我们先不评价共源放大器的性能。最后解释一下,共源放大器的输入电阻为何是无穷大?因为共源极的输入是绝缘栅极,不存在电流。

最后的最后,若不考虑沟道调制和体效应,源极负反馈对增益的影响是简单的
\begin{Equation}
    A_V=\frac{-g_m R_D}{1+g_mR_S}
\end{Equation}
假若上下同除$g_m$
\begin{Equation}
    A_V=-\frac{R_D}{R_S+g_m^{-1}}
\end{Equation}
这带来一些直观理解:\empx{共源放大的增益,等于漏极节点看到的电阻和源极通路上看到的电阻之比的负值},漏极节点电阻即$R_D$,源极通路电阻即$R_S$与“栅源间跨导”电阻$g_m^{-1}$的串联。