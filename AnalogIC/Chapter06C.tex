\section{共漏级放大器的频率响应}
本节考虑共漏级放大器的频率响应,如\xref{fig:共漏级放大器的频率响应电路}所示。注意到两点:首先为简化分析这里使用电流源而不是电阻作为负载,其次这里电容仅出现了$C_{GS},C_{GD},C_{BS}$,其中,$C_{BD}$由于漏接电源被短接。另外,$C_{GD}$表现为栅的接地电容也是因为漏接了电源,在小信号下就相当于地。

\begin{Figure}[共漏级放大器的频率响应电路]
    \includegraphics[scale=0.8]{build/Chapter06C_02.fig.pdf}
\end{Figure}

\subsection{共漏级放大器增益的的频率特性}
顺着之前的思路,我们会很自然的想,这里能否同样将$C_{GS}$运用米勒定理等效?理论上是可以,但实际情况是,这里的输入输出节点间通过$C_{GS}$有很强的相互作用,使用米勒定理等效将造成极大的偏差。因此,如\xref{fig:共源级放大器的频率响应小信号电路}所示,我们直接绘制\xref{fig:共漏级放大器的频率响应电路}的小信号电路进行分析。

\begin{Figure}[共源级放大器的频率响应小信号电路]
    \includegraphics[scale=0.8]{build/Chapter06C_03.fig.pdf}
\end{Figure}\setpeq{共漏级放大器增益的的频率特性}
就\xref{fig:共源级放大器的频率响应小信号电路}中标注X处,应用KCL
\begin{Equation}&[1]
    (V_{GS}+V_{out}-V_{in})R_S^{-1}+V_{GS}C_{GS}s+(V_{GS}+V_{out})C_{GD}s=0
\end{Equation}
就\xref{fig:共源级放大器的频率响应小信号电路}中标注Y处,应用KCL
\begin{Equation}&[2]
    V_{out}C_{BS}s-V_{GS}C_{GS}s-V_{GS}g_m=0
\end{Equation}
整理\xrefpeq{2}得到
\begin{Equation}&[3]
    V_{out}C_{BS}s-V_{GS}(C_{GS}s+g_m)=0
\end{Equation}
将$V_{GS}$用$V_{out}$表示
\begin{Equation}&[4]
    V_{GS}=\frac{C_{BS}s}{C_{GS}s+g_m}V_{out}
\end{Equation}
整理\xrefpeq{1}得到
\begin{Equation}&[5]
    V_{GS}(R_S^{-1}+C_{GD}s+C_{GS}s)=V_{in}R_S^{-1}-V_{out}C_{GD}s-V_{out}R_S^{-1}
\end{Equation}
将\xrefpeq{4}代入\xrefpeq{5}
\begin{Equation}&[6]
    V_{out}\frac{C_{BS}s(R_S^{-1}+C_{GD}s+C_{GS}s)}{C_{GS}s+g_m}=V_{in}R_S^{-1}-V_{out}C_{GD}s-V_{out}R_S^{-1}
\end{Equation}
将所有$V_{out}$都移至左侧,并通分
\begin{Equation}&[7]
    \qquad\qquad
    V_{out}\frac{C_{BS}s(R_S^{-1}+C_{GD}s+C_{GS}s)+(C_{GD}s+R_S^{-1})(C_{GS}s+g_m)}{C_{GS}s+g_m}=V_{in}R_S^{-1}
    \qquad\qquad
\end{Equation}
即得
\begin{Equation}&[8]
    \qquad
    A(s)=\frac{V_{out}}{V_{in}}=\frac{C_{GS}s+g_m}{R_S[C_{BS}s(R_S^{-1}+C_{GD}s+C_{GS}s)+(C_{GD}s+R_S^{-1})(C_{GS}s+g_m)]}
    \qquad
\end{Equation}
不妨上下同乘$g_m^{-1}$,这样的目的是将分母的常数项凑成$1$
\begin{Equation}&[9]
    \qquad\qquad
    A(s)=\frac{C_{GS}g_m^{-1}s+1}{R_Sg_m^{-1}[C_{BS}s(R_S^{-1}+C_{GD}s+C_{GS}s)+(C_{GD}s+R_S^{-1})(C_{GS}s+g_m)]}
    \qquad\qquad
\end{Equation}
\xrefpeq{9}必然可以转化为以下形式
\begin{Equation}&[10]
    A(s)=\frac{C_{GS}g_m^{-1}s+1}{As^2+Bs+1}
\end{Equation}
其中系数$A$为
\begin{Equation}&[11]
    A=R_Sg_m^{-1}(C_{GD}C_{GS}+C_{GS}C_{BS}+C_{BS}C_{GD})
\end{Equation}
其中系数$B$为
\begin{Equation}&[12]
    B=R_SC_{GD}+g_m^{-1}(C_{GS}+C_{BS})
\end{Equation}
将\xrefpeq{10}转化为可以凸显零点和极点的形式
\begin{Equation}&[13]
    A(s)=\frac{1-s/\omega_z}{(1-s/\omega_{p1})(1-s/\omega_{p2})}
\end{Equation}
其中$\omega_z$为
\begin{Equation}
    \omega_z=-\frac{g_m}{C_{GS}}
\end{Equation}
其中$\omega_{p1}$应用主极点近似,参照\xrefpeq[共源级放大器增益的频率特性]{19}
\begin{Equation}
    \omega_{p1}=-\frac{1}{B}=-\frac{1}{R_SC_{GD}+g_m^{-1}(C_{GS}+C_{BS})}
\end{Equation}
其中$\omega_{p2}$应用主极点近似,参照\xrefpeq[共源级放大器增益的频率特性]{20}
\begin{Equation}
    \omega_{p2}=-\frac{B}{A}=-\frac{R_SC_{GD}+g_m^{-1}(C_{GS}+C_{BS})}{R_Sg_m^{-1}(C_{GD}C_{GS}+C_{GS}C_{BS}+C_{BS}C_{GD})}
\end{Equation}
整理如下
\begin{BoxFormula}[共漏级频率响应--增益--精确结果]
    共漏级放大器,具有两个左极点$\omega_{p1},\omega_{p2}$和一个左零点$\omega_z$,系统函数为
    \begin{Equation}
        A(s)=\frac{1-s/\omega_z}{(1-s/\omega_{p1})(1-s/\omega_{p2})}
    \end{Equation}
    输入极点频率$\omega_{p1}$为(主极点近似)
    \begin{Equation}
        \omega_{p1}=-\frac{1}{R_SC_{GD}+g_m^{-1}(C_{GS}+C_{BS})}
    \end{Equation}
    输出极点频率$\omega_{p2}$为(主极点近似)
    \begin{Equation}
        \omega_{p2}=-\frac{R_SC_{GD}+g_m^{-1}(C_{GS}+C_{BS})}{R_Sg_m^{-1}(C_{GD}C_{GS}+C_{GS}C_{BS}+C_{BS}C_{GD})}
    \end{Equation}
    零点频率$\omega_z$为(精确解)
    \begin{Equation}
        \omega_z=-\frac{g_m}{C_{GS}}
    \end{Equation}
\end{BoxFormula}
若对比\xref{fml:共漏级频率响应--增益--精确结果}和\xref{fml:共源级频率响应--增益--精确结果},我们注意到
\begin{itemize}
    \item 共漏级放大器的零点$\omega_z=-g_m/C_{GS}$是一个左零点(另有两个左极点)。
    \item 共源级放大器的零点$\omega_z=+g_m/C_{GD}$是一个右零点(另有两个左极点)。
\end{itemize}

\xref{fig:共漏级放大器的零极点频率}展示了共漏级放大器的零极点频率,分析如下
\begin{itemize}
    \item 当$R_S$较大时,$\omega_{p1},\omega_{p2},\omega_z$的相对关系是是$|\omega_{p1}|<|\omega_{p2}|<|\omega_{z}|$。
    \item 当$R_S$较小时,$\omega_{p1},\omega_{p2},\omega_z$的相对关系是是$|\omega_{p1}|<|\omega_{z}|<|\omega_{p2}|$。
    \item 当$R_S$适中时,这时出现了一些非常特殊的情况,此时$\omega_{p1}$和$\omega_{p2}$不再是实数,而是一组共轭复根,具有相同的实部和相反的虚部,若从零极点图上看,两者表现为左半复平面上关于实轴对称的两个点。同时注意到,该阶段$\omega_{p1},\omega_{p2}$的实部和虚部大小基本相同。
    \item 主极点近似效果仍然相当不错,但显然并不适用于$\omega_{p1},\omega_{p2}$共轭复根的情况。
\end{itemize}

\begin{Figure}[共漏级放大器的零极点频率]
    \includegraphics[scale=0.8]{build/Chapter06C_01a.fig.pdf}
\end{Figure}

\subsection{共漏级放大器阻抗的的频率特性}
\xref{fig:共源级放大器的输入阻抗}将帮助我们分析共漏级放大器的输入阻抗,考虑体效应,暂时忽略$C_{GD}$,介于它是直接并联在输入上,可以在最后并联到结果上。另外,计算输入阻抗时无需考虑$R_S$的影响。

\begin{Figure}[共漏级放大器的输入阻抗]
    \includegraphics[scale=0.8]{build/Chapter06C_04.fig.pdf}
\end{Figure}\setpeq{共漏级放大器的输入阻抗}

输入节点$V_{in}$可以表示为下式
\begin{Equation}&[1]
    V_{in}=I_{in}\frac{1}{sC_{GS}}+V_{out}
\end{Equation}
即
\begin{Equation}&[2]
    V_{out}=V_{in}-I_{in}\frac{1}{sC_{GS}}
\end{Equation}
输出节点$V_{out}$可以表示为下式
\begin{Equation}&[3]
    V_{out}=(I_{in}+g_mV_{in}-g_mV_{out}-g_{mb}V_{out})\frac{1}{sC_{BS}}
\end{Equation}
将\xrefpeq{2}代入\xrefpeq{3}
\begin{Equation}&[4]
    \qquad\qquad
    V_{in}-I_{in}\frac{1}{sC_{GS}}=\qty[I_{in}+g_mV_{in}-(g_m+g_{mb})\qty(V_{in}-I_{in}\frac{1}{sC_{GS}})]\frac{1}{sC_{BS}}
    \qquad\qquad
\end{Equation}
整理,注意到$g_mV_{in}$被约掉了
\begin{Equation}&[5]
    V_{in}-I_{in}\frac{1}{sC_{GS}}=\qty[I_{in}-g_{mb}V_{in}-I_{in}(g_m+g_{mb})\frac{1}{sC_{GS}}]\frac{1}{sC_{BS}}
\end{Equation}
将$V_{in}$和$I_{in}$的项分别置于等式两侧
\begin{Equation}&[6]
    V_{in}\qty(1+g_{mb}\frac{1}{sC_{BS}})=I_{in}\qty(\frac{1}{sC_{BS}}+\frac{1}{sC_{GS}}+\frac{g_m+g_{mb}}{s^2C_{GS}C_{BS}})
\end{Equation}
通分
\begin{Equation}&[7]
    V_{in}\frac{sC_{BS}+g_{mb}}{sC_{BS}}=I_{in}\frac{sC_{GS}+sC_{BS}+g_m+g_{mb}}{s^2C_{GS}C_{BS}}
\end{Equation}
由此即得$Z_{in}$
\begin{Equation}&[8]
    Z_{in}=\frac{V_{in}}{I_{in}}=\frac{sC_{GS}+sC_{BS}+g_m+g_{mb}}{sC_{GS}(sC_{BS}+g_{mb})}
\end{Equation}
该基础上,我们还要并上$1/sC_{GD}$
\begin{Equation}&[9]
    Z_{in}=\frac{sC_{GS}+sC_{BS}+g_m+g_{mb}}{sC_{GS}(sC_{BS}+g_{mb})}\parallel\frac{1}{sC_{GD}}
\end{Equation}
整理如下
\begin{BoxFormula}[共漏级频率响应--输入阻抗--精确结果]
    共漏级放大器,输入阻抗$Z_{in}$的精确结果是
    \begin{Equation}
        Z_{in}=\frac{sC_{GS}+sC_{BS}+g_m+g_{mb}}{sC_{GS}(sC_{BS}+g_{mb})}\parallel\frac{1}{sC_{GD}}
    \end{Equation}
\end{BoxFormula}
在低频时,有$|C_{BS}s|\ll g_{mb}$(下式暂时略去$1/sC_{GD}$)
\begin{Equation}
    Z_{in}=\frac{sC_{GS}+g_m+g_{mb}}{sC_{GS}g_{mb}}=\frac{1}{sC_{GS}}\qty(1+\frac{g_m}{g_{mb}})+\frac{1}{g_m}
\end{Equation}
在高频时,有$|C_{BS}s|\gg g_{mb}$(下式暂时略去$1/sC_{GD}$)
\begin{Equation}
    Z_{in}=\frac{sC_{GS}+sC_{BS}+g_m}{s^2C_{GS}C_{BS}}=\frac{1}{sC_{GS}}+\frac{1}{sC_{BS}}+\frac{g_m}{s^2C_{GS}C_{BS}}
\end{Equation}
整理如下
\begin{BoxFormula}[共漏级频率响应--输入阻抗--近似结果]
    共漏级放大器,输入阻抗$Z_{in}$在低频下的近似为
    \begin{Equation}
        Z_{in}=\qty[\frac{1}{sC_{GS}}\qty(1+\frac{g_m}{g_{mb}})+\frac{1}{g_m}]\parallel\frac{1}{sC_{GD}}
    \end{Equation}
    共漏级放大器,输入阻抗$Z_{in}$在高频下的近似为
    \begin{Equation}
        Z_{in}=\qty[\frac{1}{sC_{GS}}+\frac{1}{sC_{BS}}+\frac{g_m}{s^2C_{GS}C_{BS}}]\parallel\frac{1}{sC_{GD}}
    \end{Equation}
\end{BoxFormula}

\xref{fig:共漏级放大器的输入阻抗}展示了共漏级放大器的输入阻抗特性,分析如下
\begin{itemize}
    \item \xref{fml:共漏级频率响应--输入阻抗--近似结果}的低频近似和高频近似,对电抗$X_{in}$和阻抗的模$|Z_{in}|$而言很可靠。
    \begin{itemize}
        \item 低频近似用虚线表示,注意到其在低频段汇入实线(精确解)。
        \item 高频近似用点线表示,注意到其在高频段汇入实线(精确解)。
    \end{itemize}
    \item 共漏级放大器的一个特殊之处在于,其输入阻抗中的电阻成分是负的,这种负阻特性可能引起不稳定,需要注意。另外,应明确的是,\xref{fml:共漏级频率响应--输入阻抗--近似结果}给出的低频近似和高频近似不适用于电阻,其中,低频近似给出了正电阻,高频近似给出了高了一个数量级的负电阻。
\end{itemize}
\begin{Figure}[共漏级放大器的输入阻抗]
    \begin{FigureSub}[电阻和电抗;共漏级放大器的输入阻抗电阻和电抗]
        \includegraphics[scale=0.8]{build/Chapter06C_01b.fig.pdf}
    \end{FigureSub}\\ \vspace{0.5cm}
    \begin{FigureSub}[阻抗的幅值;共漏级放大器的输入阻抗幅值]
        \includegraphics[scale=0.8]{build/Chapter06C_01c.fig.pdf}
    \end{FigureSub}
    \begin{FigureSub}[阻抗的相角;共漏级放大器的输入阻抗相角]
        \includegraphics[scale=0.8]{build/Chapter06C_01d.fig.pdf}
    \end{FigureSub}
\end{Figure}

\xref{fig:共漏级放大器的输出阻抗}将帮助我们分析共漏级放大器的输出阻抗,考虑体效应,暂时忽略$C_{BS}$,介于它是直接并联在输出上,可以最后并联到结果上。同时,为了计算简洁,我们不考虑$C_{GD}$的影响。\setpeq{共漏级放大器的输出阻抗}

首先$R_S$和$1/sC_{GS}$的串联阻抗为
\begin{Equation}&[1]
    R_S+\frac{1}{sC_{GS}}=\frac{1+sC_{GS}R_S}{sC_{GS}}
\end{Equation}

\begin{Figure}[共漏级放大器的输出阻抗]
    \includegraphics[scale=0.8]{build/Chapter06C_05.fig.pdf}
\end{Figure}
我们注意到$V_{G}$是$V_{out}$在$R_S$上的分压
\begin{Equation}&[2]
    V_G=\frac{sC_{GS}R_S}{1+sC_{GS}R_S}
\end{Equation}
我们可以将$V_{out}$表示为$R_S$和$C_{GS}$的串联支路上的压降
\begin{Equation}&[3]
    V_{out}=\qty[I_{out}+g_m(V_{G}-V_{out})-g_{mb}V_{out}]\frac{1+sC_{GS}R_S}{sC_{GS}}
\end{Equation}
整理
\begin{Equation}&[4]
    V_{out}=\qty[I_{out}+g_mV_{G}-(g_m+g_{mb})V_{out}]\frac{1+sC_{GS}R_S}{sC_{GS}}
\end{Equation}
将$V_{G}$用\xrefpeq{2}代换
\begin{Equation}&[5]
    V_{out}=\qty[I_{out}+g_m\frac{sC_{GS}R_S}{1+sC_{GS}R_S}-(g_m+g_{mb})V_{out}]\frac{1+sC_{GS}R_S}{sC_{GS}}
\end{Equation}
展开
\begin{Equation}&[6]
    \qquad\qquad\qquad
    V_{out}=I_{out}\frac{1+sC_{GS}R_S}{sC_{GS}}+g_mR_SV_{out}-(g_m+g_{mb})\frac{1+sC_{GS}R_S}{sC_{GS}}V_{out}
    \qquad\qquad\qquad
\end{Equation}
整理
\begin{Equation}&[7]
    V_{out}\qty[1-g_mR_S+(g_m+g_{mb})\frac{1+sC_{GS}R_S}{sC_{GS}}]=I_{out}\frac{1+sC_{GS}R_S}{sC_{GS}}
\end{Equation}
通分
\begin{Equation}
    V_{out}\frac{sC_{GS}+g_{mb}(1+sC_{GS}R_S)+g_m}{sC_{GS}}=I_{out}\frac{1+sC_{GS}R_S}{sC_{GS}}
\end{Equation}
由此即得$Z_{out}$
\begin{Equation}
    Z_{out}=\frac{1+sC_{GS}R_S}{sC_{GS}(1+g_{mb}R_S)+g_m+g_{mb}}
\end{Equation}
该基础上,我们还要并上$1/sC_{BS}$
\begin{Equation}
    Z_{out}=\frac{1+sC_{GS}R_S}{sC_{GS}(1+g_{mb}R_S)+g_m+g_{mb}}\parallel\frac{1}{sC_{BS}}
\end{Equation}
整理如下
\begin{BoxFormula}[共漏级频率响应--输出阻抗--精确结果]
    共漏级放大器,输出阻抗$Z_{out}$的精确结果是
    \begin{Equation}
        Z_{out}=\frac{1+sC_{GS}R_S}{sC_{GS}(1+g_{mb}R_S)+g_m+g_{mb}}\parallel\frac{1}{sC_{BS}}
    \end{Equation}
\end{BoxFormula}
\xref{fig:共漏级放大器的输出阻抗}展示了共漏级放大器的输出阻抗特性,分析如下
\begin{itemize}
    \item 如\xref{fig:共漏级放大器的输出阻抗幅值}所示,低频下,共漏级的输出阻抗为$1/(g_m+g_{mb})$,符合认知。
    \item 当$R_S$较小时,共漏放大器的输出阻抗中,电阻为正,电抗为负。
    \item 当$R_S$较大时,共漏放大器的输出阻抗中,电阻为正,电抗的正负则与频率有关,我们注意到,频率较小时电抗为正,频率较大时电抗为负。这表明,输出阻抗随着频率增加经历了感性到容性的转化,尽管放大器中并不存在电感。这一特性可用于实现有源电感。
    \item 阻抗模值$|Z_{in}|$的高频特性与$R_S$有关,这大致取决于$R_S$和$1/(gm+g_{mb})$之间的相对大小关系,若$R_S$较小则$|Z_{in}|$是单调减小的,若$R_S$较大则$|Z_{in}|$会先增大后减小。
\end{itemize}

\begin{Figure}[共漏级放大器的输出阻抗]
    \begin{FigureSub}[电阻和电抗;共漏级放大器的输出阻抗电阻和电抗]
        \includegraphics[scale=0.8]{build/Chapter06C_01e.fig.pdf}
    \end{FigureSub}\\ \vspace{0.5cm}
    \begin{FigureSub}[阻抗的幅值;共漏级放大器的输出阻抗幅值]
        \includegraphics[scale=0.8]{build/Chapter06C_01f.fig.pdf}
    \end{FigureSub}
    \begin{FigureSub}[阻抗的相角;共漏级放大器的输出阻抗相角]
        \includegraphics[scale=0.8]{build/Chapter06C_01g.fig.pdf}
    \end{FigureSub}
\end{Figure}