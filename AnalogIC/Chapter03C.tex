\section{共栅级放大器}

\subsection{采用电阻负载的共栅级}
共栅极是一种比较怪异的接法,输入从源极接入,输出从漏极接出,负载位于漏极,栅极则接偏置电压$V_{b}$,这里$V_{b}$未必等于$V_{DD}$,但为了让MOS管导通,可以预期$V_{b}$比较接近$V_{DD}$。

\begin{Figure}[采用电阻负载的共栅级电路]
    \includegraphics[scale=0.8]{build/Chapter03C_02.fig.pdf}
\end{Figure}

这种组态下代换关系是
\begin{Equation}
    V_{GS}=V_{b}-V_{in}\qquad 
    V_{BS}=-V_{in}\qquad
    V_{DS}=V_{out}-V_{in}
\end{Equation}

照例我们从大信号分析开始,同样以电流建立方程
\begin{Equation}
    \mu_nC_{ox}(W/L)[(V_{b}-V_{in}-V_{TH})^2/2]=(V_{DD}-V_{out})/R_D
\end{Equation}
容易解出
\begin{Equation}
    V_{out}=V_{DD}-\mu_nC_{ox}(W/L)[(V_b-V_{in}-V_{TH})^2/2]R_D
\end{Equation}
求导计算增益
\begin{Equation}
    A_V\pdv{V_{out}}{V_{in}}=-\mu_nC_{ox}(W/L)(V_b-V_{in}-V_{TH})\qty(-1-\pdv{V_{TH}}{V_{in}})R_D
\end{Equation}
注意到
\begin{Equation}
    \pdv{V_{TH}}{V_{in}}=-\pdv{V_{TH}}{V_{BS}}=\eta
\end{Equation}
因此有
\begin{Equation}
    A_V=-\mu_nC_{ox}(W/L)(V_b-V_{in}-V_{TH})\qty(-1-\eta)R_D
\end{Equation}
或者
\begin{Equation}
    A_V=\mu_nC_{ox}(W/L)(V_b-V_{in}-V_{TH})\qty(1+\eta)R_D
\end{Equation}
再用跨导代换
\begin{Equation}
    A_V=g_mR_D(1+\eta)
\end{Equation}
共源的增益为$A_V=-g_mR_D$,共栅的增益为$A_V=g_mR_D(1+\eta)$,这表明:共栅级可以提供和共源级相当的高增益,但为非反相的放大,有意思的是。体效应在共栅中还提高了增益。

\begin{Figure}[采用电阻负载的共栅极增益]
    \begin{FigureSub}[区间图;区间图采用电阻负载的共栅极增益]
        \includegraphics[scale=0.82]{build/Chapter03C_01a.fig.pdf}
    \end{FigureSub}\\ \vspace{0.5cm}
    \begin{FigureSub}[电压图;电压图采用电阻负载的共栅极增益]
        \includegraphics[scale=0.82]{build/Chapter03C_01b.fig.pdf}
    \end{FigureSub}
    \begin{FigureSub}[增益图;增益图采用电阻负载的共栅极增益]
        \includegraphics[scale=0.82]{build/Chapter03C_01c.fig.pdf}
    \end{FigureSub}
\end{Figure}

\xref{fig:采用电阻负载的共栅极增益}绘制了共栅放大的完整大信号特性,其结果可以与共源放大的\xref{fig:采用电阻负载的共源级增益}对照。

\begin{BoxFormula}[采用电阻负载的共栅极增益]
    共栅级放大,采用电阻负载,增益为
    \begin{Equation}
        A_V=g_mR_D(1+\eta)
    \end{Equation}
\end{BoxFormula}

\subsection{带有源极负反馈的共栅级}
共栅和共源类似,都可以引入源极电阻,本小节将考虑这种情况(同时考虑沟道调制)。

带有源极负反馈的共栅级的电路如\xref{fig:带有源极负反馈的共栅级电路}所示
\begin{Figure}[带有源极负反馈的共栅级电路]
    \includegraphics[scale=0.8]{build/Chapter03C_03.fig.pdf}
\end{Figure}
它的小信号模型的结构是目前看起来最奇特的,但只不过是输入的位置比较特别
\begin{Figure}[带有源极负反馈的共栅级小信号电路]
    \includegraphics[scale=0.8]{build/Chapter03C_07.fig.pdf}
\end{Figure}
我们照例还是采用$G_m,G_O,A_V$的那种分析方法。

\subsubsection{计算共栅放大等效转移电导}
计算等效转移电导,对于\xref{fig:带有源极负反馈的共栅级小信号电路},将输出端口短接,如\xref{fig:计算共栅放大等效转移电导}
\begin{Figure}[计算共栅放大等效转移电导]
    \includegraphics[scale=0.8]{build/Chapter03C_04.fig.pdf}
\end{Figure}

列写代换关系
\begin{Equation}
    \qquad\qquad
    V_{GS}=-V_{in}-I_{out}R_S\qquad V_{BS}=-V_{in}-I_{out}R_S\qquad
    V_{DS}=-V_{in}-I_{out}R_S
    \qquad\qquad
\end{Equation}
基于电流建立方程
\begin{Equation}
    I_{out}=g_mV_{GS}+g_{mb}V_{BS}+r_O^{-1}V_{DS}
\end{Equation}
即
\begin{Equation}
    I_{out}=(g_{m}+g_{mb}+r_{O}^{-1})(-V_{in}-I_{out}R_S)
\end{Equation}
整理得到
\begin{Equation}
    I_{out}\qty[1+(g_m+g_{mb}+r_{O}^{-1})R_S]=-(g_m+g_{mb}+r_O^{-1})V_{in}
\end{Equation}
因此
\begin{Equation}
    G_m=\frac{I_{out}}{V_{in}}=-\frac{g_m+g_{mb}+r_{O}^{-1}}{1+(g_m+g_{mb}+r_{O}^{-1})R_S}
\end{Equation}
上下同乘$r_O$
\begin{Equation}
    G_m=-\frac{1+r_O(g_m+g_{mb})}{R_S+r_O\qty[1+(g_m+g_{mb})R_S]}
\end{Equation}
% 将结论整理如下
\begin{BoxFormula}[共栅放大器的等效转移电导]*
    共栅极放大,当存在源极反馈时,等效转移电导为
    \begin{Equation}
        G_m=-\frac{g_m+g_{mb}+r_{O}^{-1}}{1+(g_m+g_{mb}+r_{O}^{-1})R_S}
    \end{Equation}
    也可以表示为
    \begin{Equation}
        G_m=-\frac{1+r_O(g_m+g_{mb})}{R_S+r_O\qty[1+(g_m+g_{mb})R_S]}
    \end{Equation}
\end{BoxFormula}

\subsubsection{计算共栅放大等效输出电导}
计算等效输出电导,对于\xref{fig:带有源极负反馈的共栅级小信号电路},将输入端口短接,如\xref{fig:计算共栅放大等效输出电导}所示
\begin{Figure}[计算共栅放大等效输出电导]
    \includegraphics[scale=0.8]{build/Chapter03A_26.fig.pdf}
\end{Figure}
有趣的是,我们注意到\xref{fig:计算共栅放大等效输出电导}和\xref{fig:计算共源放大等效输出电导}完全一致,这表明:\empx{共栅放大和共源放大具有相同的输出电导},因此没必要重复计算,而是可以直接调用\fancyref{fml:共源放大器的等效输出电导}。

\begin{BoxFormula}[共栅放大器的等效输出电导]
    共栅极放大,当存在源极反馈时,等效输出电导为
    \begin{Equation}
        G_O=\frac{r_O^{-1}}{1+(g_m+g_{mb}+r_{O}^{-1})R_S}
    \end{Equation}
    也可以表示为
    \begin{Equation}
        G_O=\frac{1}{R_S+r_O\qty[1+(g_m+g_{mb})R_S]}
    \end{Equation}
\end{BoxFormula}

\subsubsection{计算共栅放大的增益和输入输出电阻}
这里和过去一样,基于$R_{out}=R_O\parallel R_D$和$A_V=-G_mR_{out}$计算$A_V$和$R_{out}$,但是有一点不同,共源和共漏的输入电阻$R_{in}$没什么可计算的,都是无穷大,因为它们的输入信号从栅极接入,而栅极绝缘。然而,共栅的输入是从源极接入了,这时输入电阻$R_{in}$就成为了一个需要认真计算的量了。另一个问题是,输入电阻$R_{in}$的计算和输出电阻$R_{out}$方法上有些差异
\begin{itemize}
    \item 计算输出电阻时,\textbf{短路}输入接口,此时$R_{out}$由$R_{out}=V_{out}/I_{out}$给出。
    \item 计算输入电阻时,\textbf{开路}输出接口,此时$R_{in}$由$R_{in}=V_{in}/I_{in}$给出。
\end{itemize}
输出电阻计算时,有一个讨巧的办法,介于负载$R_D$总是并联在输出端口上,我们可以先计算不带$R_D$的输出电阻$R_O$,随后由$R_{out}=R_D\parallel R_O$给出总的输出电阻。输入电阻$R_{in}$并没有这种技巧可用,无法先得到一个不带负载$R_D$的$R_I$后再得到$R_{in}$,输入电阻始终要考虑负载!

\begin{Figure}[计算共栅放大的的输入输出电阻]
    \begin{FigureSub}[共栅放大的的输入电阻]
        \includegraphics[scale=0.8]{build/Chapter03C_05.fig.pdf}
    \end{FigureSub}\\ \vspace{0.5cm}
    \begin{FigureSub}[共栅放大的的输出电阻]
        \includegraphics[scale=0.8]{build/Chapter03C_06.fig.pdf}
    \end{FigureSub}
\end{Figure}

现在我们来依据\xref{fig:共栅放大的的输入电阻}计算$R_{in}$,代换关系是
\begin{Equation}
    \qquad
    V_{GS}=-V_{in}+I_{in}R_S\qquad
    V_{BS}=-V_{in}+I_{in}R_S\qquad
    V_{DS}=-V_{in}+I_{in}(R_S+R_D)
    \qquad
\end{Equation}
基于$I_{in}$列写方程
\begin{Equation}
    I_{in}=g_mV_{GS}+g_{mb}V_{BS}+r_O^{-1}V_{DS}
\end{Equation}
进行代换
\begin{Equation}
    I_{in}+(g_m+g_{mb}+r_O^{-1})(-V_{in}+I_{in}R_S)+r_{O}^{-1}I_{in}R_D=0
\end{Equation}
将$I_{in}$和$V_{in}$分离
\begin{Equation}
    I_{in}[(g_m+g_{mb}+r_{O}^{-1})R_S+r_O^{-1}R_D+1]=V_{in}(g_{m}+g_{mb}+r_{O}^{-1})
\end{Equation}
我们计算$R_{in}^{-1}=I_{in}/V_{in}$(这是因为$R_{in}^{-1}$的形式与$R_{out}$比较匹配)
\begin{Equation}
    R_{in}^{-1}=\frac{I_{in}}{V_{in}}=\frac{(g_m+g_{mb}+r_{O}^{-1})}{(g_m+g_{mb}+r_O^{-1})R_S+r_{O}^{-1}R_D+1}
\end{Equation}
上下同乘$r_O$可以得到
\begin{Equation}
    R_{in}^{-1}=\frac{(g_m+g_{mb})r_O+1}{R_S+R_D+r_O[1+(g_m+g_{mb})R_S]}
\end{Equation}
至此,就得到了$R_{in}^{-1}$的结果,而$R_{out}=R_{O}\parallel R_{D}$是很容易求的
\begin{Equation}
    R_{out}=\frac{R_S+r_O\qty[1+(g_m+g_{mb})R_S]}{R_D+R_S+r_O\qty[1+(g_m+g_{mb})R_S]}R_D
\end{Equation}
增益$A_V=-G_mR_{out}$也很好计算
\begin{Equation}
    A_V=\frac{(g_m+g_{mb})r_O+1}{R_D+R_S+r_O\qty[1+(g_m+g_{mb})R_S]}R_D
\end{Equation}
将全部结论总结如下
\begin{BoxFormula}[共栅级放大器综述]
    共栅级放大器的增益为
    \begin{Equation}
        A_V=\frac{(g_m+g_{mb})r_O+1}{R_D+R_S+r_O\qty[1+(g_m+g_{mb})R_S]}R_D
    \end{Equation}
    共栅级放大器的输出电阻为
    \begin{Equation}
        R_{out}=\frac{R_S+r_O\qty[1+(g_m+g_{mb})R_S]}{R_D+R_S+r_O\qty[1+(g_m+g_{mb})R_S]}R_D
    \end{Equation}
    共栅级放大器的输入电阻为
    \begin{Equation}
        R_{in}^{-1}=\frac{(g_m+g_{mb})r_O+1}{R_S+R_D+r_O[1+(g_m+g_{mb})R_S]}
    \end{Equation}
\end{BoxFormula}

在本节最后,我们简要总结一下共源、共漏、共栅放大器的输入和输出的位置。
\begin{Tablex}[放大器的输入输出位置]{lXXX}
<类型&共源级放大器源&共漏级放大器&共栅级放大器\\>
源&\hphantom{\textbf{输出}}~~接GND&\textbf{输出}~~接负载接GND&\textbf{输入}\\
漏&\textbf{输出}~~接负载接VDD&\hphantom{\textbf{输出}}~~接VDD&\textbf{输出}~~接负载接VDD\\
栅&\textbf{输入}&\textbf{输入}&\hphantom{\textbf{输出}}~~接偏置Vb\\
\end{Tablex}

% \begin{TableLong}[三类放大器的公式总结]{|c|ccc|}
% <参数&共源级CS&共栅级CG&共漏级CD\\>
%     %-- G_m --%
%     \mr{3}[-5ex]{$G_m$}
%     &\xgp[2ex]{$\mal{\frac{g_mr_O}{R_S+r_O[1+(g_m+g_{mb})R_S]}}$}
%     &\xgp[2ex]{$\mal{-\frac{(g_m+g_{mb})r_O+1}{R_S+r_O[1+(g_m+g_{mb})R_S]}}$}
%     &--\\ \clinelig{2-4}
%     &\xgp[2ex]{$\mal{\frac{g_m}{1+(g_m+g_{mb}+r_O^{-1})R_S}}$}
%     &\xgp[2ex]{$\mal{-\frac{g_m+g_{mb}+r_O^{-1}}{1+(g_m+g_{mb}+r_O^{-1})R_S}}$}
%     &--\\ \clinelig{2-4}
%     &\xgp[2ex]{$\mal{g_m}$}
%     &\xgp[2ex]{$\mal{-g_m-g_{mb}-r_{O}^{-1}}$}
%     &\xgp[2ex]{$\mal{-g_m}$}\\
%     \hlinemid
%     %-- G_O --%
%     \mr{3}[-5ex]{$G_O$}
%     &\xgp[2ex]{$\mal{\frac{1}{R_S+r_O[1+(g_m+g_{mb})R_S]}}$}
%     &\xgp[2ex]{$\mal{\frac{1}{R_S+r_O[1+(g_m+g_{mb})R_S]}}$}
%     &--\\ \clinelig{2-4}
%     &\xgp[2ex]{$\mal{\frac{r_O^{-1}}{1+(g_m+g_{mb}+r_O^{-1})R_S}}$}
%     &\xgp[2ex]{$\mal{\frac{r_O^{-1}}{1+(g_m+g_{mb}+r_O^{-1})R_S}}$}
%     &--\\ \clinelig{2-4}
%     &\xgp[2ex]{$\mal{r_O^{-1}}$}
%     &\xgp[2ex]{$\mal{r_O^{-1}}$}
%     &\xgp[2ex]{$\mal{r_O^{-1}+g_m+g_{mb}}$}\\
%     %-- A_V --%
%     \mr{3}[-5ex]{$A_V$}
%     &\xgp[2ex]{$-\mal{\frac{g_mr_O}{R_D+R_S+r_O[1+(g_m+g_{mb})R_S]}}R_D$}
%     &\xgp[2ex]{$\mal{\frac{(g_m+g_{mb})r_O+1}{R_D+R_S+r_O[1+(g_m+g_{mb})R_S]}R_D}$}
%     &--\\ \clinelig{2-4}
%     &\xgp[2ex]{$\mal{\frac{r_O^{-1}}{1+(g_m+g_{mb}+r_O^{-1})R_S}}$}
%     &\xgp[2ex]{$\mal{\frac{r_O^{-1}}{1+(g_m+g_{mb}+r_O^{-1})R_S}}$}
%     &--\\ \clinelig{2-4}
%     &\xgp[2ex]{$\mal{r_O^{-1}}$}
%     &\xgp[2ex]{$\mal{r_O^{-1}}$}
%     &\xgp[2ex]{$\mal{r_O^{-1}+g_m+g_{mb}}$}\\
% \end{TableLong}