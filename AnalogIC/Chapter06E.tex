\section{共源共栅级放大器的频率响应}
本节考虑共源共栅级放大器的频率响应,如\xref{fig:共源共栅级放大器的频率响应电路}所示。
\begin{Figure}[共源共栅级放大器的频率响应电路]
    \includegraphics[scale=0.8]{build/Chapter06E_02.fig.pdf}
\end{Figure}

\subsection{共源共栅级放大器增益的的频率特性}
共源共栅级的频率响应分析思路和共栅级类似,但有三个节点。

输入节点的电阻是$R_S$,电容是$C_{GS1}$和$C_{GD1}$在输入侧的米勒等效电容,这里共源级的增益是$-g_{m1}(g_{m2}+g_{mb2})^{-1}$,其中$(g_{m2}+g_{mb2})^{-1}$是共栅级的输入电阻,其充当了共源级负载
\begin{Equation}
    \omega_{p1}=-\frac{1}{R_S(C_{GS}+(1+g_{m1}(g_{m2}+g_{mb2})^{-1})C_{GD1})}
\end{Equation}
输出节点的电阻是$R_D$,电容是$C_{BD2}$和$C_{GD2}$
\begin{Equation}
    \omega_{p2}=-\frac{1}{R_D\qty(C_{BD2}+C_{GD2})}
\end{Equation}
中间节点的电阻是$(g_{m2}+g_{mb2})^{-1}$即共栅级的输入电阻,电容是$C_{BD1},C_{BD2},C_{BS2},C_{GS2}$以及$C_{GD1}$在输出侧的米勒等效电容(仍等于$C_{GD1}$),因此有
\begin{Equation}
    \omega_{p3}=-\frac{g_{m2}+g_{mb2}}{C_{GD1}+C_{GS2}+C_{BD1}+C_{BD2}+C_{BS2}}
\end{Equation}
\begin{Figure}[共源共栅级放大器的零极点频率]
    \includegraphics[scale=0.8]{build/Chapter06E_01a.fig.pdf}
\end{Figure}
整理如下
\begin{BoxFormula}[共源共栅级频率响应--增益--米勒近似]*
    共源共栅级放大器,依据米勒近似,具有三个左极点$\omega_{p1},\omega_{p2},\omega_{p3}$,系统函数为
    \begin{Equation}
        A(s)=\frac{-g_mR_D}{(1-s/\omega_{p1})(1-s/\omega_{p2})(1-s/\omega_{p3})}
    \end{Equation}
    输入极点频率$\omega_{p1}$为(米勒近似)
    \begin{Equation}
        \omega_{p1}=-\frac{1}{R_S(C_{GS}+(1+g_{m1}(g_{m2}+g_{mb2})^{-1})C_{GD1})}
    \end{Equation}
    输出极点频率$\omega_{p2}$为(精确结果)
    \begin{Equation}
        \omega_{p2}=-\frac{1}{R_D\qty(C_{BD2}+C_{GD2})}
    \end{Equation}
    中间极点频率$\omega_{p3}$为(米勒近似)
    \begin{Equation}
        \omega_{p3}=-\frac{1}{C_{GD1}+C_{GS2}+C_{BD1}+C_{BD2}+C_{BS2}}
    \end{Equation}
\end{BoxFormula}
\xref{fig:共源共栅级放大器的零极点频率}展示了共源共栅级放大器的零极点频率,通常而言中间极点$\omega_{p3}$的频率最大。


