\section{共栅级放大器的频率响应}
本节考虑共栅级放大器的频率响应,如\xref{fig:共栅级放大器的频率响应电路},由于栅接偏置$V_b$在小信号下就相当于地,因此栅电容$C_{GD},C_{GS}$和$C_{BD},C_{BS}$一样均表现为接地电容。请注意!输入$V_{in}$现在位于源极。

\begin{Figure}[共栅级放大器的频率响应电路]
    \includegraphics[scale=0.8]{build/Chapter06D_02.fig.pdf}
\end{Figure}

\subsection{共栅级放大器增益的的频率特性}
共栅级放大器的频率响应是尤为简单的,因为其输入和输出节点是孤立的。

输入节点的电阻是$R_S$和共栅级输入电阻$1/(g_m+g_{mb})$的并联
\begin{Equation}
    \omega_{p1}=-\frac{1}{\qty[R_S\parallel(g_m+g_{mb})^{-1}](C_{GS}+C_{BS})}
\end{Equation}
输出节点的电阻是共栅级输出电阻$R_D$
\begin{Equation}
    \omega_{p2}=-\frac{1}{R_D\qty(C_{GD}+C_{BD})}
\end{Equation}
整理如下
\begin{BoxFormula}[共栅级频率响应--增益--精确结果]*
    共栅级放大器,具有两个左极点$\omega_{p1},\omega_{p2}$,无零点,系统函数为
    \begin{Equation}
        A(s)=\frac{g_mR_D}{(1-s/\omega_{p1})(1-s/\omega_{p2})}
    \end{Equation}
    输入极点频率$\omega_{p1}$为(精确结果)
    \begin{Equation}
        \omega_{p1}=-\frac{1}{\qty[R_S\parallel(g_m+g_{mb})^{-1}](C_{GS}+C_{BS})}
    \end{Equation}
    输出极点频率$\omega_{p2}$为(精确结果)
    \begin{Equation}
        \omega_{p2}=-\frac{1}{R_D\qty(C_{GD}+C_{BD})}
    \end{Equation}
\end{BoxFormula}

\xref{fig:共漏级放大器的零极点频率}展示了共栅级放大器的零极点频率,分析如下
\begin{itemize}
    \item 共栅级中,输出极点$\omega_{p2}$比输入极点$\omega_{p1}$频率更低,这与共源和共漏相反。
    \item 共栅级中极点$\omega_{p1},\omega_{p2}$均随$R_S$增加趋于稳定,而不会持续减小。
\end{itemize}
\begin{Figure}[共栅级放大器的零极点频率]
    \includegraphics[scale=0.8]{build/Chapter06D_01a.fig.pdf}
\end{Figure}
