\section{基本电流镜}
电流镜的结构和功能我们在模拟电路中其实已经很熟悉了,它可以复制出一个电流源。嗯,搞出一个电流源,这听上去很酷,不是吗!毕竟,在实际生活中,我们很容易想象一个电阻或电压源,它们可以是一节金属或一块干电池,但电流源则似乎只是一些仅存在于电路图纸中的理想符号。但是,电流镜之所以如此重要并不是因为其能产生一个电流源。要知道,在晶体管电路中,遍地都是电流源!任何一个MOS管加适当偏置$V_b$进入饱和区后,都可以视为一个相当不错的电流源,如\xref{fig:朴素电流源}所示,这里偏置电压$V_b$由电阻$R_1,R_2$对电源$V_{DD}$分压得到。

\vspace{0.25cm}
\begin{Figure}[朴素电流源]
    \includegraphics[scale=0.8]{build/Chapter05A_03.fig.pdf}
\end{Figure}
\vspace{0.25cm}

那\xref{fig:朴素电流源}有什么问题吗?我们或许会说$R_1,R_2$上会通过一定电流,产生功耗。确实如此,但这其实不是重点。若要理解这个问题,我们就要充分理解模拟集成电路的设计背景——PVT。\goodbreak

我们总说,PVT是悬在模拟集成电路上的一把键,PVT怎样怎样重要,但什么是PVT?\nopagebreak
\begin{itemize}
    \item P代表Process,代表电路加工过程的工艺波动,晶体管的尺寸和阈值可能会发生偏差。
    \item V代表Voltage,代表电路的供电$V_{DD}$可能会发生波动(想想快用完的干电池)。
    \item T代表Tempreature,代表电路所处的环境温度可能也会发生波动。
\end{itemize}
在\xref{fig:朴素电流源}中,我们产生的$I_{out}$的表达式为
\begin{Equation}
    I_{out}=\frac{1}{2}\mu_nC_{ox}(W/L)\qty(\frac{R_2}{R_1+R_2}V_{DD}-V_{TH})^2
\end{Equation}
但该式表明$I_{out}$是PVT的强函数,P意味着$V_{TH}$的波动,V意味着$V_{DD}$的波动,T则会影响$\mu_n$和$V_{TH}$的值。总而言之,这样的电流源,面对PVT的波动会产生相当大的误差。

由此可见,模拟集成电路的设计并不是提出一个理论上可行的电路就可以了,我们的电路在设计上必须要考虑PVT波动的耐受,否则是脱离实际毫无意义的。我们可以提出,该电路需要工作在某个温度范围和电压范围内,但不能要求必须多少伏或多少度,这是不切实际的。

电流镜的设计是基于对基准电流的“复制”,当然,这样做的前提是,已经存在一个精确的基准电流源可供利用。然而,这可能会导致一个无休止的循环,第一个电流源从哪里来?这就要谈到一个有趣的话题,当今,即便在模拟集成电路中,MOS管也已经基本取代了原先BJT的所有应用,比如我们不会再用BJT构建放大器。那么,BJT是否不再被需要了?答案是否定的,事实上,基准电流源需要通过一个相对复杂,被称为“带隙基准”的电路产生,这其中仍需要用到BJT。当然,本章并不关心基准电流源是怎么来的,我们只关心如何精确的复制它。

\begin{Figure}[基本电流镜]
    \includegraphics[scale=0.8]{build/Chapter05A_04.fig.pdf}
\end{Figure}

电流镜的结构如\xref{fig:基本电流镜},这是一个最基本的结构,$M_1$管和$M_2$管分别位于左右。$M_1$管采用二极管接法并串联了基准电流源$I_{REF}$。$M_2$管在输出端处复制了电流$I_{out}=I_{REF}$,且该电流基本不随输出端的电压$V_{out}$的变化而变化。那么,现在的问题是,这种复制是如何完成的?

关键在于,对于饱和区的MOS管,若我们记$I_{D}=f(V_{GS})$,则有$V_{GS}=f^{-1}(I_D)$。

在\xref{fig:基本电流镜}中,$M_1$管完成了电流电压的转化,使得$V_{GS1}=f^{-1}(I_{REF})$,由于两管栅极短接,栅压相等,使得$V_{GS2}=f^{-1}(I_{REF})$,此时只要输出电压$V_{DS2}=V_{out}$不太小能使$M_2$管处于饱和区,$M_2$管就可以将电压还原为电流,使得$I_{out}=f[f^{-1}(I_{REF})]=I_{REF}$,从而完成镜像。

我们也可以用表达式来证明这一点,较一般的说,$M_1$管和$M_2$管的宽长比可以不同。\setpeq{电流镜}

$M_1$管遵从的方程是
\begin{Equation}&[1]
    I_{D1}=I_{REF}=\frac{1}{2}\mu_nC_{ox}(W/L)_1\qty(V_{GS1}-V_{TH})^2
\end{Equation}
$M_2$管遵从的方程是
\begin{Equation}&[2]
    I_{D2}=I_{out}=\frac{1}{2}\mu_nC_{ox}(W/L)_2\qty(V_{GS2}-V_{TH})^2
\end{Equation}
而考虑到$V_{GS1}=V_{GS2}$,将\xrefpeq{2}与\xrefpeq{1}作比,得到
\begin{Equation}
    I_{out}=\frac{(W/L)_2}{(W/L)_1}I_{REF}
\end{Equation}
换言之,输出端的电流$I_{out}$按照$M_2,M_1$间的器件尺寸比复制$I_{REF}$,特别的,如果$M_2,M_1$尺寸相同,则有$I_{out}=I_{REF}$,这种设计就可以很好的抵御PVT的波动,例如工艺波动使得尺寸发生变化,但是两个邻近的晶体管的尺寸波动通常是一致的,尺寸比不会发生变化,从而确保了电流复制的精确性。这是很重要的,若复制本身受PVT影响,这一切就都没有意义了。

\begin{BoxFormula}[电流镜的复制关系]
    电流镜的输出电流$I_{out}$和基准电流$I_{REF}$的关系是
    \begin{Equation}
        I_{out}=\frac{(W/L)_2}{(W/L)_1}I_{REF}
    \end{Equation}
\end{BoxFormula}

关于基本电流镜,常见的一个问题是为什么$M_1$需要二极管接法,考虑\xref{fig:无效电流镜}的电路
\begin{Figure}[无效电流镜]
    \includegraphics[scale=0.8]{build/Chapter05A_06.fig.pdf}
\end{Figure}
在这个例子中,$M_1$的栅极未与漏极短接,而是接了$V_b$的偏置。然而这个电流镜其实是无法正常工作,尽管两管都工作在饱和区。这里$M_2$上的电流$I_{out}$仅受$V_b$控制,而$M_1$处出现了两个电流源串联的情况,即$I_{REF}$和由$V_b$控制的$I_{D1}$,这是不合法的。实际会发生的情况是
\begin{itemize}
    \item 若$V_b=V_{GS1}$使$I_{D1}>I_{REF}$,那么$V_{DS1}$会减低将其拖入线性区(使电流比预期小)。
    \item 若$V_b=V_{GS1}$使$I_{D1}<I_{REF}$,那么$V_{DS1}$会增加以增强沟道调制(使电流比预期大)。
\end{itemize}
简而言之,我们不能过约束一个饱和区MOS管,即同时控制栅源电压和漏电流。

\begin{Figure}[基本电流镜的特性]
    \begin{FigureSub}[电流特性;基本电流镜电流特性]
        \includegraphics[scale=0.58]{build/Chapter05A_01_0.fig.pdf}
    \end{FigureSub}
    \begin{FigureSub}[节点电压;基本电流镜节点电压]
        \includegraphics[scale=0.58]{build/Chapter05A_01_1.fig.pdf}
    \end{FigureSub}\\ \vspace{0.25cm}
    \begin{FigureSub}[$M_2$管端口;基本电流镜M2管端口]
        \includegraphics[scale=0.58]{build/Chapter05A_01_2.fig.pdf}
    \end{FigureSub}
\end{Figure}
在\xref{fig:基本电流镜的特性}中我们给出了基本电流镜的仿真结果,我们注意到
\begin{itemize}
    \item 若忽略沟道调制效应,即$\lambda=0$,如\xref{fig:基本电流镜电流特性}的灰虚线所示,随着$V_{out}$增大电流$I_{out}$最终趋于$I_{REF}$,这与预期相符,因为$V_{out}$必须增大至能使$M_1$管由线性区进入饱和区。
    \item 若考虑沟道调制效应,即$\lambda=0$,如\xref{fig:基本电流镜电流特性}的黑线所示,注意到即便输出电压$V_{out}$增大至使$M_1$管进入饱和区,此时$I_{out}$仍然会随$V_{out}$的增大而变化,这是偏离理想电流源特性的。问题的根源在于,先前讨论忽略了沟道调制,认为饱和区有$I_D=f(V_{GS})$,但沟道调制使得即便在饱和区仍有$I_D=f(V_{GS},V_{DS})$,这里$V_{DS1}$固定,但$V_{DS2}=V_{out}$则会随输出端口所加的电压而变化。故$V_{DS}$的偏差就会导致经过$f$和$f^{-1}$电流发生偏离。
\end{itemize}

通过表达式,我们也可以看出沟道调制效应是如何破坏电流镜的特性的。\setpeq{电流镜问题}\goodbreak

$M_1$管遵从的方程是
\begin{Equation}&[1]
    I_{D1}=I_{REF}=\frac{1}{2}\mu_nC_{ox}(W/L)_1\qty(V_{GS1}-V_{TH})^2(1+\lambda V_{DS1})
\end{Equation}
$M_2$管遵从的方程是
\begin{Equation}&[2]
    I_{D2}=I_{out}=\frac{1}{2}\mu_nC_{ox}(W/L)_2\qty(V_{GS2}-V_{TH})^2(1+\lambda V_{DS2})
\end{Equation}
这里仍有$V_{GS1}=V_{GS2}$,将\xrefpeq{2}与\xrefpeq{1}作比,得到
\begin{Equation}
    I_{out}=\frac{(W/L)_2}{(W/L)_1}\frac{1+\lambda V_{DS2}}{1+\lambda V_{DS1}}I_{REF}
\end{Equation}
容易看出$V_{DS1}\neq V_{DS2}$将导致电流复制的偏差,那如何解决该问题呢?且看下一节的讨论。