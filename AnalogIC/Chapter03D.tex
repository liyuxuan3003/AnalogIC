\section{共源共栅放大器}

\subsection{套筒式共源共栅}
共源共栅是一个很有趣的结构,它常被称为Cascode,是Cascaded Triodes即级联三极管的缩写,该名称或许可以追溯到真空管时代。尽管这里提到了“级联”,但我们要指出,共源共栅并不能简单视为共源和共栅的级联,它结构有一些特殊,如\xref{fig:套筒式共源共栅级的电路}所示,在共源共栅中

\begin{Figure}[套筒式共源共栅级的电路]
    \includegraphics[scale=0.8]{build/Chapter03D_03.fig.pdf}
\end{Figure}

\begin{itemize}
    \item $M_1$管是“共源”部分,电压输入,电流输出,即$V_{in}\to I_D$,称为输入器件。
    \item $M_2$管是“共栅”部分,电流输入,电压输出,即$I_{D}\to V_{out}$,称为共源共栅器件。
\end{itemize}
由此可见,共源共栅的本质特点是:\empx{流经两个器件的信号电流是相同的}。本小节展示的共源共栅电路被称为套筒式,除此之外,还有另外一种共源共栅电路被称为折叠式,在下一小节介绍。

现在我们来分析套筒式共源共栅的大信号特性,\xref{fig:套筒式共源共栅放大器的特性}给出了\xref{fig:套筒式共源共栅级的电路}全面的仿真结果。

$M_1$管的代换关系是
\begin{Equation}
    V_{GS1}=V_{in}\qquad V_{DS1}=V_X\qquad V_{BS1}=0
\end{Equation}

$M_2$管的代换关系是
\begin{Equation}
    V_{GS2}=V_{b}-V_X\qquad V_{DS2}=V_{out}-V_X\qquad V_{BS2}=-V_X
\end{Equation}

当$V_{in}=<V_{TH1}$时,$M_1$管截止,$M_2$管事实上也截止(为什么?),此时有
\begin{Equation}
    V_{out}=V_{DD}\qquad V_{X}=V_b-V_{TH2}
\end{Equation}
当$V_{in}=V_{GS1}>V_{TH1}$时,$M_1$管和$M_2$管导通且处于饱和区,$V_{out},V_X$将会逐渐下降。

随着$V_{in}$的进一步增大,取决于电路的具体参数$V_b,R_D$的组合,可能出现两个结果。

\paragraph{$M_1$管先进入线性区}
$V_{in}$使$V_{DS1}\leq V_{GS1}-V_{TH1}$的条件先达到,即$M_1$管先进入线性区,代入代换关系
\begin{Equation}
    V_{X}\leq V_{in}-V_{TH1}
\end{Equation}
即
\begin{Equation}
    V_{X}-V_{in}\leq -V_{TH1}
\end{Equation}
这表明,当$V_X$比$V_{in}$低一个阈值电压$V_{TH1}$时,$M_1$将被迫进入线性区。

\paragraph{$M_2$管先进入线性区}
$V_{in}$使$V_{DS2}\leq V_{GS2}-V_{TH2}$的条件先达到,即$M_2$管先进入线性区,代入代换关系
\begin{Equation}
    V_{out}-V_{X}\leq V_b-V_X-V_{TH2}
\end{Equation}
注意到$V_X$可以被约掉
\begin{Equation}
    V_{out}-V_b\leq -V_{TH2}
\end{Equation}
这表明,当$V_{out}$比$V_b$低一个阈值电压$V_{TH2}$时,$M_2$将被迫进入线性区。

现在我们讨论一下共源共栅的小信号电路,如\xref{fig:套筒式共源共栅级的小信号电路}所示

\begin{Figure}[套筒式共源共栅级的小信号电路]
    \includegraphics[scale=0.8]{build/Chapter03D_06.fig.pdf}
\end{Figure}
第一步,分析共源共栅的增益,这里我们不做具体的计算,只是做一些定性的考量。分析增益时我们姑且忽略$r_{O1}, r_{O2}$,我们会注意到:\empx{共源共栅的增益等同于共源的增益},\empx{共栅级对于增益没有贡献}。这是因为所有电流最终都要流过共源级,因此,增益仍然是
\begin{Equation}
    A_V=-g_mR_D
\end{Equation}
第二步,分析共源共栅的输出电阻,这里可以适用一个有趣的近似,将共源极电阻$r_{O1}$视为共栅极的源极负反馈,由此可以立即套用\fancyref{fml:共栅放大器的等效输出电导}的结论
\begin{Equation}
    R_O=r_{O1}+r_{O2}[1+(g_{m2}+g_{mb2})r_{O1}]
\end{Equation}
展开为
\begin{Equation}
    R_O=r_{O1}+r_{O2}+(g_{m2}+g_{mb2})r_{O1}r_{O2}
\end{Equation}
由于转移电导总是比较大的,可以只保留
\begin{Equation}
    R_O=(g_{m2}+g_{mb2})r_{O1}r_{O2}
\end{Equation}
由此可见,共栅管$M_2$可以将共源管$M_1$的输出电阻自$r_{O1}$放大$r_{O2}(g_{m2}+g_{mb2})$倍。我们知道,当$R_D$很大时,增益要修正为$A_V=-g_m(R_{O}\parallel R_D)$并最终取决于$A_V=-g_mR_O$,换言之,输出电阻限制了通过增大负载电阻所能达到的增益上限,共源共栅提高了输出电阻,也就提高了这种上限,增益可以从$A_V=-g_{m1}r_{O1}$提升至$A_V=-g_{m1}(g_{m2}+g_{mb2})r_{O1}r_{O2}$,若近似认为$g_{m1}\approx g_{m2}+g_{mb2}$及$r_{O1}\approx r_{O2}$,即有:\empx{共源共栅可以将本征增益提高至原来的平方}。

\subsection{折叠式共源共栅}
共源共栅的设计思路是,将输入电压转化为电流作为共栅的输入,然而,输入器件和共源共栅器件不一定是同一种类型的,更确切的说,不必都是NMOS的,可以用PMOS和NMOS组合完成功能。这种器件选择上的变化,就让我们从套筒式共源共栅得到了折叠式共源共栅。

折叠式共源共栅的电路如\xref{fig:折叠式共源共栅级的电路}所示,我们指出
\begin{itemize}
    \item \xref{fig:使用PMOS作输入器件}中,输入器件$M_1$是PMOS,共源共栅器件$M_2$是NMOS。
    \item \xref{fig:使用NMOS作输入器件}中,输入器件$M_1$是NMOS,共源共栅器件$M_2$是PMOS。
    \item 折叠式共源共栅还需要一个恰当的电流源$I_0$作偏置。
\end{itemize}
\begin{Figure}[折叠式共源共栅级的电路]
    \begin{FigureSub}[使用PMOS作输入器件]
        \includegraphics[scale=0.8]{build/Chapter03D_04.fig.pdf}
    \end{FigureSub}
    \hspace{1cm}
    \begin{FigureSub}[使用NMOS作输入器件]
        \includegraphics[scale=0.8]{build/Chapter03D_05.fig.pdf}
    \end{FigureSub}
\end{Figure}
折叠式共源共栅中的“折叠式”是指,电流从VDD或GND出发,折叠一圈后又回到原位。

现在我们来分析折叠式共源共栅的大信号特性,\xref{fig:折叠式共源共栅放大器的特性}给出了\xref{fig:使用PMOS作输入器件}全面的仿真结果。

$M_1$管的代换关系是
\begin{Equation}
    V_{GS1}=V_{in}-V_{DD}\qquad V_{DS1}=V_X-V_{DD}\qquad V_{BS1}=0
\end{Equation}
$M_2$管的代换关系是
\begin{Equation}
    V_{GS2}=V_{b}-V_X\qquad V_{DS2}=V_{out}-V_X\qquad V_{BS2}=-V_X
\end{Equation}
这一点上,折叠式Cascode和套筒式Cascode完全一致。

\begin{Figure}[套筒式共源共栅放大器的特性]
    \begin{FigureSub}[中间电压;中间电压套筒4]
        \includegraphics[scale=0.58]{build/Chapter03D_01_2.fig.pdf}
    \end{FigureSub} \hspace{0.3cm}
    \begin{FigureSub}[中间电压;中间电压套筒3]
        \includegraphics[scale=0.58]{build/Chapter03D_01_6.fig.pdf}
    \end{FigureSub}\\ \vspace{0.5cm}
    \begin{FigureSub}[输出电压;输出电压套筒4]
        \includegraphics[scale=0.58]{build/Chapter03D_01_0.fig.pdf}
    \end{FigureSub} \hspace{0.3cm}
    \begin{FigureSub}[输出电压;输出电压套筒3]
        \includegraphics[scale=0.58]{build/Chapter03D_01_4.fig.pdf}
    \end{FigureSub}\\ \vspace{0.5cm}
    \begin{FigureSub}[电压增益;电压增益套筒4]
        \includegraphics[scale=0.58]{build/Chapter03D_01_1.fig.pdf}
    \end{FigureSub} \hspace{0.15cm}
    \begin{FigureSub}[电压增益;电压增益套筒3]
        \includegraphics[scale=0.58]{build/Chapter03D_01_5.fig.pdf}
    \end{FigureSub}\\ \vspace{0.5cm}
    \begin{FigureSub}[电流特性;电流特性套筒4]
        \includegraphics[scale=0.58]{build/Chapter03D_01_3.fig.pdf}
    \end{FigureSub} \hspace{0.15cm}
    \begin{FigureSub}[电流特性;电流特性套筒3]
        \includegraphics[scale=0.58]{build/Chapter03D_01_7.fig.pdf}
    \end{FigureSub}
\end{Figure}


\begin{Figure}[折叠式共源共栅放大器的特性]
    \begin{FigureSub}[中间电压;中间电压折叠4]
        \includegraphics[scale=0.58]{build/Chapter03D_02_2.fig.pdf}
    \end{FigureSub} \hspace{0.3cm}
    \begin{FigureSub}[中间电压;中间电压折叠3]
        \includegraphics[scale=0.58]{build/Chapter03D_02_6.fig.pdf}
    \end{FigureSub}\\ \vspace{0.5cm}
    \begin{FigureSub}[输出电压;输出电压折叠4]
        \includegraphics[scale=0.58]{build/Chapter03D_02_0.fig.pdf}
    \end{FigureSub} \hspace{0.3cm}
    \begin{FigureSub}[输出电压;输出电压折叠3]
        \includegraphics[scale=0.58]{build/Chapter03D_02_4.fig.pdf}
    \end{FigureSub}\\ \vspace{0.5cm}
    \begin{FigureSub}[电压增益;电压增益折叠4]
        \includegraphics[scale=0.58]{build/Chapter03D_02_1.fig.pdf}
    \end{FigureSub}
    \begin{FigureSub}[电压增益;电压增益折叠3]
        \includegraphics[scale=0.58]{build/Chapter03D_02_5.fig.pdf}
    \end{FigureSub}\\ \vspace{0.5cm}
    \begin{FigureSub}[电流特性;电流特性折叠4]
        \includegraphics[scale=0.58]{build/Chapter03D_02_3.fig.pdf}
    \end{FigureSub}
    \begin{FigureSub}[电流特性;电流特性折叠3]
        \includegraphics[scale=0.58]{build/Chapter03D_02_7.fig.pdf}
    \end{FigureSub}
\end{Figure}

折叠式Cascode的大信号分析可以重点关注电流,写出节点X的电流方程
\begin{Equation}
    I_0=I_{D2}-I_{D1}
\end{Equation}
$M_1$管的电流$I_{D1}$可以表示为(这里$V_{GS1}=V_{in}-V_{DD}$)
\begin{Equation}
    I_{D1}=-\mu_pC_{ox}(W/L)_1\qty(V_{in}-V_{DD}-V_{TH1})^2/2
\end{Equation}
$M_2$管的电流$I_{D2}$可以用$I_{D2}=I_0+I_{D1}$表示为
\begin{Equation}
    I_{D2}=I_0+I_{D1}=I_0-\mu_pC_{ox}(W/L)_1\qty(V_{in}-V_{DD}-V_{TH1})^2/2
\end{Equation}
\begin{itemize}
    \item 当$V_{in}>V_{DD}+V_{TH1}$时,将有$I_{D1}=0$和$I_{D2}=I_{0}$,电流完全流过$M_2$管,这是因为当出现$V_{GS1}>V_{TH1}$时$M_1$管会截止($M_1$是PMOS管),代换后即$V_{in}>V_{DD}+V_{TH1}$。
    \item 当$V_{in}<V_{in1}$时,将有$I_{D1}=-I_0$和$I_{D2}=0$,电流完全流过$M_1$管。
\end{itemize}

这里$V_{in1}$可以计算出来,由于$I_{D1}=-I_0$
\begin{Equation}
    I_{0}=\mu_pC_{ox}(W/L)_1\qty(V_{in1}-V_{DD}-V_{TH1})^2/2
\end{Equation}
容易解出
\begin{Equation}
    V_{in1}=V_{DD}+V_{TH1}-\sqrt{\frac{2I_0}{\mu_pC_{ox}(W/L)_1}}
\end{Equation}
这里开根号要取负号,因为显然$V_{in1}-V_{DD}-V_{TH1}<0$成立。