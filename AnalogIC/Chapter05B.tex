\section{共源共栅电流镜}
正如上一节所述,沟长调制使镜像电流产生了极大的误差。共源共栅电流镜试图解决该问题。

共源共栅电流镜的结构如\xref{fig:共源共栅电流镜}所示
\begin{Figure}[共源共栅电流镜]
    \includegraphics[scale=0.8]{build/Chapter05A_05.fig.pdf}
\end{Figure}

第一步,我们先解释$M_4$的作用,暂时忽略$M_3$管的存在而认为$M_4$的栅极接到了某个特定的偏置上。这里$M_2$与$M_4$构成了共源共栅结构,$M_2$共源,$M_4$共栅。我们知道,沟道长度效应对镜像特性的破坏是在于$V_{DS2}\neq V_{DS1}$,原先$V_{DS2}=V_{out}$使其会随$V_{out}$的变化而显著发生变化。共源共栅的想法就是,通过$M_4$管保护$M_4,M_2$的中间节点,使其电压$V_{D2}=V_{DS2}$随$V_{out}$的变化被削弱,换言之,$M_4$管在一定程度上,屏蔽了$V_{out}$的波动对$V_{DS2}$的影响。

这一结果得益于共源共栅的高输出电阻,根据\fancyref{fml:共源共栅放大器的输出电阻}
\begin{Equation}
    R_{out}=r_{O2}r_{O4}(g_{m4}+g_{mb4})
\end{Equation}
这里$R_{out}$是$M_2,M_4$的总电阻,而$M_2$的输出电阻为
\begin{Equation}
    R_{out,CS}=r_{O2}
\end{Equation}\goodbreak
因此,就有
\begin{Equation}
    V_{D2}=\frac{r_{O2}V_{out}}{r_{O2}r_{O4}(g_{m4}+g_{mb4})}
\end{Equation}
即
\begin{Equation}
    V_{D2}=V_{DS2}=\frac{V_{out}}{r_{O4}(g_{m4}+g_{mb4})}
\end{Equation}
这表明$V_{DS2}$的波动通过共源共栅结构,相较$V_{out}$被削弱了$r_{O4}(g_{m4}+g_{mb4})$倍。

共源共栅结构对中间节点的屏蔽作用实际上也可以在\xref{fig:中间电压套筒4}和\xref{fig:输出电压套筒4}的对比中看出。

\begin{Figure}[共源共栅电流镜的特性]
    \begin{FigureSub}[电流特性;共源共栅电流镜电流特性]
        \includegraphics[scale=0.58]{build/Chapter05A_02_0.fig.pdf}
    \end{FigureSub}
    \begin{FigureSub}[节点电压;共源共栅电流镜节点电压]
        \includegraphics[scale=0.58]{build/Chapter05A_02_1.fig.pdf}
    \end{FigureSub}\\ \vspace{0.25cm}
    \hspace*{0.12cm}
    \begin{FigureSub}[$M_2$管端口;共源共栅电流镜M2管端口]
        \includegraphics[scale=0.58]{build/Chapter05A_02_2.fig.pdf}
    \end{FigureSub}
    \hspace{0.12cm}
    \begin{FigureSub}[$M_4$管端口;共源共栅电流镜M4管端口]
        \includegraphics[scale=0.58]{build/Chapter05A_02_3.fig.pdf}
    \end{FigureSub}
\end{Figure}

第二步,共源共栅结构使$V_{DS2}$一定程度上屏蔽了$V_{out}$的波动,那么,我们该如何选取$M_3$的栅偏置$V_b$使得大信号下令$V_{DS2}=V_{DS1}$呢?精确的令$V_{DS2}=V_{DS1}$是不可能做到的,毕竟$V_{DS2}$仍然会随$V_{out}$波动。如\xref{fig:共源共栅电流镜}所示,我们在$M_1$管的上方,引入了一个同样二极管接法的$M_3$管来产生$V_b$,很显然$M_3$的存在完全不会影响$M_1$的工作。那么,为什么这种方式产生的$V_b$就是合适的?这是因为电路的对称性,当$V_{out}=V_{D4}$使$V_{D4}=V_{D3}$时,两侧电路电压分布就完全一致了,此时必有$V_{D2}=V_{D1}$即$V_{DS2}=V_{DS1}$使得$I_{out}=I_{REF}$。换言之,这种偏置方法使$V_{out}$在其变化范围内的某处能使$V_{DS2}=V_{DS1}$,而共源共栅结构的屏蔽效应又使$V_{DS2}$基本不会随$V_{out}$变化太多,这就是共源共栅电流镜的原理。\xref{fig:共源共栅电流镜的特性}给出了共源共栅电流镜的特性,可以看出,共源共栅确实显著改善了电流镜的特性,$I_{out}$几乎不随$V_{out}$变化了。
