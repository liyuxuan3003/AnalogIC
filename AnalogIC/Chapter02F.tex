\section{MOS的寄生电容}

有时,MOS的寄生电容也必须加以考虑,以研究高频特性。MOS共有四个端子,MOS的寄生电容存在于四个端子的任意两个间,不过源和漏间的电容是可以忽略的,如\xref{fig:MOS的寄生电容}所示。
\begin{Figure}[MOS的寄生电容]
    \includegraphics[scale=0.8]{build/Chapter02F_02.fig.pdf}
\end{Figure}

简而言之,MOS中的电容可以分为两部分
\begin{itemize}
    \item 栅相关的电容:$C_{GS}$,$C_{GD}$,$C_{GB}$。
    \item 体相关的电容:$C_{BS}$,$C_{BD}$。
\end{itemize}
电极间的电容不是凭空出现的,而是和MOS的结构有密切关系的,下面的工作可以分为两个步骤,第一步弄清MOS中的哪些结构存在电容,第二步搞清这些电容在电极间的分配。

\subsection{栅氧电容}
栅极和沟道间存在氧化层电容,其可以表示为
\begin{Equation}
    C_1=WLC_{ox}
\end{Equation}
其中$C_{ox}=\varepsilon_{ox}/t_{ox}$是栅氧的单位面积电容,而$W,L$和分别为沟道的长和宽。
\begin{BoxFormula}[栅氧电容]
    栅氧电容$C_1$的公式为
    \begin{Equation}
        C_1=WLC_{ox}
    \end{Equation}
\end{BoxFormula}

\subsection{覆盖电容}
栅极在理想情况下,应刚好与沟道边沿平齐,然而,扩散或离子注入形成源和漏时,水平扩散将不可避免的发生,致使栅也一定程度上会覆盖在源和漏上,这就是覆盖电容。按照\xref{subsec:MOS的结构}中的规定,横向扩散长度为$L_D$,因此,我们可能会理所当然的认为覆盖电容应表示为
\begin{Equation}
    C_2=WL_DC_{ox}
\end{Equation}
但由于覆盖电容位于电场边缘,电场线在横向上分布不均匀,这样计算并不准确。解决方案是,引入覆盖区的单位宽度电容$C_{ov}$(若边缘电场均匀,这里$C_{ov}$就是$C_{ox}L_D$),有
\begin{Equation}
    C_{ov}=WC_{ov}
\end{Equation}
其中$C_{ov}$是覆盖区的单位宽度电容,具体可能会记为$C_{ov}=C_{GSO}$或$C_{ov}=C_{GDO}$。

\begin{BoxFormula}[覆盖电容]
    覆盖电容$C_2$的公式为
    \begin{Equation}
        C_2=WC_{ov}
    \end{Equation}
\end{BoxFormula}

\subsection{耗尽电容}
沟道和栅极间的氧化层存在电容,沟道和衬底间的耗尽区同样存在电容,这就是耗尽区电容
\begin{Equation}
    C_3=WL\sqrt{q\varepsilon_{si}N_{sub}/(4\Phi_F)}
\end{Equation}
这里$N_{sub}$是衬底的掺杂浓度,而$\Phi_F$是费米势。
\begin{BoxFormula}[耗尽电容]
    耗尽电容$C_3$的公式为
    \begin{Equation}
        C_3=WL\sqrt{q\varepsilon_{si}N_{sub}/(4\Phi_F)}
    \end{Equation}
\end{BoxFormula}

\subsection{结电容}
结电容来自源漏与衬底间的PN结的势垒电容,这可以分为两个部分
\begin{itemize}
    \item 与结底部相关的极板电容$C_j$,$C_j$是一个单位面积电容。
    \item 与结四周相关的侧壁电容$C_{jsw}$,$C_{jsw}$是一个单位深度电容。
\end{itemize}
结电容$C_4$因此可以表示为
\begin{Equation}
    C_4=WEC_{j}+2(W+E)C_{jsw}
\end{Equation}
结的宽度可以认为与沟道宽度$W$一致,结的长度$E$则是一个新的自由变量。实践中,有时给出的是结的面积$A=WE$和结的周长$P=2(W+E)$,此时有$C_3=AC_j+PC_{jsw}$成立。

\begin{BoxFormula}[结电容]
    结电容$C_4$的公式为
    \begin{Equation}
        C_4=WEC_{j}+2(W+E)C_{jsw}
    \end{Equation}
    其中$C_j$和$C_{jsw}$分别满足
    \begin{Equation}
        C_{j}=C_{j0}/\qty[1+V_R/\Phi_B]^m\qquad
        C_{jsw}=C_{jsw0}/\qty[1+V_R/\Phi_B]^m
    \end{Equation}
    其中$V_R$为该结上的反偏电压。
\end{BoxFormula}

\subsection{MOS的寄生电容}
现在的问题是$C_1,C_2,C_3,C_4$是如何分配至$C_{GB},C_{GS},C_{GD},C_{BS},C_{BD}$上的呢?
\begin{itemize}
    \item 结电容$C_4$是较独立的,始终有$C_{BS}=C_4$和$C_{BD}=C_4$,但$C_{BS},C_{BD}$却未必是相同的,因为$V_R$不同。若衬底和源短接,$C_{BS}$中$V_R=V_{SB}=0$,$C_{BD}$中$V_R=V_{DB}=V_{DS}$。
    \item 覆盖电容$C_2$始终是$C_{GB}$和$C_{GD}$的一部分。
    \item 截止区,沟道未形成,栅氧电容$C_1$和耗尽电容$C_3$串接在栅和衬底间,构成$C_{GB}$。
    \item 线性区,沟道形成,栅氧电容$C_1$平均的分配给$C_{GS},C_{GD}$。
    \item 饱和区,沟道夹断且与漏断开,栅氧电容$C_1$折损至$2/3$且仅分配给$C_{GS}$。
    \item 栅氧电容$C_2$分配给$C_{GS},C_{GD}$的部分与其固有的覆盖电容$C_2$并联,即相加。
    \item 耗尽电容$C_3$在线性区和饱和区中不再被考虑,这些工作区中认为$C_{GB}=0$。
\end{itemize}
综上,我们可以总结出以下的公式。

\begin{BoxFormula}[栅--源电容]
    在截止区,栅--源电容$C_{GS}$仅包含覆盖电容$C_2$
    \begin{Equation}
        C_{GS}=C_2
    \end{Equation}
    在线性区,栅--源电容$C_{GS}$是覆盖电容$C_2$和栅氧电容$C_1/2$的并联
    \begin{Equation}
        C_{GS}=C_2+C_1/2
    \end{Equation}
    在饱和区,栅--源电容$C_{GS}$是覆盖电容$C_2$和栅氧电容$2C_1/3$的并联
    \begin{Equation}
        C_{GS}=C_2+2C_1/3
    \end{Equation}
\end{BoxFormula}

\begin{BoxFormula}[栅--漏电容]
    在截止区,栅--漏电容$C_{GD}$仅包含覆盖电容$C_2$
    \begin{Equation}
        C_{GD}=C_2
    \end{Equation}
    在线性区,栅--漏电容$C_{GD}$是覆盖电容$C_2$和栅氧电容$C_1/2$的并联
    \begin{Equation}
        C_{GD}=C_2+C_1/2
    \end{Equation}
    在饱和区,栅--漏电容$C_{GD}$是覆盖电容$C_2$
    \begin{Equation}
        C_{GD}=C_2
    \end{Equation}
\end{BoxFormula}

\begin{BoxFormula}[栅--衬电容]
    在截止区,栅--衬电容$C_{GB}$是栅氧电容$C_1$和耗尽电容$C_3$的串联
    \begin{Equation}
        C_{GB}=\frac{C_1C_3}{C_1+C_3}
    \end{Equation}
    在线性区和饱和区,栅--衬电容$C_{GB}$为零
    \begin{Equation}
        C_{GB}=0
    \end{Equation}
\end{BoxFormula}

\begin{BoxFormula}[衬--源电容]
    衬--源电容$C_{BS}$即源极和体的结电容$C_4$
    \begin{Equation}
        C_{BS}=C_4\qquad (V_R=V_{SB}=0)
    \end{Equation}
\end{BoxFormula}

\begin{BoxFormula}[衬--漏电容]
    衬--漏电容$C_{BD}$即漏极和体的结电容$C_4$
    \begin{Equation}
        C_{BD}=C_4\qquad (V_R=V_{DB}=V_{DS})
    \end{Equation}
\end{BoxFormula}

然而,我们可能会注意到一个严重的问题,对于$C_{GS},C_{GD},C_{GB}$,其表达式在三个工作区间是不连续的,这会导致收敛性问题,下面的结论改写了线性区$C_{GS},C_{GD}$的表达式,使线性区和饱和区间能平滑的过渡,这部分解决了该问题。耗尽区边界仍然将出现不连续的现象。
\begin{BoxFormula}[栅--源电容的线性区修正]
    在线性区,栅--源电容$C_{GS}$可以修正为
    \begin{Equation}
        C_{GS}=C_2+2C_1/3\qty{1-\frac{(V_{GS}-V_{DS}-V_{TH})^2}{\qty[2(V_{GS}-V_{TH})-V_{DS}]^2}}
    \end{Equation}
\end{BoxFormula}

\begin{BoxFormula}[栅--源电容的线性区修正]
    在线性区,栅--漏电容$C_{GD}$可以修正为
    \begin{Equation}
        C_{GS}=C_2+2C_1/3\qty{1-\frac{(V_{GS}-V_{TH})^2}{\qty[2(V_{GS}-V_{TH})-V_{DS}]^2}}
    \end{Equation}
\end{BoxFormula}

\xref{fig:MOS的寄生电容}是基于上述成果绘制的$C_{GS},C_{GD},C_{GB},C_{BS},C_{BD}$,关于这些图像,我们要说明两点。首先,源区和漏区的长度$E$是一个自由变量,这里我们取$E=2L=1.0\si{um}$。其次,耗尽电容$C_3$在绘图时被选择性忽略,因为计算发现$C_3\ll C_1$,若考虑$C_3$,则$C_{GB}$处于耗尽区时看起来也几乎为零($C_{GB}$在线性区和饱和区本身就是零了),使$C_{GB}$的图像变得非常乏味。\goodbreak

\begin{Figure}[MOS的寄生电容]
    \begin{FigureSub}[栅--源电容]
        \includegraphics[scale=0.68]{build/Chapter02F_01a.fig}
        \includegraphics[scale=0.5]{build/Chapter02F_01f.fig}
    \end{FigureSub}
    \hspace{0.1cm}
    \begin{FigureSub}[栅--漏电容]
        \includegraphics[scale=0.68]{build/Chapter02F_01b.fig}
        \includegraphics[scale=0.5]{build/Chapter02F_01f.fig}
    \end{FigureSub}\\ \vspace{0.2cm}
    \begin{FigureSub}[栅--体电容]
        \includegraphics[scale=0.7]{build/Chapter02F_01c.fig}
        \includegraphics[scale=0.5]{build/Chapter02F_01f.fig}
    \end{FigureSub}\\ \vspace{0.2cm}
    \begin{FigureSub}[衬--源电容]
        \includegraphics[scale=0.68]{build/Chapter02F_01d.fig}
        \includegraphics[scale=0.5]{build/Chapter02F_01f.fig}
    \end{FigureSub}
    \hspace{0.1cm}
    \begin{FigureSub}[衬--漏电容]
        \includegraphics[scale=0.68]{build/Chapter02F_01e.fig}
        \includegraphics[scale=0.5]{build/Chapter02F_01f.fig}
    \end{FigureSub}
\end{Figure}

\xref{fig:MOS的寄生电容}指出,MOS的电容大约在$\si{fF}$量级,是非常小的。在模拟集成电路中,我们通常只关心MOS处于饱和区时的电容。注意到$C_{GS}>C_{GD},C_{BD},C_{BS}$和$C_{GB}=0$。其中$C_{GS}$比较大是因为MOS中最大的电容来自栅氧电容,而处于饱和区时,只有$C_{GS}$会包含部分栅氧电容。

\begin{Tablex}[MOS处于饱和区($V_{DS}=V_{GS}=3\si{V}$)时的参数值;MOS处于饱和区时的参数值]{XX}
    <项&值\\>
    $C_{GS}$&$\SI{8.3947}{fF}$\\
    $C_{GD}$&$\SI{2.0000}{fF}$\\
    $C_{GB}$&$\SI{0.0000}{fF}$\\
    $C_{BS}$&$\SI{2.8406}{fF}$\\
    $C_{BD}$&$\SI{1.4777}{fF}$\\
    $g_m$&$\SI{0.3089}{mS}$\\
\end{Tablex}

\xref{tab:MOS处于饱和区时的参数值}是在$V_{DS}=3\si{V}$和$V_{GS}=3\si{V}$时,计算的一组电容和跨导值。这些参数将在\xref{chap:频率响应}讨论频率响应时(需在偏置点未给定时进行小信号分析),作为电容和跨导间的相对数量级参考。