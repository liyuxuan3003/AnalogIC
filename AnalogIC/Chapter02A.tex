\section{MOS的基本概念}

\subsection{MOS的结构}
MOS的结构我们已经很熟悉了,如\xref{fig:MOS的结构}所示,这里我们重点说明两点
\begin{itemize}
    \item 关于栅极尺寸:栅长$L$是源漏间的横向尺寸,栅宽$W$是对应的垂直尺寸(在\xref{fig:MOSFET的结构}中栅宽$W$位于垂直直面的方向上),栅氧化层的厚度记为$t_{ox}$。理想状态下,栅极与源漏的边界应当是刚好对齐的,但实际上,源漏区在形成时会不可避免的发生横向扩散,源漏间的实际距离(这是我们真正关心的,称为有效栅长)是略小于实际栅长的。为了避免混淆\footnote{这里drawn代表版图\textbf{绘制}的栅长,而eff表示effective即\textbf{有效}。},这里,栅极的实际长度记为$L_\ti{drawn}$,栅极的有效长度记为$L_\ti{eff}$,源漏的横向扩散长度记为$L_{D}$,显然有$L_\ti{eff}=L_\ti{drawn}-2L_{D}$。之后,当出现$L$时,我们总是指代有效长度$L=L_\ti{eff}$。栅长$L$和栅氧化层厚度$t_{ox}$对MOS的性能有非常重要的作用,MOS技术的发展的主旋律,就是在保证MOS的其他参数退化的的情况下,使$L$和$t_{ox}$不断减小。
    \item 关于衬底和阱:衬底作为MOS的第四端,也需要引至表面,衬底从功能上是需要轻掺杂的,因此引线处需要进行重掺杂,以便与引线金属形成欧姆接触。除此之外,我们知道,NMOS需要$p$衬底,PMOS需要$n$衬底,但是两者在CMOS工艺需要被作在同一块衬底上,这通常会是$p$衬底。因此,PMOS就需要被作在$n$阱中,阱可以被认为是一种“局部衬底”。这也指出:\empx{所有NMOS共享一个衬底,而每个PMOS的衬底是相互独立的阱}。换言之,PMOS可以单独的调整阈值电压,PMOS的这种灵活性在一些模拟电路中会被应用(另外,双阱工艺可以使NMOS和PMOS同时具有这种灵活性)。
\end{itemize}

通常,NMOS的衬底与最低电位(GND)相连,PMOS的衬底与最高电位(VDD)相连。

\begin{Figure}[MOS的结构]
    \includegraphics{build/Chapter02A_02.fig.pdf}
\end{Figure}

\subsection{MOS的符号}
MOS的符号如\xref{fig:MOS的符号}所示,这里说明几点
\begin{itemize}
    \item MOS是四端器件,但由于衬底一般接地或接电源,通常可以省略。
    \item 数字电路中,习惯用栅极是否有圈区分PMOS和NMOS,NMOS无圈,PMOS有圈。
    \item 模拟电路中,习惯用箭头方向区分PMOS和NMOS,箭头位于源侧(这可以有效区分源漏),箭头方向代表了衬底--源结的正向,NMOS箭头指向源,PMOS箭头背向源。
    \item 这里为适应两种习惯,同时标注了圈和箭头(某种意义上,也是一种异端)。
    \item PMOS的源位于上侧(源电位高),NMOS的源位于下侧(源电位低)。
\end{itemize}
\begin{Figure}[MOS的符号]
    \begin{FigureSub}[三端NMOS]
        \includegraphics{build/Chapter02A_03.fig.pdf}
    \end{FigureSub}
    \qquad
    \begin{FigureSub}[三端PMOS]
        \includegraphics{build/Chapter02A_04.fig.pdf}
    \end{FigureSub}
    \qquad
    \begin{FigureSub}[四端NMOS]
        \includegraphics{build/Chapter02A_05.fig.pdf}
    \end{FigureSub}
    \qquad
    \begin{FigureSub}[四端PMOS]
        \includegraphics{build/Chapter02A_06.fig.pdf}
    \end{FigureSub}
\end{Figure}