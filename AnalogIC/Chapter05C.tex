\section{五管运算跨导放大器}
\uwave{有源电流镜}(Active Current Mirror)是指如\xref{fig:五管运算跨导放大器}所示的电路,尽管名称中包含电流镜,但相较于基本电流镜和共源共栅电流镜,有源电流镜可以说是“来混的”。它实际上是将电流镜的结构用作差分放大器的负载,由于这种用法下,电流镜中的晶体管也会参与放大,故称为有源。有源电流镜更为恰当的一个名称是\uwave{五管运算跨导放大器}(Operational Transconductance Amplifier, OTA),换言之,这本质上就是一种特定的放大器,而不是什么复制电流的玩意。

顺着“五管运算跨导放大器”的名称,我们来分析一下这个电路的结构
\begin{itemize}
    \item 名称中的“五管”是指该结构由五个晶体管构成,$M_1,M_2$是属于差分放大器的一对差分管,$M_3,M_4$是电流镜,它们充当了差分放大器的负载,但请特别注意的是这里$M_3,M_4$是PMOS电流镜而不是NMOS电流镜。$M_5$充当了尾电流源$I_{SS}$及其并联电阻$R_{SS}$。
    \item 名称中的“运算放大器”我们在模拟电路中很熟悉,但是它到底是什么意思?较一般的说,\empx{运算放大器泛指一切差分输入单端输出的放大器}。这里输入$V_{in1},V_{in2}$是差分的,但却只有一个输出$V_{out}$,符合运算放大器的定义。事实上,从差分放大器的角度,该电路的负载是高度不对称的,我们后面会看到$V_{out1},V_{out2}$的变化差异很大,也确实并不适合差分输出,而选取$V_{out2}$作为输出端$V_{out}$就是因为其性质比较好,这是以牺牲$V_{out1}$为代价的。关于记号,弃用的输出端$V_{out1}$记为$V_F$,尾电流源上的中间节点照例记为$V_P$。
\end{itemize}

\begin{Figure}[五管运算跨导放大器]
    \includegraphics[scale=0.8]{build/Chapter05C_03.fig.pdf}
\end{Figure}

接下来,我们从差模和共模的特性,分别分析这个电路。

\subsection{五管运算跨导放大器的差模特性}
\xref{fig:五管运算跨导放大器的差模特性}给出了OTA的差模特性的仿真,在定量计算前,让我们来试着定性分析一下。

\begin{Figure}[五管运算跨导放大器的差模特性]
    \begin{FigureSub}[电压特性;五管差模电压]
        \includegraphics[scale=0.58]{build/Chapter05C_01_0.fig.pdf}
    \end{FigureSub} \hspace{0.2cm}
    \begin{FigureSub}[增益特性;五管差模增益]
        \includegraphics[scale=0.58]{build/Chapter05C_01_1.fig.pdf}
    \end{FigureSub} \\ \vspace{0.5cm}
    %----------------------------------%
    \begin{FigureSub}[电流特性;五管差模电流]
        \includegraphics[scale=0.58]{build/Chapter05C_01_2.fig.pdf}
    \end{FigureSub} \hspace{0.2cm}
    \begin{FigureSub}[$M_5$管;五管差模5]
        \includegraphics[scale=0.58]{build/Chapter05C_01_7.fig.pdf}
    \end{FigureSub} \\ \vspace{0.5cm}
    %----------------------------------%
    \begin{FigureSub}[$M_1$管;五管差模1]
        \includegraphics[scale=0.58]{build/Chapter05C_01_3.fig.pdf}
    \end{FigureSub} \hspace{0.2cm}
    \begin{FigureSub}[$M_2$管;五管差模2]
        \includegraphics[scale=0.58]{build/Chapter05C_01_4.fig.pdf}
    \end{FigureSub} \\ \vspace{0.5cm}
    %----------------------------------%
    \begin{FigureSub}[$M_3$管;五管差模3]
        \includegraphics[scale=0.58]{build/Chapter05C_01_5.fig.pdf}
    \end{FigureSub} \hspace{0.2cm}
    \begin{FigureSub}[$M_4$管;五管差模4]
        \includegraphics[scale=0.58]{build/Chapter05C_01_6.fig.pdf}
    \end{FigureSub}
    %----------------------------------%
\end{Figure}

\xref{fig:五管运算跨导放大器的差模特性}中,关于各MOS管的端口特性曲线,务必注意NMOS和PMOS的差异
\begin{itemize}
    \item 绿线代表$V_{DS}$,紫线代表$V_{GS}-V_{TH}$
    \item 对于NMOS,饱和区的判据是$V_{DS}>V_{GS}-V_{TH}$,绿线在紫线上为饱和区。
    \item 对于PMOS,饱和区的判据是$V_{DS}<V_{GS}-V_{TH}$,绿线在紫线下为饱和区。
\end{itemize}

\xref{fig:五管运算跨导放大器的差模特性}使用电路的参数如下
\begin{framed}
    \begin{Equation}*
        V_{DD}=4\si{V}\qquad
        V_{b}=1.2\si{V}\qquad 
        V_{in,CM}=V_{DD}/2=2\si{V}
    \end{Equation}
\end{framed}

我们分析如下\footnote{这里PMOS的$I_{D3},I_{D4}$的方向顺从其串联NMOS的$I_{D1},I_{D3}$的方向,自上而下,而不是由漏至源。}
\begin{itemize}
    \item 在平衡点$V_{in1}=V_{in2}=V_{in,CM}$附近,$M_2,M_4$管均处于饱和区。
    \item 若$V_{in}$略微增大,$V_{in1}$增大使$I_{D1}$增大,$V_{in2}$减小使$I_{D2}$减小,而$M_3$会通过电流镜使通过$M_4$的电流$I_{D4}$增大,然而应有$I_{D4}=I_{D2}$,但两者变化趋势相反,此时电流较小的$I_{D2}$将“取胜”,$M_4$被拖入线性区使$I_{D4}$降低,这意味着$V_{out}$将被迅速上拉至$V_{DD}$。
    \item 若$V_{in}$略微减小,$V_{in1}$减小使$I_{D1}$减小,$V_{in2}$增大使$I_{D2}$增大,而$M_3$会通过电流镜使通过$M_4$的电流$I_{D4}$减小,然而应有$I_{D4}=I_{D2}$,但两者变化趋势相反,此时电流较小的$I_{D4}$将“取胜”,$M_2$被拖入线性区使$I_{D2}$降低,这意味着$V_{out}$将被迅速下拉至$V_{P}$。
    \item 若$V_{in}$很大,则$M_2$管将截止。
    \item 若$V_{in}$很小,则$M_1$管将截止,期间还会经历$M_5$管由饱和区转变为线性区。
    \item \xref{fig:五管运算跨导放大器的差模特性}中,饱和--线性边界由黑点标识,饱和--截止边界由灰点标识。
\end{itemize}

现在我们进行定量分析,目标是解算$A_{DM}$,方法是分别求出$G_m$和$R_{out}$的表达式。分析电路如\xref{fig:五管运算跨导放大器的差模分析}所示\footnote{这里PMOS的$I_{D3},I_{D4}$的方向是符合习惯的由漏至源。},计算$G_m$时短接输出考察$I_{out}/V_{in}$,计算$R_{out}$时短接输入考察$V_{out}/I_{out}$。

第一步,计算转移电导$G_m$,如\xref{fig:计算转移电导OTA差模}所示,首先我们注意到
\begin{Equation}
    I_{D1}=+g_{m1}V_{in}/2\qquad I_{D2}=-g_{m2}V_{in}/2
\end{Equation}
这是MOS管的基本原理,其中的正负号和$1/2$都是来自$M_1,M_2$的输入电压$\pm V_{in}/2$。

由于$I_{D3}$和$I_{D1}$是串联线路上的同一电流,但取向相反
\begin{Equation}
    I_{D3}=-I_{D1}=-g_{m1}V_{in}/2
\end{Equation}
由于$I_{D3}$和$I_{D4}$间的电流镜关系
\begin{Equation}
    I_{D4}=+I_{D3}=-g_{m1}V_{in}/2
\end{Equation}

\begin{Figure}[五管运算跨导放大器的差模分析]
    \begin{FigureSub}[计算转移电导;计算转移电导OTA差模]
        \includegraphics[scale=0.8]{build/Chapter05C_04.fig.pdf}
    \end{FigureSub}
    \hspace{0.25cm}
    \begin{FigureSub}[计算输出电阻;计算输出电阻OTA差模]
        \includegraphics[scale=0.8]{build/Chapter05C_05.fig.pdf}
    \end{FigureSub}
\end{Figure}

这里有疑问,这里$M_1,M_3$串联,而$M_2,M_4$也串联,为什么$I_{D3}=-I_{D1}$但$I_{D4}\neq -I_{D2}$?这是因为$M_2$和$M_4$间还有来自输出端口的$I_{out}$,所以显然不等,相反,$I_{out}$可以表示为
\begin{Equation}
    I_{out}=I_{D4}+I_{D2}=-(g_{m1}+g_{m2})V_{in}/2
\end{Equation}
改记$g_{m1}=g_{m2}=g_{m1,2}$
\begin{Equation}
    I_{out}-g_{m1,2}V_{in}
\end{Equation}
因此
\begin{Equation}
    G_m=\frac{I_{out}}{V_{in}}=-g_{m1,2}
\end{Equation}
这是相当简洁的一个结论,这种简洁性,很大程度上来自我们在\xref{fig:计算转移电导OTA差模}中假定节点$V_P$是接地的。照道理作为电流源,节点$V_P$应该与地断路,但由于$V_P$近似不变,可认为其“虚地”。
\begin{BoxFormula}[五管运放的转移电导]
    五管运放的转移电导$G_m$为
    \begin{Equation}
        G_m=-g_{m1,2}
    \end{Equation}
\end{BoxFormula}\setpeq{OTA输出电阻}

第二步,计算输出电阻$R_{out}$,如\xref{fig:计算输出电阻OTA差模}所示,这一次我们不再认为$V_P$虚地了。在更准确的计算前,首先我们先做一些比较粗糙的分析,姑且认为$V_F$不变是一个虚地点,当$I_{out}$进入后,沿着$M_2\to M_1\to V_F$和$M_4\to V_{DD}$两条路径到达地,两者电阻分别为$r_{O2}+r_{O1}$和$r_{O4}$
\begin{Equation}
    R_{out}=(r_{O2}+r_{O1})\parallel r_{O4}
\end{Equation}
改记$r_{O1}=r_{O2}=r_{O1,2}$,改记$r_{O3}=r_{O4}=r_{O3,4}$
\begin{Equation}
    R_{out}=2r_{O1,2}\parallel r_{O3,4}
\end{Equation}

然而认为$V_F$虚地是不准确的,我们需要更严格的分析过程。由于$V_F$不再接地,电流$I_{D2}$的路径会由$M_2\to M_1\to V_F$扩展至$M_2\to M_1\to M_3\to V_{DD}$,其中,$M_2,M_1$的电阻仍然分别是$r_{O2},r_{O2}$,$M_3$是二极管接法的MOS管,通过其的电阻为$r_{O3}\parallel g_{m3}^{-1}$,故$I_{D2}$就可以表示为
\begin{Equation}&[1]
    I_{D2}=\frac{V_{out}}{r_{O1}+r_{O2}+r_{O3}\parallel g_{m3}^{-1}}
\end{Equation}

这里$I_{D4}$要复杂些,因为$M_4$的栅压仍在变化,是有源的,故
\begin{Equation}&[2]
    I_{D4}=\frac{V_{out}}{r_{O4}}+g_{m4}V_{GS4}
\end{Equation}
这里$V_{GS4}$如何计算,由于$V_{GS4}=V_{GS3}=V_{DS3}$,而后者就是$I_{D2}$在$M_3$上的压降
\begin{Equation}&[3]
    V_{GS4}=I_{D2}(r_{O3}\parallel g_{m3}^{-1})
\end{Equation}
将\xrefpeq{1}代入\xrefpeq{3}
\begin{Equation}&[4]
    V_{GS4}=\frac{V_{out}}{r_{O1}+r_{O2}+r_{O3}\parallel g_{m3}^{-1}}(r_{O3}\parallel g_{m3}^{-1})
\end{Equation}
将\xrefpeq{4}代入\xrefpeq{2}
\begin{Equation}&[5]
    I_{D4}=\frac{V_{out}}{r_{O1}+r_{O2}+r_{O3}\parallel g_{m3}^{-1}}(r_{O3}\parallel g_{m3}^{-1})g_{m4}+\frac{V_{out}}{r_{O4}}
\end{Equation}
总电流$I_{out}=I_{D2}+I_{D4}$中代入\xrefpeq{1}和\xrefpeq{5}
\begin{Equation}&[6]
    \qquad\qquad
    I_{out}=\frac{V_{out}}{r_{O1}+r_{O2}+r_{O3}\parallel g_{m3}^{-1}}+\frac{V_{out}}{r_{O1}+r_{O2}+r_{O3}\parallel g_{m3}^{-1}}(r_{O3}\parallel g_{m3}^{-1})g_{m4}+\frac{V_{out}}{r_{O4}}
    \qquad\qquad
\end{Equation}
化简
\begin{Equation}
    I_{out}=\frac{V_{out}}{r_{O1}+r_{O2}+r_{O3}\parallel g_{m3}^{-1}}\qty[1+(r_{O3}\parallel g_{m3}^{-1})g_{m4}]+\frac{V_{out}}{r_{O4}}
\end{Equation}
做一些近似,我们认为$r_{O1}+r_{O2}\gg r_{O3}\parallel g_{m3}^{-1}$以及$(r_{O3}\parallel g_{m3^{-1}})g_{m4}=1$
\begin{Equation}
    I_{out}=\frac{2V_{out}}{r_{O1}+r_{O2}}+\frac{V_{out}}{r_{O4}}
\end{Equation}
改记$r_{O1}=r_{O2}=r_{O1,2}$,改记$r_{O3}=r_{O4}=r_{O3,4}$
\begin{Equation}
    I_{out}=\frac{V_{out}}{r_{O1,2}}+\frac{V_{out}}{r_{O3,4}}
\end{Equation}
因此
\begin{Equation}
    G_{out}=\frac{I_{out}}{V_{out}}=\frac{1}{r_{O1,2}}+\frac{1}{r_{O3,4}}
\end{Equation}
进而得到(倒数和的倒数等于并联)
\begin{Equation}
    R_{out}=r_{O1,2}\parallel r_{O3,4}
\end{Equation}
由此可见,$R_{out}$正确的结果是$r_{O1,2}\parallel r_{O3,4}$而不是前面的$2r_{O1,2}\parallel r_{O3,4}$,不过很讽刺的是,假如这里我们同样认为$V_P$虚地,则可以立即通过观察得到$R_{out}=r_{O1,2}\parallel r_{O3,4}$的正确结论。
\begin{BoxFormula}[五管运放的输出电阻]
    五管运放的输出电阻$R_{out}$为
    \begin{Equation}
        R_{out}=r_{O1,2}\parallel r_{O3,4}
    \end{Equation}
\end{BoxFormula}

综上,根据\fancyref{fml:五管运放的转移电导}和\fancyref{fml:五管运放的输出电阻},得到
\begin{BoxFormula}[五管运放的差模增益]
    五管运放的差模增益$A_{DM}$为
    \begin{Equation}
        A_{DM}=g_{m1,2}(r_{O1,2}\parallel r_{O3,4})
    \end{Equation}
\end{BoxFormula}

\subsection{五管运算跨导放大器的共模特性}
\xref{fig:五管运算跨导放大器的共模特性}给出了OTA的共模特性的仿真,我们注意到,当$V_{in,CM}$从$V_{DD}/2$处开始增大时,输出电压$V_{out}$略微的减小,因而该结构是存在共模增益$A_{CM}$的。在进一步讨论之前,我们要指出的是,由于这里是差分输入$V_{in1},V_{in2}$单端输出$V_{out}$,因此,共模增益$A_{CM}$和共模--差模增益$A_{CM-DM}$两个概念发生了简并,仍采用$A_{CM}$的记号。故稍后计算共模抑制比CMRR时,请不要为$\te{CMRR}=\abs{A_{DM}/A_{CM}}$的公式感到惊讶,现在$A_{CM}$和$A_{CM-DM}$是同一回事了。

\xref{fig:五管运算跨导放大器的共模特性}还指出了一个有趣的现象,尽管电路不完全对称,例如$M_3$采用了二极管接法但其对位的$M_4$没有采用。但是共模输入时,电路两侧的所有的电压和电流都完全相同,这意味着此处完全可以采用对称差分电路的共模分析结论。分析电路如\xref{fig:五管运算跨导放大器的共模分析}所示,$V_F$和$V_{out}$的短接是对称性的结果(这也意味着$M_3,M_4$现在都是二极管接法),$R_{SS}=r_{O5}$作为尾电流源的并联电阻被考虑,这里我们在方法上不再弯弯绕绕的弄什么$G_m$和$R_{out}$,而是直接计算增益$A_{CM}$。
\begin{Figure}[五管运算跨导放大器的共模分析]
    \includegraphics[scale=0.8]{build/Chapter05C_06.fig.pdf}
\end{Figure}

\clearpage

\begin{Figure}[五管运算跨导放大器的共模特性]
    \begin{FigureSub}[电压特性;五管共模电压]
        \includegraphics[scale=0.58]{build/Chapter05C_02_0.fig.pdf}
    \end{FigureSub} \hspace{0.2cm}
    \begin{FigureSub}[增益特性;五管共模增益]
        \includegraphics[scale=0.58]{build/Chapter05C_02_1.fig.pdf}
    \end{FigureSub} \\ \vspace{0.5cm}
    %----------------------------------%
    \begin{FigureSub}[电流特性;五管共模电流]
        \includegraphics[scale=0.58]{build/Chapter05C_02_2.fig.pdf}
    \end{FigureSub} \hspace{0.2cm}
    \begin{FigureSub}[$M_5$管;五管共模5]
        \includegraphics[scale=0.58]{build/Chapter05C_02_7.fig.pdf}
    \end{FigureSub} \\ \vspace{0.5cm}
    %----------------------------------%
    \begin{FigureSub}[$M_1$管;五管共模1]
        \includegraphics[scale=0.58]{build/Chapter05C_02_3.fig.pdf}
    \end{FigureSub} \hspace{0.2cm}
    \begin{FigureSub}[$M_2$管;五管共模2]
        \includegraphics[scale=0.58]{build/Chapter05C_02_4.fig.pdf}
    \end{FigureSub} \\ \vspace{0.5cm}
    %----------------------------------%
    \begin{FigureSub}[$M_3$管;五管共模3]
        \includegraphics[scale=0.58]{build/Chapter05C_02_5.fig.pdf}
    \end{FigureSub} \hspace{0.2cm}
    \begin{FigureSub}[$M_4$管;五管共模4]
        \includegraphics[scale=0.58]{build/Chapter05C_02_6.fig.pdf}
    \end{FigureSub}
    %----------------------------------%
\end{Figure}\setpeq{OTA共模增益}

直接算$A_{CM}$是因为可以套用\fancyref{fml:差分放大器的共模增益}的结论
\begin{Equation}&[1]
    A_{CM}=-\frac{R_D/2}{R_{SS}+(2_{g_m})^{-1}}
\end{Equation}
这个例子中
\begin{Equation}&[2]
    R_{D}=r_{O3}\parallel g_{m3,4}^{-1}\qquad R_{SS}=r_{O5}\qquad g_{m}=g_{m1,2}
\end{Equation}
将\xrefpeq{2}代入\xrefpeq{1}
\begin{Equation}
    A_{CM}=-\frac{(r_{O3,4}\parallel g_{m3,4}^{-1})/2}{r_{O5}+(2g_{m1,2})^{-1}}
\end{Equation}
做一些化简,上下同乘$2g_{m1,2}$,并近似认为$r_{O3,4}\parallel g_{m3,4}^{-1}=g_{m3,4}^{-1}$
\begin{Equation}
    A_{CM}=-\frac{g_{m1,2}g_{m3,4}^{-1}}{1+2g_{m1,2}r_{O5}}
\end{Equation}
或者记作
\begin{Equation}
    A_{CM}=-\frac{1}{1+2g_{m1,2}r_{O5}}\frac{g_{m1,2}}{g_{m3,4}}
\end{Equation}
整理结论如下
\begin{BoxFormula}[五管运放的共模增益]
    五管运放的共模增益$A_{CM}$为
    \begin{Equation}
        A_{CM}=-\frac{1}{1+2g_{m1,2}r_{O5}}\frac{g_{m1,2}}{g_{m3,4}}
    \end{Equation}
\end{BoxFormula}
从\xref{fig:五管共模增益}和\xref{fig:五管差模增益}的对比上可以看出,五管运放的差模增益$A_{CM}$要比共模增益$A_{DM}$大的多,但是,我们还是希望能对共模抑制比$\te{CMRR}=|A_{DM}/A_{CM}|$的数量级有一些直观认知。

根据\fancyref{fml:五管运放的差模增益}和\fancyref{fml:五管运放的共模增益}
\begin{Equation}
    \te{CMRR}=\abs{\frac{A_{DM}}{A_{CM}}}=g_{m1,2}(r_{O1,2}\parallel r_{O3,4})(1+2g_{m1,2}r_{O5})\frac{g_{m3,4}}{g_{m1,2}}
\end{Equation}
约掉一个$g_{m1,2}$
\begin{Equation}
    \te{CMRR}=g_{m3,4}(r_{O1,2}\parallel r_{O3,4})(1+2g_{m1,2}r_{O5})
\end{Equation}
作为数量级估计,假定所有的$g_{mx}=g_m$,假定所有的$r_{Ox}=r_{O}$
\begin{Equation}
    \te{CMRR}=g_{m}(r_O/2)\qty(1+2g_{m}r_O)
\end{Equation}
有理由相信$2g_{m}r_{O}\gg 1$
\begin{Equation}
    \te{CMRR}=(g_mr_{O})^2
\end{Equation}
这个结果是相当不错的,即五管运放的CMRR可以达到$g_mr_O$的平方的数量级。
\begin{BoxFormula}[五管运放的共模抑制比]
    五管运放的共模抑制比CMRR为
    \begin{Equation}
        \te{CMRR}=(g_mr_O)^2
    \end{Equation}
\end{BoxFormula}
