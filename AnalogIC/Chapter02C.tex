\section{沟道调制}
\uwave{沟道调制}(Channel Modulation)是指,当MOS管进入饱和区,沟道夹断后,随着$V_{DS}$的进一步增大,沟道的实际长度是在减小的,换言之$L$在减小。而我们知道,电流$I_D$是正比于宽长比$(W/L)$的,因此,电流$I_D$在进入饱和区后并不会如\xref{fig:理想特性输出特性曲线}所示保持恒定,而是会随着$V_{DS}$进一步增大,经验表明,这种增大是线性的,可以用$\lambda V_{DS}$表征,即饱和区公式
\begin{Equation}
    I_D=\mu_nC_{ox}\frac{W}{L}\qty[(V_{GS}-V_{TH})^2/2]
\end{Equation}
应当被修正为,这里$\lambda$称为\uwave{沟道调制系数}
\begin{Equation}
    I_D=\mu_nC_{ox}\frac{W}{L}\qty[(V_{GS}-V_{TH})^2/2](1+\lambda V_{DS})
\end{Equation}
应当指出的是,沟道调制只存在于饱和区,线性区并不存在沟道调制效应,这是因为在线性区沟道从源端到漏端是连续的,$V_{DS}$不能调制沟道的长度。换言之,线性区的$I_D$公式后不应当乘以$(1+\lambda V_{DS})$因子。然而,这会导致一个明显的问题,线性区和饱和区的$I_D$公式将不连续。这种不连续性将为计算和分析带来困难,为了解决这个问题,我们可以抛开事实不谈,从建模的角度为线性区$I_D$公式添加$(1+\lambda V_{DS})$使$I_D$连续化,讨论线性区时令$\lambda=0$即可。
% \footnote{有趣的是,MOS的Level 1的SPICE模型中,关于沟道调制,也是这么建模的。}

\begin{BoxFormula}[MOS的沟道调制]
    当$V_{GS}>V_{TH}, V_{DS}<V_{GS}-V_{TH}$,处于线性区,考虑沟道调制
    \begin{Equation}
        I_D=\mu_nC_{ox}\frac{W}{L}[(V_{GS}-V_{TH})V_{DS}-V_{DS}^2/2](1+\lambda V_{DS})
    \end{Equation}
    当$V_{GS}>V_{TH}, V_{DS}>V_{GS}-V_{TH}$,处于饱和区,考虑沟道调制
    \begin{Equation}
        I_D=\mu_nC_{ox}\frac{W}{L}[(V_{GS}-V_{TH})^2/2](1+\lambda V_{DS})
    \end{Equation}
\end{BoxFormula}

再次注意,线性区的沟道调制项$(1+\lambda V_{DS})$只是建模考虑!考虑沟道调制后,特性曲线与特性曲面如\xref{fig:MOS的沟道调制}所示,明显的改变是,输出特性曲线发生上翘,且$V_{GS}$越大曲线随$V_{DS}$的上翘幅度越大,这是因为斜率是$\lambda$与饱和电流的乘积,而$V_{GS}$越大,饱和发生时的电流也越大。

\begin{Figure}[MOS的沟道调制]
    \begin{FigureSub}[输出特性曲线;沟道调制输出特性曲线]
        \includegraphics[scale=0.83]{build/Chapter02A_01e.fig.pdf}
    \end{FigureSub}
    \begin{FigureSub}[转移特性曲线;沟道调制转移特性曲线]
        \includegraphics[scale=0.83]{build/Chapter02A_01f.fig.pdf}
    \end{FigureSub}\\ \vspace{0.5cm}
    \begin{FigureSub}[特性曲面;沟道调制特性曲面]
        \includegraphics{build/Chapter02A_01d.fig.pdf}
    \end{FigureSub}
\end{Figure}
现在,让我们考虑一个重要的问题。MOS的电流$I_D=I_{D}(V_{DS},V_{GS})$是一个二元函数,在微积分中,我们知道,二元函数是可以求偏导数的,那么$I_D$的两个偏导数分别是什么?
\begin{itemize}
    \item $I_D(V_{DS},V_{GS})$对$V_{GS}$的偏导数,称为\uwave{转移电导}(Transconductance),也称为\uwave{跨导}。
    \item $I_D(V_{DS},V_{GS})$对$V_{DS}$的偏导数,称为\uwave{输出电导}(Output Conductance)。
\end{itemize}

正式定义如下
\begin{BoxDefinition}[MOS的转移电导]
    转移电导定义为$I_D$对$V_{GS}$的偏导
    \begin{Equation}
        g_m=\pdv{I_D}{V_{GS}}
    \end{Equation}
\end{BoxDefinition}

\begin{BoxDefinition}[MOS的输出电导]
    输出电导定义为$I_D$对$V_{DS}$的骗到
    \begin{Equation}
        g_O=\pdv{I_D}{V_{DS}}
    \end{Equation}
\end{BoxDefinition}

这里值得注意的是,转移电导$g_m$是通常的提法,它代表了$I_D$对$V_{GS}$变化的敏感程度,转移电导$g_m$也是MOS管用于放大的关键参数。输出电导$g_O$则不是,输出电导$g_O$的本质就是沟道的小信号电导,故应用中,我们关心的往往都是$g_O$的倒数,即,输出电阻$r_O=1/g_O$。

\begin{Figure}[转移电导与输出电导]
    \begin{FigureSub}[转移电导(理想特性)]
        \includegraphics[scale=0.77]{build/Chapter02A_01k.fig.pdf}
    \end{FigureSub}
    \begin{FigureSub}[输出电导(理想特性)]
        \includegraphics[scale=0.77]{build/Chapter02A_01l.fig.pdf}
    \end{FigureSub}\\ \vspace{0.75cm}
    \begin{FigureSub}[转移电导(沟道调制)]
        \includegraphics[scale=0.77]{build/Chapter02A_01m.fig.pdf}
    \end{FigureSub}
    \begin{FigureSub}[输出电导(沟道调制)]
        \includegraphics[scale=0.77]{build/Chapter02A_01n.fig.pdf}
    \end{FigureSub}
\end{Figure}

\xref{fig:转移电导与输出电导}绘制了$g_m$和$g_O$的图像,其公式将在稍后给出。

\subsection{转移电导}
\begin{BoxFormula}[MOS的转移电导]
    转移电导$g_m$在线性区可以表示为
    \begin{Equation}&[A]
        g_m=\mu_nC_{ox}\frac{W}{L}[V_{DS}+\lambda V_{DS}^2]
    \end{Equation}
    转移电导$g_m$在饱和区可以表示为
    \begin{Equation}&[B]
        g_m=\mu_nC_{ox}\frac{W}{L}[(V_{GS}-V_{TH})+\lambda V_{DS}(V_{GS}-V_{TH})]
    \end{Equation}
    特别的,当$\lambda=0$时,在线性区($V_{DS}<V_{GS}-V_{TH}$)的转移电导为
    \begin{Equation}&[C]
        g_m=\mu_nC_{ox}\frac{W}{L}V_{DS}
    \end{Equation}
    特别的,当$\lambda=0$时,在饱和区($V_{DS}>V_{GS}-V_{TH}$)的转移电导为
    \begin{Equation}&[D]
        g_m=\mu_nC_{ox}\frac{W}{L}(V_{GS}-V_{TH})
    \end{Equation}
    或者可以写为
    \begin{Equation}
        g_m=\sqrt{2\mu_nC_{ox}\frac{W}{L}I_D}
    \end{Equation}
    或者可以写为
    \begin{Equation}
        g_m=\frac{2I_{D}}{V_{GS}-V_{TH}}
    \end{Equation}
\end{BoxFormula}

从\xref{fig:转移电导(理想特性)}中,我们看到,转移电导$g_m$通常在$\si{mS}$的数量级
\begin{itemize}
    \item 当MOS处于饱和区时,转移电导$g_m$随着$V_{GS}$的增加而增加。
    \item 当MOS处于线性区时,转移电导$g_m$不再随$V_{GS}$变化,保持恒定。
\end{itemize}
\begin{Figure}[饱和区转移电导]
    \includegraphics[scale=0.83]{build/Chapter02A_01q.fig.pdf}
\end{Figure}

由此可见,若器件进入线性区,跨导$g_m$会下降,故用于放大时通常令MOS工作在饱和区。


\subsection{输出电导}
\begin{BoxFormula}[MOS的输出电导]*
    输出电导$g_O$在线性区可以表示为
    \begin{Equation}&[A]
        \qquad\qquad
        g_O=\mu_nC_{ox}\frac{W}{L}[(V_{GS}-V_{TH})-V_{DS}+(V_{GS}-V_{TH})\lambda V_{DS}/2-3\lambda V_{DS}^2/2]
        \qquad\qquad
    \end{Equation}
    输出电导$g_O$在饱和区可以表示为
    \begin{Equation}&[B]
        g_O=\mu_nC_{ox}\frac{W}{L}[(V_{GS}-V_{TH})^2\lambda/2]
    \end{Equation}
    特别的,当$\lambda=0$时,在深线性区($V_{DS}\approx 0$)的输出电导为
    \begin{Equation}&[C]
        g_O=\mu_nC_{ox}\frac{W}{L}(V_{GS}-V_{TH})
    \end{Equation}
    特别的,当$\lambda= 0$时,在饱和区($V_{DS}>V_{GS}-V_{TH}$)的输出电导为
    \begin{Equation}
        g_O=0
    \end{Equation}
    特别的,当$\lambda\neq 0$时,在饱和区($V_{DS}>V_{GS}-V_{TH}$)的输出电导为
    \begin{Equation}
        g_O=\frac{\lambda I_D}{1+\lambda V_{DS}}\approx\lambda I_D
    \end{Equation}
\end{BoxFormula}

这里可以观察到一个相当有趣的规律,\empx{饱和区转移电导等于深线性区输出电导}
\begin{Equation}
    g_m|_\te{饱和区}=g_O|_\te{深线性区}
\end{Equation}
关于深线性区,由于$I_D$与$V_{DS}$的关系是线性的,此时小信号电阻和电阻的意义是一致的,换言之,MOS沟道在深线性区可以视为一个栅压控制的可变电阻,\xref{fig:深线性区输出电导}展示了$g_O$和$r_O$。
\begin{Figure}[深线性区输出电导]
    \begin{FigureSub}[输出电导;深线性区输出电导s]
        \includegraphics[scale=0.83]{build/Chapter02A_01o.fig.pdf}
    \end{FigureSub}
    \begin{FigureSub}[输出电阻;深线性区输出电阻s]
        \includegraphics[scale=0.83]{build/Chapter02A_01p.fig.pdf}
    \end{FigureSub}
\end{Figure}

在深线性区,栅压越大,MOS沟道的电阻越小(栅压能使沟道导通程度增加)。

关于饱和区,若$\lambda=0$不考虑沟道调制,此时输出电导$g_O=0$是恒定于零的,换言之,沟道是完全不导通是理想的电流源。因此,在饱和区,输出电导$g_O$或输出电阻$r_O$比较有意义的讨论只能建立在考虑沟道调制$\lambda\neq 0$的情况下(这也是为何转移电导和输出电导的讨论需放在沟道调制效应引入后)。请注意!饱和区下$r_O$是数百$\si{k\ohm}$,深线性区下$r_O$是数十$\si{k\ohm}$。

\begin{Figure}[饱和区输出电导]
    \begin{FigureSub}[输出电导;饱和区输出电导s]
        \includegraphics[scale=0.83]{build/Chapter02A_01r.fig.pdf}
    \end{FigureSub}
    \begin{FigureSub}[输出电阻;饱和区输出电阻s]
        \includegraphics[scale=0.83]{build/Chapter02A_01s.fig.pdf}
    \end{FigureSub}
\end{Figure}

饱和区的$g_m$和$r_O$是模拟集成电路中的两个重要参数,关注\xref{fig:饱和区转移电导}和\xref{fig:饱和区输出电阻s}
\begin{itemize}
    \item 随着$V_{GS}$增大,转移电导$g_m$增大,输出电阻$r_O$减小。
    \item 随着$V_{GS}$减小,转移电导$g_m$减小,输出电阻$r_O$增大。
\end{itemize}
通常,我们希望,转移电导$g_m$要尽可能大,输出电阻$r_O$要尽可能小,这两者需要权衡。