\section{亚阈值导电}
\uwave{亚阈值导电}(Subthreshold conduction)是指,当MOS的$V_{GS}<V_{TH}$时,沟道实际不会立即关断(仍然存在一个“弱反型层”),$I_D$也并非零,而是与$V_{GS}$呈现指数关系,经验公式为
\begin{Equation}
    I_D=\alpha\frac{W}{L}\exp\frac{V_{GS}-V_{TH}}{\xi V_T}
\end{Equation}
而我们知道,在饱和区(这里假定$V_{DS}$足够大使器件不处于线性区),$I_D$与$V_{GS}$呈平方关系
\begin{Equation}
    I_D=\mu_n C_{ox}\frac{W}{L}[\qty(V_{GS}-V_{TH})^2/2]
\end{Equation}

而现在的问题是,$V_{GS}$在增大到多少时,器件会由亚阈值区过渡到饱和区?这看上去是一个相当复杂的问题,但有一个巧妙的定义是:定义亚阈值区和饱和区的界限是两者跨导相等处。

\begin{BoxFormula}[亚阈值跨导]
    当器件处于亚阈值区时,跨导可以表示为
    \begin{Equation}
        g_m=\frac{I_D}{\xi V_{T}}
    \end{Equation}
    当器件处于饱和区时,跨导可以表示为
    \begin{Equation}
        g_m=\frac{2I_D}{V_{GS}-V_{TH}}
    \end{Equation}
\end{BoxFormula}

依照定义,界限处的$V_{GS}$应当满足
\begin{Equation}
    \frac{I_D}{\xi V_T}=\frac{2I_D}{V_{GS}-V_{TH}}
\end{Equation}
约去$I_D$,得到
\begin{Equation}
    \frac{1}{\xi V_T}=\frac{2}{V_{GS}-V_{TH}}
\end{Equation}
即
\begin{Equation}
    V_{GS}=V_{TH}+2\xi V_{T}
\end{Equation}
由此我们就得到了亚阈值区和饱和区的分界线,即$V_{GS}=V_{TH}+2\xi V_{T}$,整理电流的结论
\begin{BoxFormula}[亚阈值电流]
    当$V_{GS}<V_{TH}+2\xi V_{T}$,器件处于亚阈值区,电流适用指数规律
    \begin{Equation}
        I_D=\alpha\frac{W}{L}\exp\frac{V_{GS}-V_{TH}}{\xi V_T}
    \end{Equation}
    当$V_{GS}>V_{TH}+2\xi V_{T}$,器件处于饱和区,电流适用平方规律
    \begin{Equation}
        I_D=\mu_n C_{ox}\frac{W}{L}[(V_{GS}-V_{TH})^2/2]
    \end{Equation}
\end{BoxFormula}
这里$\xi$是一个非理想因子,通常取$\xi=1.5$,而$\alpha$是一个与器件特性有关的独立参数。但现在的问题是,分界点$V_{GS}=V_{TH}+2\xi V_T$保证了跨导$g_m$的连续性,却未能保证电流$I_D$的连续性,而$I_D$中还包含独立参数$\alpha$。再一次,出于建模的目的,我们抛开事实不谈,将$\alpha$作为非独立参数,其取值应当使$I_D$在分界点$V_{GS}=V_{TH}+2\xi V_T$连续,由此,得到方程
\begin{Equation}
    \alpha\frac{W}{L}\exp\frac{2\xi V_{T}}{\xi V_T}=2\mu_nC_{ox}\frac{W}{L}(\xi V_{T})^2
\end{Equation}
即
\begin{Equation}
    \alpha=2\mu_nC_{ox}(\xi V_T)^2/
    \e^2
\end{Equation}
基于此,我们就可以连续的绘制出$I_D$的图像了,如\xref{fig:亚阈值电流}所示。我们注意到,亚阈值电流是如此之小,以至于在线性坐标下完全看不出区别,但在对数坐标下,亚阈值电流就很明显了。

\begin{Figure}[亚阈值电流]
    \begin{FigureSub}[线性坐标连续曲线]
        \includegraphics[scale=0.82]{build/Chapter02A_01g.fig.pdf}
    \end{FigureSub}
    \begin{FigureSub}[对数坐标连续曲线]
        \includegraphics[scale=0.82]{build/Chapter02A_01h.fig.pdf}
    \end{FigureSub}\\ \vspace{0.5cm}
    \begin{FigureSub}[线性坐标分区曲线]
        \includegraphics[scale=0.82]{build/Chapter02A_01i.fig.pdf}
    \end{FigureSub}
    \begin{FigureSub}[对数坐标分区曲线]
        \includegraphics[scale=0.82]{build/Chapter02A_01j.fig.pdf}
    \end{FigureSub}
\end{Figure}

在\xref{fig:对数坐标连续曲线}中,我们也可以看到,当进入亚阈值区时,电流$I_D$大约在\xnum{1e-6}{A}以下。