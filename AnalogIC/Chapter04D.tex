\section{共模特性的定量分析}
本节我们将定量分析差分放大器的共模特性,差分放大器的一个重要特点就是其对共模扰动影响的抑制能力,这里我们要明确的是,共模扰动的影响事实上表现在两个方面
\begin{itemize}
    \item 输入共模的变化,导致了输出共模的变化,这会导致偏置点的变化。
    \item 输入共模的变化,导致了输出差模的变化,这是更为严重的,破坏了差分原理。
\end{itemize}
前者已经在\xref{def:差模增益}给出了,后者则需要一个新的定义
\begin{BoxDefinition}[共模--差模增益]
    共模--差模增益$A_{CM-DM}$,差模输出电压$V_{out,DM}$对共模输入电压$V_{in,CM}$的偏导
    \begin{Equation}
        A_{CM-DM}=\pdv{V_{out,DM}}{V_{in,CM}}
    \end{Equation}
\end{BoxDefinition}

然而,幸运的是,对于理想差分放大电路而言,共模增益$A_{CM}$和共模--差模增益$A_{CM-DM}$都是零。然而,实际的差分放大电路存在诸多不理想的因素,这主要包含以下两个方面
\begin{itemize}
    \item 差分电路的尾电流源$I_{SS}$是非理想的,存在一个电阻$R_{SS}$与之并联。
    \item 差分电路未必完全对称,分别可能有$R_{D1}\neq R_{D1}$和$g_{m1}\neq g_{m2}$的不对称。
\end{itemize}
我们先给出结论
\begin{itemize}
    \item 当$R_{SS}$存在时,将仅出现共模增益$A_{CM}$。
    \item 当$R_{SS}$存在且电路不对成,将出现共模增益$A_{CM}$和共模--差模增益$A_{CM-DM}$。
\end{itemize}

\begin{Figure}[尾电流源非理想时的差分放大器]
    \includegraphics[scale=0.8]{build/Chapter04D_01.fig.pdf}
\end{Figure}

接下来,我们将分别就这些情况进行定量分析。

\subsection{当尾电流源非理想时的共模增益}
现在我们分析\xref{fig:尾电流源非理想时的等效电路},这听上去可能很困难,但是介于这里分析共模特性,事情可以得到简化。

\begin{Figure}[尾电流源非理想时的等效电路]
    \includegraphics[scale=0.8]{build/Chapter04D_02.fig.pdf}
\end{Figure}

\begin{itemize}
    \item 输出$V_{out1}=V_{out2}=V_{out,CM}$,意味着两个输出端事实上短接了,故$V_{DD}$至$V_{out,CM}$两条路径上的两个电阻$R_{D1},R_{D2}$就相当于并联了,等效电阻为$R_{D1}\parallel R_{D2}=R_D/2$。
    \item 输出$V_{out1}=V_{out2}=V_{out,CM}$和输入$V_{in1}=V_{in2}=V_{in,CM}$,意味着$M_1,M_2$并联了,两者接受相同的栅源电源$V_{in,CM}-V_P$在两个相同的端点$V_{out,CM}$和$V_P$间产生相同的电流,并联时电流应相加,因此这里$M_1,M_2$管可以等效为一个跨导为$2g_m$的MOS管。
\end{itemize}
这样就得到了\xref{fig:尾电流源非理想时的等效电路},此时问题回到了带源极负反馈的共源放大,引用\xref{fml:源极负反馈下的共源增益}
\begin{BoxFormula}[差分放大器的共模增益]
    差分放大器的共模增益为
    \begin{Equation}
        A_{CM}=-\frac{R_D/2}{R_{SS}+(2g_m)^{-1}}
    \end{Equation}
\end{BoxFormula}

\subsection{当电阻不对称时的共模--差模增益}
在这一小节,我们分析电阻不对称,即$R_{D1}\neq R_{D2}$时的共模--差模增益$A_{CM-DM}$。

根据\xref{fig:尾电流源非理想时的等效电路},我们仍然将$M_1,M_2$管等效为一个跨导为$2_{gm}$的MOS管,由于小信号下$I_{SS}$可以视为开路,这里$V_{in,CM}$至$V_P$可以视为一个共漏放大器,而依据\xref{fml:采用电阻负载的共漏极增益}\setpeq{电阻不对称}
\begin{Equation}&[1]
    V_P=\frac{2g_mR_{SS}}{1+2g_{m}R_{SS}}V_{in,CM}
\end{Equation}
我们注意到
\begin{Equation}&[2]
    V_{in,CM}-V_P=\frac{1}{1+2g_mR_{SS}}V_{in,CM}
\end{Equation}
计算上式的目的是为了计算电流
\begin{Equation}&[3]
    I_{D1}=I_{D2}=g_m(V_{in,CM}-V_P)=\frac{g_m}{1+2g_mR_{SS}}V_{in,CM}
\end{Equation}
然而,电流相同,由于$R_{D1}\neq R_{D2}$,输出电压$V_{out1},V_{out2}$是不同的。

$V_{out1}$为
\begin{Equation}&[4]
    V_{out1}=-I_{D1}R_{D1}=-\frac{g_mR_{D1}}{1+2g_mR_{SS}}V_{in,CM}
\end{Equation}

$V_{out2}$为
\begin{Equation}&[5]
    V_{out2}=-I_{D2}R_{D2}=-\frac{g_mR_{D2}}{1+2g_mR_{SS}}V_{in,CM}
\end{Equation}

将\xrefpeq{4}和\xrefpeq{5}相减
\begin{Equation}
    V_{out}=V_{out1}-V_{out2}=-\frac{g_m(R_{D1}-R_{D2})}{1+2g_mR_{SS}}V_{in,CM}
\end{Equation}

容易得到
\begin{Equation}
    A_{CM-DM}=\frac{V_{out}}{V_{in,CM}}=-\frac{g_m(R_{D1}-R_{D2})}{1+2g_mR_{SS}}
\end{Equation}
这告诉我们,当电阻不对称时,共模--差模增益$A_{CM-DM}$正比于电阻的差值。

\begin{Figure}[当电阻不对称时差分放大器的特性]
    \begin{FigureSub}[共模电压;共模电压电阻]
        \includegraphics[scale=0.58]{build/Chapter04A_02_0.fig.pdf}
    \end{FigureSub} \hspace{0.3cm}
    \begin{FigureSub}[差模电压;差模电压电阻]
        \includegraphics[scale=0.58]{build/Chapter04A_02_1.fig.pdf}
    \end{FigureSub}\\ \vspace{0.1cm}
    \begin{FigureSub}[共模增益;共模增益电阻]
        \includegraphics[scale=0.58]{build/Chapter04A_02_3.fig.pdf}
    \end{FigureSub}
    \begin{FigureSub}[共模--差模增益;共模差模增益电阻]
        \includegraphics[scale=0.58]{build/Chapter04A_02_2.fig.pdf}
    \end{FigureSub}
\end{Figure}
\xref{fig:当电阻不对称时差分放大器的特性}展现了电阻不对称时差分放大器的特性,其中,绘图使用的参数如下
\begin{framed}
    \begin{Equation}*
        V_{DD}=4\si{V}\quad
        R_{D1}=15\si{k\ohm}\quad 
        R_{D2}=5\si{k\ohm}\quad 
        I_{SS}=0.3\si{mA}\quad 
        R_{SS}=40\si{k\ohm}
    \end{Equation}
\end{framed}

\begin{BoxFormula}[共模--差模增益(电阻不对称)]
    当电阻不对称时,共模--差模增益可以表示为
    \begin{Equation}
        A_{CM-DM}=-\frac{g_m(R_{D1}-R_{D2})}{1+2g_mR_{SS}}
    \end{Equation}
\end{BoxFormula}

% 我们可以看出,尾电流源的并联电阻$R_{SS}$是导致共模增益$A_{CM}$的主因,而共模0

\subsection{当跨导不对称时的共模--差模增益}
在这一小节,我们分析跨导不对称,即$g_{m1}\neq g_{m2}$时的共模--差模增益$A_{CM-DM}$。\setpeq{跨导不对称}

$I_{D1}$可以表示为
\begin{Equation}&[1]
    I_{D1}=g_{m1}(V_{in,CM}-V_P)
\end{Equation}
$I_{D2}$可以表示为
\begin{Equation}&[2]
    I_{D2}=g_{m2}(V_{in,CM}-V_P)
\end{Equation}
在小信号下$I_{SS}$是开路的,因此$V_{P}$的电压等于电流$I_{D1}+I_{D2}$流过$R_{SS}$
\begin{Equation}&[3]
    V_P=(I_{D1}+I_{D2})R_{SS}
\end{Equation}
将\xrefpeq{1}和\xrefpeq{2}代入\xrefpeq{3}
\begin{Equation}&[4]
    V_P=(V_{in,CM}-V_P)(g_{m1}+g_{m2})R_{SS}
\end{Equation}
展开
\begin{Equation}&[5]
    V_P=V_{in,CM}(g_{m1}+g_{m2})R_{SS}-V_P(g_{m1}+g_{m2})R_{SS}
\end{Equation}
我们的目的是解出$V_P$
\begin{Equation}&[6]
    V_P=\frac{(g_{m1}+g_{m2})R_{SS}}{1+(g_{m1}+g_{m2})R_{SS}}V_{in,CM}
\end{Equation}
注意到
\begin{Equation}&[7]
    V_{in,CM}-V_P=\frac{1}{1+(g_{m1}+g_{m2})R_{SS}}V_{in,CM}
\end{Equation}

$V_{out1}$可以表示为,代入\xrefpeq{1}和\xrefpeq{7}
\begin{Equation}
    V_{out1}=-I_{D1}R_{D}=\frac{-g_{m1}R_D}{1+(g_{m1}+g_{m2})R_{SS}}V_{in,CM}
\end{Equation}
$V_{out2}$可以表示为,代入\xrefpeq{2}和\xrefpeq{7}
\begin{Equation}
    V_{out2}=-I_{D2}R_{D}=\frac{-g_{m2}R_D}{1+(g_{m1}+g_{m2})R_{SS}}V_{in,CM}
\end{Equation}
相减得到
\begin{Equation}
    V_{out}=-\frac{(g_{m1}-g_{m2})R_{D}}{1+(g_{m1}+g_{m2})R_{SS}}V_{in,CM}
\end{Equation}
容易得到
\begin{Equation}
    A_{CM-DM}=\frac{V_{out}}{V_{in,CM}}=-\frac{(g_{m1}-g_{m2})R_{D}}{1+(g_{m1}+g_{m2})R_{SS}}
\end{Equation}
这告诉我们,当跨导不对称时,共模--差模增益$A_{CM-DM}$正比于跨导的差值。

\begin{BoxFormula}[共模--差模增益(跨导不对称)]
    当跨导不对称时,共模--差模增益可以表示为
    \begin{Equation}
        A_{CM-DM}=-\frac{(g_{m1}-g_{m2})R_{D}}{1+(g_{m1}+g_{m2})R_{SS}}
    \end{Equation}
\end{BoxFormula}

\begin{Figure}[当跨导不对称时差分放大器的特性]
    \begin{FigureSub}[共模电压;共模电压跨导]
        \includegraphics[scale=0.58]{build/Chapter04A_03_0.fig.pdf}
    \end{FigureSub} \hspace{0.3cm}
    \begin{FigureSub}[差模电压;差模电压跨导]
        \includegraphics[scale=0.58]{build/Chapter04A_03_1.fig.pdf}
    \end{FigureSub}\\ \vspace{0.1cm}
    \begin{FigureSub}[共模增益;共模增益跨导]
        \includegraphics[scale=0.58]{build/Chapter04A_03_3.fig.pdf}
    \end{FigureSub}
    \begin{FigureSub}[共模--差模增益;共模差模增益跨导]
        \includegraphics[scale=0.58]{build/Chapter04A_03_2.fig.pdf}
    \end{FigureSub}
\end{Figure}

\xref{fig:当跨导不对称时差分放大器的特性}展现了跨导不对称时差分放大器的特性,其中,绘图使用的参数如下
\begin{framed}
    \begin{Equation}*
        V_{DD}=4\si{V}\quad
        (W/L)_1=2.0\quad
        (W/L)_1=0.5\quad
        I_{SS}=0.3\si{mA}\quad 
        R_{SS}=40\si{k\ohm}
    \end{Equation}
\end{framed}

\subsection{共模抑制比}
很有趣的一个问题是,对于一个差分放大器,共模输入会多大程度上影响差模输出?为此,我们在这里引入了共模抑制比CMRR的概念,它被定义为$A_{DM}$和$A_{CM-DM}$的比的绝对值。
\begin{BoxDefinition}[共模抑制比]
    共模抑制比(Common Mode Rejection Ratio,CMRR)定义为
    \begin{Equation}
        \te{CMRR}=\abs{\frac{A_{DM}}{A_{CM-DM}}}
    \end{Equation}
\end{BoxDefinition}