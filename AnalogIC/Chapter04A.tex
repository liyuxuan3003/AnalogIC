\section{差分的基本思想}
噪声,噪声,我们总是在讨论噪声!但到底这些噪声是哪里来的?而为什么差分可以抑制噪声?本节将会具体讨论这两个问题。在正式开始之前,我们先来明确一下差分信号的定义。

差分信号是指,两个结点的电位$V_{1},V_{2}$相较于某个固定电位的差,大小相等,相位相反
\begin{itemize}
    \item 这两个结点的均值,即参照的固定电位,称为\uwave{共模}(Common Mode, CM)。
    \item 这两个结点的差值,称为\uwave{差模}(Differential Mode, DM)。
\end{itemize}
若$V_{1},V_{2}$是一对差分信号,共模可以表示为(均值)
\begin{Equation}
    V_{CM}=(V_1+V_2)/2
\end{Equation}
若$V_{1},V_{2}$是一对差分信号,差模可以表示为(差值)
\begin{Equation}
    V_{DM}=V_1-V_2
\end{Equation}
实践中,共模$V_{CM}$通常是偏置,差模$V_{DM}$则是信号。符号约定如下,例如一对差分的输入是$V_{in1},V_{in2}$,共模记为$V_{in,CM}$,差模记为$V_{in,DM}$,由于差模代表了信号,故用$V_{in}$简记。

当共模和差模已知时,差分信号可以反过来被表示为
\begin{Equation}
    V_{1}=V_{CM}+V_{DM}/2\qquad
    V_{2}=V_{CM}-V_{DM}/2
\end{Equation}
这就是差分信号的定义,那么为什么差分可以抑制噪声?这就要从噪声的来源说起。

试想,如\xref{fig:单端传输}所示,现在我们有一根信号线传输$V_1$,然而,不幸的是,其旁边恰好有一路时钟信号CK,照常理两者是互不影响的,但根据大学物理的知识,两根平行导线间是存在电容的,电容的特性是隔直通交,当CK经过上升沿和下降沿时,这相当于一个相当高频的变化,因此能通过电容对$V_1$造成串扰,在$V_1$原信号上叠加了一个向上或向下的噪声脉冲。这种导线间的寄生电容就噪声的一个重要来源!那差分是如何对抗这种噪声的干扰的呢?现在我们引入一个和$V_1$相位相反的信号$V_2$,如\xref{fig:差分传输}所示,CK与信号线$V_1$和$V_2$间都存在寄生电容,CK会在$V_1$和$V_2$上产生相同的噪声,然而,作为差分传输,我们的信号存在于与两者的差模$V_{DM}=V_1-V_2$上,共模噪声不会影响差模信号,这就是差分抑制噪声的原因。

\begin{Figure}[噪声来源和差分对噪声的抑制]
    \begin{FigureSub}[单端传输]
        \includegraphics[scale=0.8]{build/Chapter04A_05.fig.pdf}
    \end{FigureSub}
    \hspace{0.75cm}
    \begin{FigureSub}[差分传输]
        \includegraphics[scale=0.8]{build/Chapter04A_04.fig.pdf}
    \end{FigureSub}
\end{Figure}